\documentclass{article}
\usepackage{style}
\title{Frontend Cheatsheet}
\author{Hanhee Lee}
\lhead{ECE358}
\rhead{Hanhee Lee}

\begin{document}
\maketitle

\tableofcontents

\listoffigures

\listoftables

\section{CSS}
\subsection{Postion}
\begin{definition}
    Determines how an HTML element is positioned within its containing element or overall website.
    \begin{itemize}
        \item \textbf{Static} - Normal
        \item \textbf{Relative} - Shifts the element from its normal position
        \item \textbf{Fixed} - Fixed even when the page is scrolled
        \item \textbf{Absolute} - Element moves independently of other elements
        \item \textbf{Sticky} - Element is normal but becomes fixed when it reaches a certain point
    \end{itemize}
\end{definition}

\subsection{Display}
\begin{definition}
    Determines how the element behaves in terms of layout and visibility within the document. Controls how the elements are displayed.
    \begin{itemize}
        \item \textbf{Block} - Takes up the full width available
        \item \textbf{Inline} - Takes up only as much width as necessary
        \item \textbf{Inline-block} - Combines the two
        \item \textbf{None} - Element is not displayed
        \item \textbf{Flex} - Allows for flexible box model
        \item \textbf{Grid} - Allows for grid layout
    \end{itemize}
\end{definition}

\begin{warning}
    Use FlexBox Froggy to learn how to use Flex. 
\end{warning}

\section{Tailwind CSS}
\subsection{Custom Styles}
\begin{definition}
    \begin{enumerate}
        \item \textbf{Square Brackets} - Used to apply custom styles to an element (good for single use).
        \item \textbf{Tailwind Config} - Used to add custom styles to the tailwind config file (good for global use).
    \end{enumerate}
\end{definition}

\subsection{Clean Code}
\begin{definition}
    \begin{itemize}
        \item Break your layouts into specific components.
        \item Use directives (ie. @) for longer list of classes.
        \begin{itemize}
            \item Base: Applied globally.
            \item Components: Applied to specific components.
            \item Utilities: Applied to specific elements.
        \end{itemize}
        \item Use component libraries for buttons, cards, etc.
    \end{itemize}
\end{definition}
\end{document}
