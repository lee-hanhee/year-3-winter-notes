\subsection{Graphs as General Structured Data}
\begin{definition}
    \begin{itemize}
        \item Nodes (Vertices): Attributes 
        \begin{itemize}
            \item e.g. Node identity, \# of neighbors, etc.
            \customFigure[0.25]{../Images/L12_0.png}{}
        \end{itemize}
        \item Edges (Link): Attributes and Directions
        \begin{itemize}
            \item e.g. edge identity, edge weight, etc.
            \customFigure[0.25]{../Images/L12_1.png}{}
        \end{itemize}
        \item Global (master node): Attributes
        \begin{itemize}
            \item e.g. \# of nodes, longest path, etc.
            \customFigure[0.25]{../Images/L12_2.png}{}
        \end{itemize}
    \end{itemize}
\end{definition}

\subsubsection{What Info Can Store in a Graph?}
\begin{notes} Add vector (embeddings) to each of the graph components.
    \begin{itemize}
        \item Nodes (Vertices): Embeddings
        \item Edges (Link): Attributes and Embeddings
        \begin{itemize}
            \item Undirected and directed edges
        \end{itemize}
        \item Global (master node): Embeddings
    \end{itemize}
    \customFigure[0.25]{../Images/L12_3.png}{}
\end{notes}
\newpage

\subsection{Examples of Graphs}
\begin{example}
    \begin{enumerate}
        \item Maps
        \item Code, algorithms, mathematical formulas, neural networks 
        \item Images: Pixels connected to their neighboring pixels ("Locality" inductive bias)
        \customFigure[0.5]{../Images/L12_4.png}{}
        \item Test: Chain of words or characters ("Sequential" inductive bias)
        \customFigure[0.5]{../Images/L12_5.png}{}
        \item Molecules: Atoms are nodes, covalent bonds are edges
        \customFigure[0.5]{../Images/L12_6.png}{}
        \item Social Networks: People can be nodes, and their interactions are edges. 
        \customFigure[0.5]{../Images/L12_7.png}{}
    \end{enumerate}
\end{example}
\newpage

\subsection{Graph Problems}
\begin{notes}
    \begin{itemize}
        \item Global tasks:
        \customFigure[0.5]{../Images/L12_8.png}{}
        \item Node tasks:
        \customFigure[0.5]{../Images/L12_9.png}{}
        \item Edge tasks:
        \customFigure[0.5]{../Images/L12_10.png}{}
    \end{itemize}
\end{notes}
\newpage

\subsubsection{Representing Graphs Numerically, Mathematically, Programmatically}
\begin{notes} Represent the connectivity of the graph.
    \begin{itemize}
        \item Numerically 
        \begin{itemize}
            \item Adjacency Matrix: $A \in \mathbb{R}^{N \times N}$
            \customFigure[0.5]{../Images/L12_11.png}{}
            \item Adjacency List: $A \in \mathbb{R}^{N \times E}$
            \customFigure[0.5]{../Images/L12_12.png}{}
        \end{itemize}
        \item Mathematically: Structured tensors $G = (X,E,A,U)$
        \item Programmatically: Graph data structure
        \customFigure[0.5]{../Images/L12_13.png}{}
    \end{itemize}
\end{notes}
\newpage

\subsection{GNN}
\begin{summary}
    \begin{center}
        \begin{tabular}{ll}
            \toprule
            \textbf{Concept} & \textbf{Description} \\
            \midrule
            \textbf{Prediction Methods} & \\
            Node to node & Take nodes and pass in our favorite classifer \\
            \multicolumn{2}{p{\linewidth}}{\begin{center}
                \customFigure[0.5]{../Images/L12_15.png}{}
                \vspace{-4em}
            \end{center}} \\
            Edge to node & Passing info from one attribute to another requires a pooling op. \\
            \multicolumn{2}{p{\linewidth}}{\begin{center}
                \customFigure[0.5]{../Images/L12_16.png}{}
                \vspace{-4em}
            \end{center}} \\
            Node to edges & Passing info from one attribute to another requires a pooling op. \\
            \multicolumn{2}{p{\linewidth}}{\begin{center}
                \customFigure[0.5]{../Images/L12_18.png}{}
                \vspace{-4em}
            \end{center}} \\
            \midrule
        \end{tabular}
    \end{center}
\end{summary}
\newpage

\begin{summary}
    \begin{center}
        \begin{tabular}{ll}
            \toprule
            \textbf{Concept} & \textbf{Description} \\
            \midrule
            \textbf{Pooling from Edges to Nodes} & Route info b/w diff. parts and aggregate them. \\
            \multicolumn{2}{p{\linewidth}}{\begin{center}
                \customFigure[0.5]{../Images/L12_17.png}{}
                \vspace{-4em}
            \end{center}} \\
            \midrule
            \textbf{Prediction Pipeline} & $\textbf{Graph} \rightarrow \text{Transform} \rightarrow \text{Predict on attribute of interest}$ \\
            \multicolumn{2}{p{\linewidth}}{\begin{center}
                \customFigure[0.5]{../Images/L12_19.png}{}
                \vspace{-4em}
            \end{center}} \\
        \end{tabular}
    \end{center}
\end{summary}
\newpage

\begin{summary}
    \begin{center}
        \begin{tabular}{ll}
            \toprule
            \textbf{Concept} & \textbf{Description} \\
            \midrule
            \textbf{Message passing} & Pool information, aggregate it, transform it, and update it \\
            \multicolumn{2}{p{\linewidth}}{\begin{center}
                \customFigure[0.5]{../Images/L12_20.png}{}
                \vspace{-4em}
            \end{center}} \\
            Propogation pattern of message passing & After a few layers we get increasingly more complex patterns of info. \\
            \multicolumn{2}{p{\linewidth}}{\begin{center}
                \customFigure[0.5]{../Images/L12_21.png}{}
                \vspace{-4em}
            \end{center}} \\
            \midrule
            \textbf{Conditioning Info} & Many ways of adding context to a specific part of a graph \\
            \multicolumn{2}{p{\linewidth}}{\begin{center}
                \customFigure[0.5]{../Images/L12_23.png}{}
                \vspace{-4em}
            \end{center}} \\
            \midrule
        \end{tabular}
    \end{center}
\end{summary}
\newpage

\subsection{Graph Topics}
\begin{summary}
    \begin{center}
        \begin{tabular}{ll}
            \toprule
            \textbf{Concept} & \textbf{Description} \\
            \midrule
            Learning subgraph representations & Learning a function that works on portions of a graph \\
            \multicolumn{2}{p{\linewidth}}{\begin{center}
                \customFigure[0.5]{../Images/L12_25.png}{}
                \vspace{-4em}
            \end{center}} \\
            \midrule
            Batching in graphs & Very context dependent, no general solution \\
            \multicolumn{2}{p{\linewidth}}{\begin{center}
                \customFigure[0.5]{../Images/L12_26.png}{}
                \vspace{-4em}
            \end{center}} \\
            \midrule
        \end{tabular}
    \end{center}
\end{summary}
\newpage

\begin{summary}
    \begin{center}
        \begin{tabular}{ll}
            \toprule
            \textbf{Concept} & \textbf{Description} \\
            \midrule
            Heterogeneous graphs & Different types of nodes and edges \\
            \multicolumn{2}{p{\linewidth}}{\begin{center}
                \customFigure[0.5]{../Images/L12_27.png}{}
                \vspace{-4em}
            \end{center}} \\
            \midrule
            Interpretability & Many ways to extract information from graphs \\
            \multicolumn{2}{p{\linewidth}}{\begin{center}
                \customFigure[0.5]{../Images/L12_28.png}{}
                \vspace{-4em}
            \end{center}} \\
            \midrule
            Attention & Transformers can be viewed as a GNN on a fully connected graph \\
            \multicolumn{2}{p{\linewidth}}{\begin{center}
                \customFigure[0.5]{../Images/L12_29.png}{}
                \vspace{-4em}
            \end{center}} \\
            \midrule
        \end{tabular}
    \end{center}
\end{summary}

\subsection{Examples}

\subsubsection{Message passing / GraphNets on a small graph}
\begin{example}
    \customFigure[0.5]{../Images/L12_24.png}{}
\end{example}
