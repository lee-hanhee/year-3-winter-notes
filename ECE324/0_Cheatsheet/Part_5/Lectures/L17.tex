\subsection{General Trick: Gradients as Importance}
\begin{definition}
    Use the gradient as a proxy for importance:
    \begin{equation*}
        \text{att} \approx \frac{dy}{dx} \cdot x
    \end{equation*}
    \customFigure[0.75]{../../Images/L16_7.png}{}
\end{definition}

\begin{warning}
    The shape of the gradient is the same as the shape of the input.
    \begin{itemize}
        \item $\frac{d\text{image}}{dy} \rightarrow \text{image-shaped gradient}$
    \end{itemize}
    \begin{itemize}
        \item $\frac{d\text{graph}}{dy} \rightarrow \text{graph-shaped gradient}$
    \end{itemize}
    \customFigure[0.75]{../../Images/L17_0.png}{}
\end{warning}

\subsection{Attribution Methods}
\begin{summary}
    \begin{center}
        \begin{tabular}{ll}
            \toprule
            \textbf{Method} & \textbf{Description} \\
            \midrule
            \textbf{Class Activation Mapping (CAM - Gradient-less)} & $L_{\text{CAM}}^c = \sum_k w_k^c A^k$ \\
            \multicolumn{2}{p{\linewidth}}{
            \begin{itemize}
                \item \textbf{What?} Produces a heatmap highlighting image regions important for a classification decision
                \item \textbf{How?} Weighting the feature maps of the final convolutional layer by the output layer weights.
                \begin{itemize}
                    \item $w_k^c$: weight of the $k$-th feature map for class $c$
                    \item $A^k$: $k$-th feature map (image-shaped)
                    \item $L_{\text{CAM}}^c$: class activation map for class $c$
                \end{itemize}
                
            \end{itemize}} \\
            \midrule
            \textbf{Integrated Gradients} & \\
            \multicolumn{2}{p{\linewidth}}{
            \begin{itemize}
                \item 
            \end{itemize}} \\
            \midrule
            \textbf{Gradient $\times$ Input} & \\
            \multicolumn{2}{p{\linewidth}}{
            \begin{itemize}
                \item 
            \end{itemize}} \\
            \midrule
            \textbf{GradCAM} & \\
            \multicolumn{2}{p{\linewidth}}{
            \begin{itemize}
                \item 
            \end{itemize}} \\
            \midrule
            \textbf{Smooth Grad} & \\
            \multicolumn{2}{p{\linewidth}}{
            \begin{itemize}
                \item 
            \end{itemize}} \\
            \bottomrule
        \end{tabular}
    \end{center}
\end{summary}

\subsection{Examples}
