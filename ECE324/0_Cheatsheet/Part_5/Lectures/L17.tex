\subsection{General Trick: Gradients as Importance}
\begin{definition}
    Use the gradient as a proxy for importance:
    \begin{equation*}
        \text{att} \approx \frac{dy}{dx} \cdot x
    \end{equation*}
    \customFigure[0.75]{../../Images/L16_7.png}{}
\end{definition}

\begin{warning}
    The shape of the gradient is the same as the shape of the input.
    \begin{itemize}
        \item $\frac{d\text{image}}{dy} \rightarrow \text{image-shaped gradient}$
    \end{itemize}
    \begin{itemize}
        \item $\frac{d\text{graph}}{dy} \rightarrow \text{graph-shaped gradient}$
    \end{itemize}
    \customFigure[0.5]{../../Images/L17_0.png}{}
\end{warning}

\subsubsection{Saliency Maps}
\begin{definition}
    Attribution method that highlights the most important regions in an input image for a model's prediction. 
\end{definition}

\subsection{Attribution Methods}
\begin{summary}
    \begin{center}
        \begin{tabular}{ll}
            \toprule
            \textbf{Method} & \textbf{Description} \\
            \midrule
            \textbf{Class Activation Mapping (CAM - Gradient-less)} & $L_{\text{CAM}}^c = \sum_k w_k^c A^k$ \\
            \multicolumn{2}{p{\linewidth}}{
            \begin{itemize}
                \item \textbf{What?} Produces a heatmap highlighting image regions important for a classification decision
                \item \textbf{How?} Weighting the feature maps of the final convolutional layer by the output layer weights.
                \begin{itemize}
                    \item $w_k^c$: weight of the $k$-th feature map for class $c$
                    \item $A^k$: $k$-th feature map (image-shaped)
                    \item $L_{\text{CAM}}^c$: class activation map for class $c$
                \end{itemize}
                \customFigure[0.6]{../../Images/L17_1.png}{}
            \end{itemize}} \\
            \midrule
            \textbf{Gradient $\times$ Input} & $\text{Grad} \times \text{Input} = \mathbf{x} \odot \nabla_{\mathbf{x}} F(\mathbf{x})$\\
            \multicolumn{2}{p{\linewidth}}{
            \begin{itemize}
                \item \textbf{What?} Feature importance via gradient magnitude
                \customFigure[0.6]{../../Images/L17_2.png}{}
            \end{itemize}} \\
            \bottomrule
        \end{tabular}
    \end{center}
\end{summary}
\newpage

\begin{summary}
    \begin{center}
        \begin{tabular}{ll}
            \toprule
            \textbf{Method} & \textbf{Description} \\
            \midrule
            \textbf{GradCAM} & $L_{\text{Grad-CAM}}^c = \text{ReLU} \left(\sum_k \alpha_k^c A^k \right)$\\
            \multicolumn{2}{p{\linewidth}}{
            \begin{itemize}
                \item \textbf{What?} Class-discriminative localization via gradients to highlight important regions in the input image for predicting a specific class. 
                \item \textbf{How?} Using gradients flowing into the final convolutional layer to produce a coarse localization map.
                \begin{itemize}
                    \item \textbf{Similar to CAM:} CAM with gradient information and ReLU activation
                    \item $\alpha_k^c = \frac{1}{Z} \sum_i \sum_j \frac{\partial y^c}{\partial A_{ij}^k}$: weight of the $k$-th feature map for class $c$
                    \item $A^k$: $k$-th feature map (image-shaped)
                \end{itemize}
                \customFigure[0.5]{../../Images/L17_3.png}{}
            \end{itemize}} \\
            \midrule
            \textbf{Smooth Grad} & $\tilde{G}(\mathbf{x}) = \frac{1}{N} \sum_{i=1}^N \nabla F(\mathbf{x} + \mathcal{N}(0,\sigma^2))$ \\
            \multicolumn{2}{p{\linewidth}}{
            \begin{itemize}
                \item \textbf{What?} Reduce noise in saliency maps using SmoothGrad 
                \item \textbf{How?} By averaging the gradients of the model's output with noise as well. 
                \customFigure[0.6]{../../Images/L17_4.png}{}
            \end{itemize}} \\
            \bottomrule
        \end{tabular}
    \end{center}
\end{summary}
\newpage

\begin{summary}
    \begin{center}
        \begin{tabular}{ll}
            \toprule
            \textbf{Method} & \textbf{Description} \\
            \midrule
            \textbf{Attention Maps + Gradients} & An attribution method that combines attention weights \\
            & with input gradients to estimate the contribution of each input token \\ 
            & to the model’s prediction \\
            \midrule
            \textbf{Integrated Gradients} & \textbf{Completeness:} $\sum_{i=1}^{d} \mathrm{IG}_i(\mathbf{x}) = F(\mathbf{x}) - F(\mathbf{x}')$ \\
            & \textbf{Linearity:} $\mathrm{IG}_{F+G}(\mathbf{x}) = \mathrm{IG}_F(\mathbf{x}) + \mathrm{IG}_G(\mathbf{x})$ \\
            & \textbf{Implementation Invariance:} $F'(\mathbf{x}) = F(\mathbf{x}) \Rightarrow \mathrm{IG}'(\mathbf{x}) = \mathrm{IG}(\mathbf{x})$ \\
            & \textbf{Sensitivity:} $\mathbf{x}_i - \mathbf{x}'_i \ne 0 \land F(\mathbf{x}) \ne F(\mathbf{x}') \Rightarrow \exists i \text{ s.t. } \mathrm{IG}_i(\mathbf{x}) \ne 0$ \\
            \multicolumn{2}{p{\linewidth}}{
            \begin{itemize}
                \item \textbf{What?} Decompose predictions into feature contributions by aiming to explain the contribution of each input feature to a model's prediction. 
                \item \textbf{How?} By accumulating gradients along a path from a baseline input to the actual input. 
                \customFigure[0.5]{../../Images/L17_5.png}{}
            \end{itemize}} \\
            \bottomrule
        \end{tabular}
    \end{center}
\end{summary}
