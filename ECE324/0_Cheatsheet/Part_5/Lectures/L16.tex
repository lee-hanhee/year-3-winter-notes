\begin{motivation}
    A single metric is an incomplete description of most real-world tasks. 
    \begin{enumerate}
        \item Improve models
        \item Justify models \\
        $
        \left.
        \begin{array}{l}
        \text{a. Creators} \\
        \text{b. Operators} \\
        \text{c. Executors} \\
        \text{d. Decision} \\
        \text{e. Auditors} \\
        \text{f. Data Subjects}
        \end{array}
        \right\} \text{Stakeholders of a AI System}
        $
        \item Discover Insights 
    \end{enumerate}
\end{motivation}

\subsection{What is Interpretability?}
\begin{definition}
    Interpretability is:
    \begin{enumerate}
        \item Degree to which a human can understand the cause of a decision
        \item Where a user can correctly and efficiently predict the method's results. 
        \item Science of understanding AI models from the inside out. 
    \end{enumerate}
\end{definition}

\subsubsection{Mechanistic Interpretability}
\begin{definition}
    Reverse engineering the algorithm of a NN. 
\end{definition}

\subsection{Types of Interpretability}
\begin{summary}
    \customFigure[0.5]{../../Images/L16_0.png}{}
    \begin{itemize}
        \item \textbf{ML Interpretability:} 
        \begin{itemize}
            \item \textbf{By-design:} Interpretability built directly into the model (e.g., decision trees, linear models).
            \item \textbf{Post-hoc:} Interpretability techniques are applied after the model is trained, without altering the model.
            \begin{itemize}
                \item \textbf{Model-agnostic:} Interpretation methods that can be applied to any model.
                \begin{itemize}
                    \item \textbf{Global:} Provides an overall understanding of the model’s behavior across the entire dataset.
                    \item \textbf{Local:} Explains the model’s prediction for a specific input instance.
                \end{itemize}
                \item \textbf{Model-specific:} Interpretation methods that are tailored to specific models.
            \end{itemize}
        \end{itemize}
    \end{itemize}    
\end{summary}
\newpage

\subsection{Attribution}
\begin{motivation}
    One tool in the interpretability toolkit
\end{motivation}

\begin{definition}
    Attribution techniques assign ranked importance values to parts of the input that relate to the output. 
\end{definition}

\subsubsection{Issues}
\begin{summary}
    \begin{center}
        \begin{tabular}{ll}
            \toprule
            \textbf{Issue} & \textbf{Description} \\
            \midrule
            \textbf{Spurious Correlations} & Correlations learned by a model that appear predictive in the training data, \\
            & but do not reflect true causal relationships in the real world. \\
            \midrule
            \textbf{Dataset Biases} & Systematic distortions in the training data \\
            & that misrepresent the underlying population or task, \\
            & leading to unfair or inaccurate model predictions. \\
            \midrule
            \textbf{Imperfect Model} & Models do not have perfect accuracy so attributions \\
            & are likely to not be perfectly accurate \\
            \bottomrule
        \end{tabular}
    \end{center}
\end{summary}
\newpage

\subsubsection{Tools}
\begin{summary}
    Use the gradient as a proxy for importance:
    \begin{equation*}
        \text{att} \approx \frac{dy}{dx} \cdot x
    \end{equation*}
    \customFigure[0.75]{../../Images/L16_7.png}{}
\end{summary}
\newpage

\subsection{Examples}
\subsubsection{Interpretability in LLMs}
\begin{example}
    \customFigure[0.75]{../../Images/L16_1.png}{}
    \begin{itemize}
        \item \textbf{LS:} Inputs feed into a neural network (NN), which then outputs some prediction. Interpretability is easier here.
        \item \textbf{LLM:} Interpretability is very hard for next token prediction so need to use mechanistic interpretability.
    \end{itemize}
\end{example}
\newpage

\subsubsection{Attribution for Scientific Discovery: Olfaction}
\begin{example}
    \begin{enumerate}
        \item \textbf{Overview:} Identifying mechanisms and patterns is at the heart of formulating a scientific hypothesis.
        \begin{itemize}
            \item \textbf{Olfaction:} Sense of smell from chemicals 
        \end{itemize}
        \item \textbf{Boelens' Rose Rule:} A chemical compound smells like rose if: 
        \begin{itemize}
            \item \textbf{Functional Group:} OH, OR or OCOR
            \item \textbf{Carbon Chain:} Carbon atoms
            \item \textbf{F:} Alpha-branched, unsaturated, or aromatic phenyl moiety. 
        \end{itemize}
        \customFigure[0.75]{../../Images/L16_2.png}{}
        \item \textbf{Problem:} Want to build attributions that can explain the rose rule.
        \item \textbf{Solution:} Easily build attributions with generalized linear models and bag of subgraphs for graphs (in a linear way).
        \customFigure[0.75]{../../Images/L16_3.png}{}
        \begin{itemize}
            \item \textbf{Molecular Fingerprints:} Molecular structures (graphs) are converted into vectors.
            \begin{itemize}
                \item \textbf{Bag of Subgraphs:} Each dimension encodes the presence of a subgraph.
            \end{itemize}
            \item \textbf{Model:} A GLM processes these fingerprint vectors to output a prediction (smell like rose or not).
            \begin{itemize}
                \item $W$: Learned weights 
                \item $x$: Molecular fingerprint vector
            \end{itemize}
            \item \textbf{Attribution:} \( att = W \cdot x \) provides a linear attribution score per subgraph, indicating its contribution to the prediction.
            \item \textbf{Interpretation:} Positive weights in \( W \) correlate with subgraphs associated with positive labels (e.g., rose scent).
        \end{itemize}        
        \newpage
        \item \textbf{Spurious Correlation Issue:} Statistical patterns in our dataset can affect the weights (and explanations) of our model
        \customFigure[0.75]{../../Images/L16_4.png}{}
        \begin{itemize}            
            \item \textbf{Spurious Correlation vs. Significant Subgraph:} Distinguishes between features that are 
            \begin{itemize}
                \item \textbf{Actual Correlation:} Significant subgraphs
                \item \textbf{Spurious Correlation:} Coincidentally correlated with the target label in the training data.
            \end{itemize}
        
            \item \textbf{Training Set:} GLM learns to associate both real and spurious features with the label "rose" during training (i.e. red subgraph and blue spurious correlation).
        
            \item \textbf{Test Set:} If the spurious correlations are not present in the test data (i.e. blue not present), the model may:
            \begin{itemize}
                \item fail to predict the correct label.
                \item provide misleading or weak attributions.
                \item still succeed due to overlapping structure but without reliable attribution.
            \end{itemize}
        \end{itemize}
        \item \textbf{Imperfect Model:}
        \customFigure[0.75]{../../Images/L16_5.png}{}
    \end{enumerate}
\end{example}
\newpage

\subsubsection{Spurious Correlation: Wolf vs. Dog}
\begin{example}
    \customFigure[0.75]{../../Images/L16_6.png}{}
    \begin{itemize}
        \item \textbf{Spurious Correlation:} Predicts wolf not because of the characteristics of the wolf, but because of the snow.
    \end{itemize}
\end{example}
