\begin{summary}
    \begin{center}
        \begin{tabular}{ll}
            \toprule
            \textbf{Concept} & \textbf{Description} \\
            \midrule
            \textbf{Message passing} & Pool information, aggregate information, \\
            & transform graph w/ $f$, and update graph \\
            \multicolumn{2}{p{\linewidth}}{\begin{center}
                \customFigure[0.75]{../Images/L12_20.png}{}
                \vspace{-4em}
            \end{center}} \\
            \midrule
            \textbf{Propogation pattern of message passing} & After a few layers we get $\uparrow$ more complex patterns of info. \\
            \multicolumn{2}{p{\linewidth}}{\begin{center}
                \customFigure[0.75]{../Images/L12_21.png}{}
                \vspace{-4em}
            \end{center}} \\
            \bottomrule
        \end{tabular}
    \end{center}
\end{summary}
\newpage

\subsection{Examples}
\subsubsection{Molecules}
\begin{example}
    \begin{enumerate}
        \item \textbf{Given:} Nodes (Atoms), Edges (Bonds), Globals (Properties)
        \customFigure[0.25]{../../Images/L12_2.png}{}
        \item \textbf{Task:} Predict the smell of the molecule. 
        \item \textbf{Message Passing:}
        \customFigure[0.75]{../../Images/L12_3.png}{}
        \item \textbf{Prediction:}
        \customFigure[0.75]{../../Images/L12_4.png}{}
    \end{enumerate}
\end{example}

\subsubsection{Message passing / GraphNets on a small graph}
\begin{example}
    \customFigure[0.75]{../Images/L12_24.png}{}
    \begin{itemize}
        \item \textbf{Input Graph:} 3 nodes (0, 1, 2) and 2 edges: $(0,1)$ and $(1,2)$.
        \begin{itemize}
            \item Nodes and edges are initialized with features that will be updated through message passing.
        \end{itemize}
    
        \item \textbf{Step 1 – Update Edges:} Messages are passed along edges.
        \begin{itemize}
            \item Edge attributes are updated based on the features of the connected nodes.
            \item Each edge (e.g., $0 \rightarrow 1$, $1 \rightarrow 2$) aggregates information from the connected nodes to compute new edge embeddings.
        \end{itemize}
    
        \item \textbf{Step 2 – Update Nodes:} Each node collects messages from incoming edges.
        \begin{itemize}
            \item Node features are updated by aggregating information from its neighbors
        \end{itemize}
    
        \item \textbf{Step 3 – Update Global State:} A global context vector is computed based on the set of updated node and edge embeddings.
    
        \item \textbf{Output Graph:} Graph with updated node, edge, and global features.
    \end{itemize}    
\end{example}
