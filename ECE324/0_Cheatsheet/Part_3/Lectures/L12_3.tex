\begin{summary}
    \begin{center}
        \begin{tabular}{ll}
            \toprule
            \textbf{Concept} & \textbf{Description} \\
            \midrule
            \textbf{Message passing} & Pool information, aggregate information, \\
            & transform graph w/ $f$, and update graph \\
            \multicolumn{2}{p{\linewidth}}{\begin{center}
                \customFigure[0.75]{../Images/L12_20.png}{}
                \vspace{-4em}
            \end{center}} \\
            \midrule
            \textbf{Propogation pattern of message passing} & After a few layers we get $\uparrow$ more complex patterns of info. \\
            \multicolumn{2}{p{\linewidth}}{\begin{center}
                \customFigure[0.75]{../Images/L12_21.png}{}
                \vspace{-4em}
            \end{center}} \\
            \midrule
            \textbf{Conditioning Info} & Many ways of adding context to a specific part of a graph \\
            \multicolumn{2}{p{\linewidth}}{\begin{center}
                \customFigure[0.5]{../Images/L12_23.png}{}
                \vspace{-4em}
            \end{center}} \\
            \midrule
        \end{tabular}
    \end{center}
\end{summary}
\newpage

\subsection{Graph Topics}
\begin{summary}
    \begin{center}
        \begin{tabular}{ll}
            \toprule
            \textbf{Concept} & \textbf{Description} \\
            \midrule
            Learning subgraph representations & Learning a function that works on portions of a graph \\
            \multicolumn{2}{p{\linewidth}}{\begin{center}
                \customFigure[0.5]{../Images/L12_25.png}{}
                \vspace{-4em}
            \end{center}} \\
            \midrule
            Batching in graphs & Very context dependent, no general solution \\
            \multicolumn{2}{p{\linewidth}}{\begin{center}
                \customFigure[0.5]{../Images/L12_26.png}{}
                \vspace{-4em}
            \end{center}} \\
            \midrule
        \end{tabular}
    \end{center}
\end{summary}
\newpage

\begin{summary}
    \begin{center}
        \begin{tabular}{ll}
            \toprule
            \textbf{Concept} & \textbf{Description} \\
            \midrule
            Heterogeneous graphs & Different types of nodes and edges \\
            \multicolumn{2}{p{\linewidth}}{\begin{center}
                \customFigure[0.5]{../Images/L12_27.png}{}
                \vspace{-4em}
            \end{center}} \\
            \midrule
            Interpretability & Many ways to extract information from graphs \\
            \multicolumn{2}{p{\linewidth}}{\begin{center}
                \customFigure[0.5]{../Images/L12_28.png}{}
                \vspace{-4em}
            \end{center}} \\
            \midrule
            Attention & Transformers can be viewed as a GNN on a fully connected graph \\
            \multicolumn{2}{p{\linewidth}}{\begin{center}
                \customFigure[0.5]{../Images/L12_29.png}{}
                \vspace{-4em}
            \end{center}} \\
            \midrule
        \end{tabular}
    \end{center}
\end{summary}
\newpage

\subsection{Examples}
\subsubsection{Molecules}
\begin{example}
    \begin{enumerate}
        \item \textbf{Given:} Nodes (Atoms), Edges (Bonds), Globals (Properties)
        \customFigure[0.75]{../../Images/L12_2.png}{}
        \item \textbf{Task:} Predict the smell of the molecule. 
        \item \textbf{Message Passing:}
        \customFigure[0.75]{../../Images/L12_3.png}{}
        \item \textbf{Prediction:}
        \customFigure[0.75]{../../Images/L12_4.png}{}
    \end{enumerate}
\end{example}
\subsubsection{Message passing / GraphNets on a small graph}
\begin{example}
    \customFigure[0.75]{../../Images/L12_24.png}{}
\end{example}
