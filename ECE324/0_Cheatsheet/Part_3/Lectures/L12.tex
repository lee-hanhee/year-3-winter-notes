\begin{definition} Graphs consists of nodes, edges, and globals. 
    \begin{itemize}
        \item Nodes (Vertices): Attributes 
        \begin{itemize}
            \item e.g. Node identity, \# of neighbors, etc.
            \customFigure[0.25]{../Images/L12_0.png}{}
        \end{itemize}
        \item Edges (Link): Attributes and Directions
        \begin{itemize}
            \item e.g. edge identity, edge weight, etc.
            \customFigure[0.25]{../Images/L12_1.png}{}
        \end{itemize}
        \item Global (master node): Attributes
        \begin{itemize}
            \item e.g. \# of nodes, longest path, etc.
            \customFigure[0.25]{../Images/L12_2.png}{}
        \end{itemize}
    \end{itemize}
\end{definition}

\subsection{What Info Can Store in a Graph?}
\begin{notes} Add vector (embeddings) to each of the graph components.
    \customFigure[0.75]{../../Images/L12_0.png}{}
\end{notes}
\newpage

\subsection{Examples of Graphs}
\begin{example}
    \begin{enumerate}
        \item Maps
        \item Code, algorithms, mathematical formulas, neural networks 
        \item Images: Pixels connected to their neighboring pixels ("Locality" inductive bias)
        \customFigure[0.5]{../Images/L12_4.png}{}
        \item Test: Chain of words or characters ("Sequential" inductive bias)
        \customFigure[0.5]{../Images/L12_5.png}{}
        \item Molecules: Atoms are nodes, covalent bonds are edges
        \customFigure[0.5]{../Images/L12_6.png}{}
        \item Social Networks: People can be nodes, and their interactions are edges. 
        \customFigure[0.5]{../Images/L12_7.png}{}
    \end{enumerate}
\end{example}