\begin{notes}
    \begin{itemize}
        \item Transformers on large text-like datasets.
        \item Transformers on "tokens" (discretized data)
        \item Foundational models
    \end{itemize}
\end{notes}

\subsection{Inputs and Outputs of LLMs}
\begin{notes}
    \begin{itemize}
        \item \textbf{Inputs:} Tokenizing text and embedding layers
        \customFigure[0.6]{../../Images/L15_1.png}{}
        \vspace{-1em}
        \begin{itemize}        
            \item \textbf{Text to Tokens:} The input text is split into subword chunks using a tokenizer.
            \item \textbf{Token to IDs:} Each token is mapped to a unique integer ID based on a fixed vocabulary.
            \item \textbf{Embedding Lookup:} Token IDs are passed through an embedding layer, producing vector representations of the tokens.
            \item \textbf{Text Embeddings:} The output of the embedding layer is a sequence of learned vectors (color-coded), one for each token, which serve as input to the Transformer model.
        \end{itemize}        
        \item \textbf{Outputs:} Auto-regressive decoding of tokens (one at a time) using previous outputs
        \customFigure[0.6]{../../Images/L15_2.png}{}
        \vspace{-1em}
        \begin{itemize}
            \item \textbf{Auto-Regressive Decoding:} Predicts each output token sequentially. At time step $t$, the decoder generates the $t^\text{th}$ token using previously generated outputs from steps $1$ to $t-1$.
        \end{itemize}        
    \end{itemize}
\end{notes}
\newpage

\subsection{Transformers \& LLMs}
\begin{summary}
    \begin{center}
        \begin{tabular}{ll}
            \toprule
            \textbf{Concept} & \textbf{Description} \\
            \midrule
            Sizes of text datasets & Scale significantly impacts LLM's performance. \\
            \midrule
            Text to Text Tasks & Many tasks can be framed as text-to-text problems. \\
            \midrule
            Transformers \& Masking: Encoders \& Decoders & Transfomers fit the encoder decoder paradigm. \\
            \multicolumn{2}{p{\linewidth}}{
            \begin{itemize}
                \customFigure[0.5]{../../Images/L15_0.png}{}
                \vspace{-1em}
                \item \textbf{T5 (Encoder-Decoder):}
                \begin{itemize}
                    \item \textbf{Encoder:} Input tokens $x_1, x_2, x_3, x_4$ are processed by the encoder.
                    \begin{itemize}
                        \item \textbf{Fully-visible masking} is used from the encoder to the decoder (dark grey lines).
                    \end{itemize}
                    \item \textbf{Decoder:} Decoder attends to all encoder outputs and previously generated outputs $y_1, y_2$.
                    \begin{itemize}
                        \item \textbf{Causal masking} to prevent attending to future outputs.
                    \end{itemize}
                \end{itemize}
            
                \item \textbf{GPT-3 (Decoder-only Language Model):}
                \begin{itemize}
                    \item Inputs consist of all prior tokens ($x_1, x_2, x_3, y_1, y_2, \ldots$).
                    \item \textbf{Causal masking} throughout to ensure autoregressive behavior—i.e., each token only attends to previous tokens.
                \end{itemize}
            
                \item \textbf{Attention Mask Patterns:} Dark squares denote visible connections; light squares denote masked (inaccessible) tokens.
                \begin{itemize}
                    \item \textbf{Fully-visible mask:} Every token can attend to every other token 
                    \item \textbf{Causal mask:} Tokens can only attend to previous or current tokens 
                \end{itemize}
            \end{itemize}} \\
            \midrule 
            Masked Language Modelling & Predicting masked tokens in a sequence. \\
            \multicolumn{2}{p{\linewidth}}{
            \begin{itemize}
                \customFigure[0.6]{../../Images/L15_3.png}{}
                \vspace{-1em}
                \item \textbf{Original Text:} Selected spans (e.g., "for inviting", "last") are masked and replaced with sentinel tokens.
                \item \textbf{Inputs:} The corrupted version of the sentence is given as input to the model:
                \begin{itemize}
                    \item Tokens \texttt{<X>} and \texttt{<Y>} indicate the location and order of the masked spans.
                \end{itemize}
                \item \textbf{Targets:} The decoder is trained to predict the concatenated masked spans, each preceded by its corresponding token:
            \end{itemize}} \\
            \midrule
        \end{tabular}
    \end{center}
\end{summary}