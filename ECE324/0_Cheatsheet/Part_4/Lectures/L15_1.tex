\begin{motivation}
    Engineering LLMs for massive computation and dataset ingestion. 
\end{motivation}

\subsection{Matrix Multiplication and FLOPS}
\begin{definition}
    \begin{itemize}
        \item \textbf{Matrix Multiplication:} The operation of multiplying two matrices, resulting in a new matrix. 
        \item \textbf{FLOPS:} Floating Point Operations Per Second, a measure of computational performance.
    \end{itemize}
\end{definition} 
\newpage

\subsection{Techniques to Scale LLMs}
\begin{summary}
    \begin{center}
        \begin{tabular}{ll}
            \toprule
            \textbf{Technique} & \textbf{Description} \\
            \midrule
            Hardware Accelerators & GPUs, TPUs, Chips \\
            \midrule
            Data Parallelism: Sharding & Many ways to partition and distribute data across multiple devices. \\
            \multicolumn{2}{p{\linewidth}}{
                \begin{center}
                    \customFigure[0.75]{../../Images/L15_4.png}{}
                    \vspace{-2em}
                \end{center}} \\
            \midrule
            Data Mixtures & Combining diverse datasets to enhance generalization \\
            \multicolumn{2}{p{\linewidth}}{
                \begin{center}
                    \customFigure[0.5]{../../Images/L15_5.png}{}
                    \vspace{-2em}
                \end{center}} \\
            \bottomrule
        \end{tabular}
    \end{center}
\end{summary}
\newpage

\begin{summary}
    \begin{center}
        \begin{tabular}{ll}
            \toprule
            \textbf{Technique} & \textbf{Description} \\
            \midrule
            Mixture of Experts & Scaling LLMs with sparsity \\
            & Improving LLM capacity and efficiency via conditional computation \\
            \multicolumn{2}{p{\linewidth}}{
                \begin{itemize}
                    \customFigure[0.75]{../../Images/L15_6.png}{}
                    \item Ensemble of experts. 
                    \item Router model decides which expert to use. 
                \end{itemize}} \\
            \midrule
            Supervised/Instruction-Tuning & Optimizing for hard to optimize rewards (usability) \\
            \multicolumn{2}{p{\linewidth}}{
                \begin{itemize}
                    \item Collect demonstration data and train a supervised policy. 
                    \item Collect comparison data and train a reward model. 
                    \item Optimize a policy against the reward model using the PPO RL algorithm.
                \end{itemize}} \\
            \bottomrule
        \end{tabular}
    \end{center}
\end{summary}
\newpage

\begin{summary}
    \begin{center}
        \begin{tabular}{ll}
            \toprule
            \textbf{Technique} & \textbf{Description} \\
            \midrule
            Quantization & Reducing model size and cost \\ 
            & Process of mapping values from a larger set to a smaller set. \\
            \multicolumn{2}{p{\linewidth}}{
                \begin{center}
                    \customFigure[0.75]{../../Images/L15_7.png}{}
                    \vspace{-2em}
                \end{center}} \\
            \midrule
            Multi-modal tokens & Numerical representations for pieces of data. \\
            \multicolumn{2}{p{\linewidth}}{
                \begin{itemize}
                    \item \textbf{Text:} Text chunks to tokens 
                    \item \textbf{Image:} Image patches to tokens
                    \item \textbf{Audio:} Sound wave chunks to tokens
                    \item \textbf{Graphs:} Nodes are tokens 
                \end{itemize}} \\
            \midrule
            Vision Transformers (ViT) & Applying transformers for image recognition \\
            \multicolumn{2}{p{\linewidth}}{
                \begin{center}
                    \customFigure[0.9]{../../Images/L15_8.png}{}
                    \vspace{-2em}
                \end{center}} \\
            \bottomrule
        \end{tabular}
    \end{center}
\end{summary}
\newpage

\begin{summary}
    \begin{center}
        \begin{tabular}{ll}
            \toprule
            \textbf{Technique} & \textbf{Description} \\
            \midrule
            CLIP & Connecting images and text through contrastive learning \\
            & Aligning image and text tokens into a shared embedding space \\
            \multicolumn{2}{p{\linewidth}}{
                \begin{center}
                    \customFigure[0.9]{../../Images/L15_9.png}{}
                    \vspace{-2em}
                \end{center}} \\
            \midrule
            LoRA & Low-Rank Reparameterization Adaptation \\
            & Achieve parameter-efficient fine-tuning of LLMs via Low-Rank Updates \\
            \multicolumn{2}{p{\linewidth}}{
                \begin{itemize}
                    \customFigure[0.4]{../../Images/L15_10.png}{}                    
                        \item \textbf{Key Idea:} Instead of updating the full weight matrix $W \in \mathbb{R}^{d \times d}$, LoRA introduces a low-rank decomposition:
                        \begin{itemize}
                            \item $A \in \mathbb{R}^{d \times r}$ is a small trainable matrix initialized from a Gaussian distribution.
                            \item $B \in \mathbb{R}^{r \times d}$ is another trainable matrix (often initialized as zero).
                            \item Only $A$ and $B$ are trained during fine-tuning.
                        \end{itemize}
                        \item \textbf{Benefits:} When $r$ is small, this approach has a small \# of parameters and fast fine tuning. 
                        \item \textbf{Update:} $W' = W + A B$. 
                \end{itemize}} \\
            \bottomrule
        \end{tabular}
    \end{center}
\end{summary}
\newpage

\begin{summary}
    \begin{center}
        \begin{tabular}{ll}
            \toprule
            \textbf{Technique} & \textbf{Description} \\
            \midrule
            Reasoning \& Test-Time Compute & Instead of spending compute at training, \\ 
            & spend compute at test time (inference) \\
            \multicolumn{2}{p{\linewidth}}{
                \begin{itemize}
                    \customFigure[0.75]{../../Images/L15_11.png}{Process Reward Model (PRM)}                    
                    \item \textbf{Key Concept:} Model dynamically explores multiple reasoning paths at test time and selects the best one based on feedback from a reward model.
                
                    \item \textbf{Pipeline Overview:}
                    \begin{itemize}
                        \item A \textbf{math problem} is provided as input.
                        \item A \textbf{LLM} generates multiple candidate reasoning steps or intermediate solutions.
                        \item A \textbf{PRM} evaluates the quality of each step based on how useful or promising it is for reaching a final solution.
                        \item A \textbf{search strategy} uses PRM feedback to guide further generation, refining or reordering the steps.
                        \item The best sequence of reasoning steps is selected to produce the \textbf{final answer}.
                    \end{itemize}
                \end{itemize}} \\
            \midrule
            Reasoning & Chain/Tree of Thoughts \\
            & Planning answers instead of answering right away. \\
            \multicolumn{2}{p{\linewidth}}{
                \begin{center}
                    \customFigure[0.75]{../../Images/L15_12.png}{}
                    \vspace{-2em}
                \end{center}} \\
            \bottomrule
        \end{tabular}
    \end{center}
\end{summary}
\newpage

\subsection{High-Level Impacts}
\subsubsection{Costs}
\begin{notes}
    \begin{itemize}
        \item Scaling LLMs parameters $\alpha$ Cost 
        \item R\&D Staff, AI Accelerator Chips, other server components, cluster-level interconnect, and energy costs for training and experiments of ML models. 
    \end{itemize}
\end{notes}

\subsubsection{Neural Scaling Laws: Predicting Performance in DL}
\begin{notes}
    Understanding the relationship b/w model size, data, and compute. 
    \begin{itemize}
        \item $\text{Performance} \propto N^{\alpha}, D^{\beta}, C^{\gamma}$
        \begin{itemize}
            \item $D$: Dataset size (tokens), $N$: \# of parameters, $C$: Compute time (FLOPS)
        \end{itemize}
        \customFigure[0.5]{../../Images/L15_13.png}{}
        \vspace{-1em}
        \customFigure[0.5]{../../Images/L15_14.png}{}
        \vspace{-1em}
        \customFigure[0.5]{../../Images/L15_15.png}{}
    \end{itemize}
\end{notes}

\subsubsection{Bitter Lesson}
\begin{warning}
    \begin{itemize}
        \item \textbf{1)} AI researchers have historically attempted to embed explicit knowledge into their agents.
        
        \item \textbf{2)} This approach often yields short-term gains and can be personally rewarding to the researcher.
    
        \item \textbf{3)} However, such knowledge-based systems eventually plateau and may even hinder long-term progress.
    
        \item \textbf{4)} Major breakthroughs typically result from a contrasting strategy—scaling up computation and leveraging learning and search algorithms instead of hand-crafted knowledge.
    \end{itemize}    
\end{warning}