\subsection{Software Engineering}
\begin{summary}
    \begin{center}
        \begin{tabular}{ll}
            \toprule
            \textbf{Concept} & \textbf{Description} \\
            \midrule
            Code Readability & Maintainable code enables collaboration and future work. \\
            \multicolumn{2}{p{\linewidth}}{
                \centering
                \begin{minipage}{0.48\linewidth}
                    \centering
                    \customFigure[1.0]{../Images/L7_4.png}{Image 1 Description}
                \end{minipage}
                \hfill
                \begin{minipage}{0.48\linewidth}
                    \centering
                    \customFigure[1.0]{../Images/L7_1.png}{Image 2 Description}
                \end{minipage} 
            } \\
            \bottomrule
        \end{tabular}
    \end{center}
\end{summary}
\subsection{Code Readability Matters}
\begin{notes}
    \begin{itemize}
        \item \textbf{Overview:} Maintainable code enables collaboration and future work.
        \customFigure[0.5]{../Images/L7_1.png}{}
        \customFigure[0.5]{../Images/L7_4.png}{}
        \item \textbf{Naming:} Clear names enhance code understanding 
        \begin{itemize}
            \item \textbf{Functions:} verb\_do
            \item \textbf{Variables:} two\_three\_words
            \item \textbf{Classes:} CapitalizedWords
        \end{itemize}
        \customFigure[0.5]{../Images/L7_2.png}{}
    \end{itemize}
\end{notes}

\subsection{Style Guides}
\begin{notes}
    \begin{itemize}
        \item \textbf{Overview:} Ensure consistent code formatting. 
        \item \textbf{Why?} Provide a standardized set of rules for formatting code, prompting uniformity, and reducing cognitive load when reading code. 
        \item \textbf{Examples:} PEP8, Google Style Guide, etc.
    \end{itemize}
    \customFigure[0.5]{../Images/L7_3.png}{}
    \customFigure[0.5]{../Images/L7_5.png}{}
\end{notes}
\newpage

\section{Coding Mantras / Ideas}
\subsection{Zen of Python}
\begin{definition}
    \customFigure[0.5]{../Images/L7_0.png}{}
\end{definition}

\subsection{KISS: Keep It Simple and Straightforward}
\begin{notes}
    Advocate for solutions that are easy to understand and maintain, reduce unnecessary complexity.
    \customFigure[0.5]{../Images/L7_6.png}{}
\end{notes}

\subsection{YAGNI: You Aren't Gonna Need It}
\begin{notes}
    AKA premature optimization is the root of all evil. 
    \customFigure[0.5]{../Images/L7_7.png}{}
    \customFigure[0.5]{../Images/L7_8.png}{}
\end{notes}

\subsection{Principle of DRY and WET and DAMP}
\begin{notes}
    Don't Repeat Yourself; Write Everything Twice.
    \customFigure[0.5]{../Images/L7_9.png}{}
    \vspace{1em}

    Don't Abstract Methods Prematurely.
    \customFigure[0.5]{../Images/L7_10.png}{}
\end{notes}
\newpage

\section{Into The Weeds (Technical Details)}
\subsection{Python Data Containers Overview}
\begin{definition}
    Choosing the right container for the task:
    \begin{itemize}
        \item \textbf{list:} Ordered, mutable sequence; versatile for collections of items.
        \item \textbf{tuple:} Ordered, immutable sequence; suitable for fixed collections and data integrity.
        \item \textbf{set:} Unordered collection of unique elements; efficient for membership testing and removing duplicates.
        \item \textbf{dict:} Key-value mappings; ideal for efficient data lookup and representing structured information.
    \end{itemize}
    \vspace{1em}
    
    \textbf{Data Structures:}
    \begin{itemize}
        \item \textbf{collections.NamedTuple}
        \item \textbf{dataclasses.Dataclass}
    \end{itemize}
    \vspace{1em}

    \textbf{Tensors}
    \begin{itemize}
        \item \textbf{numpy.ndarray}
        \item \textbf{torch.Tensor}
        \item \textbf{jax.Array}
    \end{itemize}
\end{definition}

\subsection{List Comprehensions}
\begin{notes}
    Provide concise list creation.
    \customFigure[0.5]{../Images/L7_11.png}{}
\end{notes}

\subsection{Leveraging Set Data Structures: Set}
\begin{notes}
    Efficiently handle unique elements and set logic.
    \customFigure[0.5]{../Images/L7_12.png}{}
\end{notes}

\subsection{Dictionaries for Key-Value Pairs}
\begin{notes}
    Enable efficient data lookup by keys.
    \customFigure[0.5]{../Images/L7_13.png}{}
\end{notes}

\subsection{Itertools for Efficient Iteration}
\begin{notes}
    Provides tools for very common iteration patterns.
    \begin{itemize}
        \item e.g. permutations, chunked, chain, zip, filter, product, combinations, etc.
    \end{itemize}
    \customFigure[0.5]{../Images/L7_14.png}{}
    \customFigure[0.5]{../Images/L7_15.png}{}
\end{notes}

\subsection{Functools for Functional Tools}
\begin{notes}
    Need to manipulate a function and get a new function. 
    \begin{itemize}
        \item Caching with \texttt{functools.lru\_cache}: Cache optimizes performance by storing results (i.e. return a previous result when the input of a function has already been observed)
    \end{itemize}
    \customFigure[0.5]{../Images/L7_16.png}{}
    \customFigure[0.5]{../Images/L7_17.png}{}
\end{notes}

\subsection{Abstract Base Classes (ABCs)}
\begin{notes}
    ABCs enforce interfaces for robust design.
    \customFigure[0.5]{../Images/L7_18.png}{}
\end{notes}

\subsection{Dataclasses for Data Storage}
\begin{notes}
    Data classes simplify data-centric class creation.
    \customFigure[0.5]{../Images/L7_19.png}{}
\end{notes}

\subsection{Python Typing Provides Hints}
\begin{notes}
    Type hints enhance code clarity, communicate intent and detect errors.
    \customFigure[0.5]{../Images/L7_20.png}{} 
\end{notes}

\subsection{Tensor Typing: jaxtyping}
\begin{notes}
    Communicate expected tensor shapes and data types. 
    \customFigure[0.5]{../Images/L7_21.png}{}
\end{notes}

\subsection{pytest: Helps you write better programs}
\begin{notes}
    Simplifies testing for reliable code.
    \customFigure[0.5]{../Images/L7_22.png}{}
\end{notes}

\subsection{Ideas}
\begin{summary}
    \begin{center}
        \begin{tabular}{l}
            \toprule
            \textbf{Exercise} \\
            \toprule
            Write research ideas. Get a mentor to rate them \\
            \midrule
            Ask other researchers about their taste \\
            \midrule
            Read about history of research ("The Structure of Scientific Revolutions") \\
            \midrule
            De-risk your ideas: Proactive idea evaluation mitigates research risks (kill fast, learn fast) \\
            \multicolumn{1}{p{\linewidth}}{
            \begin{enumerate}
                \item Identify potential bottlenecks
                \item Prioritize and commit $X$ amt. of time to exploring them
                \item Decide if you should continue or pivot
            \end{enumerate}} \\
            \bottomrule
        \end{tabular}
    \end{center}
    \begin{itemize}
        \item \href{https://colah.github.io/notes/taste/}{Research Taste}
    \end{itemize}
\end{summary}

\begin{warning}
    \begin{itemize}
        \item Getting attached to one direction. 
        \item Lack of research knowledge / intimacy.
        \item Environment is not supportive of your interests.
    \end{itemize}
\end{warning}
\newpage

\subsection{Code / Experiments}
\begin{summary}
    \begin{center}
        \begin{tabular}{ll}
            \toprule
            \textbf{Tools} & \textbf{Links} \\
            \toprule
            Artifacts to create/track w/ experiments & \\
            \multicolumn{2}{p{\linewidth}}{
            \begin{itemize}
                \item Data, code (scripts / modules), models (weights, configurations), results, predictions, plots, meeting notes, papers, documentation, etc.
            \end{itemize}} \\
            \midrule
            Git (Version Control): Enables effective tracking and collaborative changes & \href{https://rogerdudler.github.io/git-guide/}{Git Guide} \\
            \multicolumn{2}{p{\linewidth}}{
            \begin{itemize}
                \item Tracking changes, collaboration, backup, revert.
                \item Add, commit -m "message", push, pull, merge, diff, revert, branch, checkout, log, status, etc.
                \item .gitignore: Ignore files, directories, or patterns
            \end{itemize}} \\
            \midrule
            GitHub (Collaborative Code Hosting): Facilitates sharing and collaboration on code & \href{https://github.com/git-guides}{GitHub} \\
            \multicolumn{2}{p{\linewidth}}{
            \begin{itemize}
                \item Collaboration, sharing, open science, project hosting.
            \end{itemize}} \\
            \midrule
            Cookiecutter (Project Template): Standardizes project structure & \href{https://cookiecutter.readthedocs.io/en/1.7.2/}{Cookiecutter} \\
            & \href{https://github.com/drivendataorg/cookiecutter-data-science}{Repo} \\
            \multicolumn{2}{p{\linewidth}}{
            \begin{itemize}
                \item Logical, flexible, and reasonably standardized project structure for doing and sharing data science work.
                \item File Structure
                \begin{itemize}
                    \item data/: 
                    \begin{itemize}
                        \item external/: Data from third party sources.
                        \item interim/: Intermediate data that has been transformed.
                        \item processed/: The final, canonical data sets for modeling.
                        \item raw/: The original, immutable data dump.
                    \end{itemize}
                    \item src/: Source code for use in this project.
                    \begin{itemize}
                        \item \_\_init\_\_.py: Makes src a Python module.
                        \item config.py: Configuration settings.
                        \item dataset.py: Code to load data.
                        \item features.py: Code to build features.
                        \item modeling/: Code to train models.
                        \begin{itemize}
                            \item \_\_init\_\_.py: Makes modeling a Python module.
                            \item predict.py: Code to make predictions.
                            \item train.py: Code to train models.
                        \end{itemize}
                        \item plots.py: Code to create plots.
                    \end{itemize}
                    \item docs/: Documentation for this project.
                    \item models/: Trained and serialized models, model predictions, or model summaries.
                    \item notebooks/: Jupyter notebooks.  
                    \item references/: Data dictionaries, manuals, and all other explanatory materials.
                    \item reports/: Generated analysis as HTML, PDF, LaTeX, etc.
                    \begin{itemize}
                        \item figures/: Generated graphics and figures to be used in reporting.
                    \end{itemize}
                    \item pyproject.toml: Project information and dependencies.
                    \item requirements.txt: The requirements file for reproducing the analysis environment.
                    \item setup.cfg: Configuration file for setting up the project.
                    \item LICENSE: 
                    \item Makefile: 
                    \item README.md: 
                \end{itemize}
            \end{itemize}} \\
            \bottomrule
        \end{tabular}
    \end{center}
\end{summary}
\newpage

\begin{summary}
    \begin{center}
        \begin{tabular}{ll}
            \toprule
            \textbf{Tools} & \textbf{Links} \\
            \toprule
            Cookiecutter (Project Template): Standardizes project structure & \href{https://cookiecutter-data-science.drivendata.org/opinions/}{Opinions} \\
            \multicolumn{2}{p{\linewidth}}{
            \begin{itemize}
                \item \textbf{Design Philosophy:} Prioritizes conventions and reasonable defaults to streamline project setup. Opinions:
                \begin{itemize}
                    \item Data analysis is a DAG:
                    \begin{enumerate}
                        \item Raw data
                        \item Compute features
                        \item Plot analysis on raw data 
                        \item Train model
                        \item Compute statistics on features
                    \end{enumerate}
                    \item Raw data is immutable (i.e. never change raw data)
                    \begin{itemize}
                        \item \textbf{Dos:}
                        \begin{itemize}
                            \item Pipeline code: Process raw data $\rightarrow$ final analysis.
                            \item Cache outputs: Serialize or cache intermediate steps. 
                            \item Reproducible results: Enable full reproduction from code and raw data only.
                        \end{itemize}
                        \item \textbf{Don'ts:}
                        \begin{itemize}
                            \item Never edit raw data: Avoid manual edits or format changes
                            \item Never overwrite raw data: Do not replace raw data with processed data.
                            \item Single raw data version: Maintain only one version of raw data.
                        \end{itemize}
                    \end{itemize}
                    \item Data should (mostly) not be kept in source control
                    \begin{itemize}
                        \item GitHub warns for files over 50MB and rejects files over 100MB.
                        \item Use s3, azcopy, gcloud, drive to store data (i.e. cloud services)
                        \item Use cloudpathlib to access cloud data in the same way as pathlib to access local data.
                    \end{itemize}
                    \item Notebooks are for exploration and communication, source files are for repetitions.
                    \item Refactor the good parts into source code (\href{https://cookiecutter-data-science.drivendata.org/opinions/\#notebooks-are-for-exploration-and-communication-source-files-are-for-repetition}{Refactor Example}).
                    \begin{itemize}
                        \item Don't write code to do the same task in multiple notebooks.
                    \end{itemize}
                    \item Keep your modelling organized \href{https://pytorch.org/tutorials/beginner/basics/saveloadrun_tutorial.html}{(PyTorch Example)}
                    \begin{itemize}
                        \item Predictions (csv), training log (csv), stats (txt), model config / hyperparameters (json)
                    \end{itemize}
                    \item Build from the environment up \href{https://github.com/mamba-org/mamba}{(Mamba)}
                    \begin{itemize}
                        \item Use mamba rather than conda for faster environment management.
                        \item Create a environment.yml file to manage dependencies.
                    \end{itemize}
                \end{itemize}
            \end{itemize}} \\
            \bottomrule
        \end{tabular}
    \end{center}
\end{summary}
\newpage

\subsection{Writing / Analyzing}
\begin{summary}
    \begin{center}
        \begin{tabular}{ll}
            \toprule
            \textbf{Exercise} & \textbf{Links} \\
            \toprule
            Writing Skeleton & \href{https://esajournals.onlinelibrary.wiley.com/doi/full/10.1002/bes2.1258}{Step-by-Step Guide to Undergraduate Writing} \\
            & \href{https://www.nature.com/articles/d41586-018-02404-4}{How to write a first-class paper (Nature)} \\
            & \href{https://www.ncbi.nlm.nih.gov/pmc/articles/PMC5037950/}{Preparing Manuscript: Scientific Writing for Publication} \\
            \multicolumn{2}{p{\linewidth}}{
            \begin{enumerate}
                \item Start w/ Figures (How would you want to telll the story?)
                \item Write the structure (Introduction, Methods, Experiments and Results, Discussion)
                \item 2-3 sentence pitch for your idea
                \item Bullet points inside of each section (What are you expecting to cover?)
                \item Fill in text, repeat.
            \end{enumerate}} \\
            \midrule
            Figures: Move quick, perfect later & \\
            \multicolumn{2}{p{\linewidth}}{
            \begin{itemize}
                \item Figure \#1: Tells problem in simple way (30s elevator pitch)
                \item Figure \#2-3: Conceptual or data centric
                \begin{itemize}
                    \item How are you solving the problem?
                    \item What does the data look like?
                \end{itemize}
                \item Figure \#4-8: Quantitative evidence
                \item \textbf{Recommendations:}
                \begin{itemize}
                    \item Napkin/whiteboard figures first
                    \item Make good enough version w/ code (svg, png) using matplotlib, seaborn, etc.
                    \item Finetune w/ InkScape, Illustrator, GIMP, etc
                \end{itemize}
                \item \textbf{Anatomy of a Figure Examples (L8):} 
                \begin{itemize}
                    \item Slide 44: Task, Slide 45: Model + EDA
                    \item Slide 46-47: Quantitative evidence, Slide 48: Different ways of telling same story (e.g. tables or plots)
                \end{itemize}
            \end{itemize}} \\
            \midrule
            Pick good, consistent colors & \href{https://colorbrewer2.org/#type=sequential&scheme=BuGn&n=3}{ColorBrewer}, \href{https://matplotlib.org/stable/gallery/color/colormap_reference.html}{Matplotlib Colormaps} \\
            & \href{http://seaborn.pydata.org/tutorial/aesthetics.html}{Seaborn Palettes}, \href{https://www.youtube.com/watch?v=ze08gwVPaXk}{NeurIPS 18 Visualization for ML tutorial} \\
            \multicolumn{2}{p{\linewidth}}{
            \begin{itemize}
                \item Be mindful of how colours can help tell a story, accessibility is also important.
            \end{itemize}}\\
            \midrule
            Pair-writing (Como Pair-Coding) & \href{https://en.wikipedia.org/wiki/Rubber_duck_debugging}{Rubber Duck Debugging}, \href{https://en.wikipedia.org/wiki/Pair_programming}{Pair Programming} \\
            & \href{http://sunnyday.mit.edu/16.355/williams.pdf}{Pair Writing}, \href{https://pds.blog.parliament.uk/2017/03/29/pair-writing/}{Pair Writing in Government} \\
            \multicolumn{2}{p{\linewidth}}{
            \begin{itemize}
                \item Working together $\rightarrow$ Help communicate thoughts adn put you in a diff. attitude.
            \end{itemize}}\\
            \midrule
            Communal writing (Social pressure $\rightarrow$ accountable) & \href{https://gsas.harvard.edu/academics/writing}{Harvard Writing Center} \\
            \multicolumn{2}{p{\linewidth}}{
            \begin{enumerate}
                \item Setup an objective, measurable goal. 
                \item Set a time for writing period, take breaks
                \item Share progress at the end of each session, share writing stuggles if needed.
                \item Reflect if there are some reasons why it is hard to write.
            \end{enumerate}}\\
            \midrule
            Writing: Focus on quick iterations & \\
            \multicolumn{2}{p{\linewidth}}{
            \begin{itemize}
                \item Google docs and Paperpile (Copy DOI, paste, click, done) $\overset{\text{Export w/ bibtex}}{\Longrightarrow}$ LaTex (Overleaf)
            \end{itemize}}\\
            \midrule
            Interactive Apps & \href{https://streamlit.io/}{Streamlit}, \href{https://www.gradio.app/}{Gradio} \\
            & \href{https://www.huggingface.co}{Hugging Face}, \href{https://www.hf.co/spaces}{Hugging Face Spaces} \\
            & \href{https://github.com/eliahuhorwitz/Academic-project-page-template}{Academic Project Page Template} \\
            \bottomrule
        \end{tabular}
    \end{center}
\end{summary}


