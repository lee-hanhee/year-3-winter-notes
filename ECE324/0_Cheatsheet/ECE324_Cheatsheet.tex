\documentclass{article}
\usepackage{style}
\title{ECE324 Lectures}
\author{Hanhee Lee}
\lhead{ECE324}
\rhead{Hanhee Lee}

\begin{document}
\section{Problem 1: Description of the Problem, Subscribe a ML/NN Based Solution}
\begin{process}

\end{process}

\begin{definition}
    
\end{definition}

\begin{example}
    
\end{example}

\section{Problem 2: Prescribe a strategy to optimize the NN.}
\begin{process}

\end{process}

\begin{definition}
    
\end{definition}

\begin{example}
    
\end{example}

\section{Problem 3: Explain step by step inference for basic algorithms (MLP, CNN, GNN, attention mechanism) in terms of numpy or basic tensor operations.}
\begin{process}

\end{process}

\begin{definition}
    
\end{definition}

\begin{example}
    
\end{example}

\section{Problem 4: Be able to explain why such a solution might work or fail.} 
\begin{process}

\end{process}

\begin{definition}
    
\end{definition}

\begin{example}
    
\end{example}
\newpage

\section{ML Concepts}
\subsection{Bias-Variance Tradeoff}
\begin{definition}
    \begin{itemize}
        \item \textbf{Bias (Underfitting):} Error due to overly simplistic assumptions in the model.
        \begin{itemize}
            \item A model with high bias makes strong assumptions about the data, leading to oversimplification.
            \item This results in poor training and test performance because the model fails to capture important patterns.
        \end{itemize}
        \item \textbf{Variance (Overfitting):} Error due to the model's sensitivity to fluctuations in the training data.
        \begin{itemize}
            \item A model with high variance is too complex and captures noise in the training data, rather than just the underlying pattern.
            \item This results in excellent training performance but poor generalization to new data.
        \end{itemize}
        \item \textbf{Low Bias, High Variance:} Overly complex models (e.g., deep neural networks with excessive parameters) suffer from overfitting, performing well on training data but poorly on test data.
        \item \textbf{High Bias, Low Variance:} Overly simplistic models (e.g., linear regression) suffer from underfitting, performing poorly on both training and test data.
    \end{itemize}
\end{definition}

\subsubsection{Solution to High Variance}
\begin{definition}
    \begin{itemize}
        \item Regularization
        \item Data Augmentation
        \item More trianing data 
        \item Simpler models
        \item Ensemble methods
    \end{itemize}
\end{definition}

\subsubsection{Solution to High Bias}
\begin{definition}
    \begin{itemize}
        \item More complex models
        \item Feature engineering
        \item Hyperparameter tuning
        \item Reduce regularization
    \end{itemize}
\end{definition}
\newpage

\section{Loss Functions}
\begin{summary}
    \begin{center}
        \begin{tabular}{ll}
            \toprule
            \textbf{Loss Fn} & \textbf{Equation} \\ 
            \toprule
            \textbf{Mean Squared Error (MSE)} & $\frac{1}{n} \sum_{i=1}^{n} (y_i - \hat{y}_i)^2$ \\
            \multicolumn{2}{p{\linewidth}}{
            \begin{itemize}
                \item \textbf{When to Use?} Penalizes large errors.
            \end{itemize}} \\
            \midrule
            \textbf{Root Mean Squared Error (RMSE)} & $\sqrt{\frac{1}{n} \sum_{i=1}^{n} (y_i - \hat{y}_i)^2}$ \\
            \multicolumn{2}{p{\linewidth}}{
            \begin{itemize}
                \item \textbf{When to Use?} Error needs to be in the same units as the target.
            \end{itemize}} \\
            \midrule
            \textbf{Mean Absolute Error (MAE)} & $\frac{1}{n} \sum_{i=1}^{n} |y_i - \hat{y}_i|$ \\
            \multicolumn{2}{p{\linewidth}}{
            \begin{itemize}
                \item \textbf{When to Use?} Equal weighting of all errors (robust to outliers).
            \end{itemize}} \\
            \midrule
            \textbf{Binary Cross Entropy} & $- \frac{1}{n} \sum_{i=1}^{n} y_i \log(\hat{y}_i) + (1 - y_i) \log(1 - \hat{y}_i)$ \\
            \multicolumn{2}{p{\linewidth}}{
            \begin{itemize}
                \item \textbf{When to Use?} Binary classification problems.
            \end{itemize}} \\
            \midrule
            \textbf{Categorical Cross Entropy} & $- \frac{1}{n} \sum_{i=1}^{n} \sum_{j=1}^{m} y_{ij} \log(\hat{y}_{ij})$ \\
            \multicolumn{2}{p{\linewidth}}{
            \begin{itemize}
                \item \textbf{When to Use?} Multi-class classification problems.
            \end{itemize}} \\
            \midrule
            \textbf{AE: Reconstruction Error} & Minimize reconstruction error. \\
            & $L_{AE} = \left\| \mathbf{x} - f_{\text{dec}}(f_{\text{enc}}(\mathbf{x})) \right\|^2 = \left\| \mathbf{x} - \hat{\mathbf{x}} \right\|^2$ \\
            \multicolumn{2}{p{\linewidth}}{
            \begin{itemize}
                \item If our data is binary, what loss function should you use? Binary cross entropy.
            \end{itemize}} \\
            \midrule
            \textbf{VAE: Reconstruction and KL Divergence} & Balances reconstruction and latent space regularization. \\
            &  $L_{\text{VAE}} = L_{\text{AE}} (x, \hat{x}) + \text{KL} (q_\phi (\mathbf{z} \mid \mathbf{x}) \parallel p(\mathbf{z}))$ \\
            \multicolumn{2}{p{\linewidth}}{
            \begin{itemize}
                \item For Gaussian Distribution: $KL(P \| Q) = KL\left(\mathcal{N}(\mu_1, \sigma_1^2) \| \mathcal{N}(\mu_2, \sigma_2^2)\right) = \frac{1}{2} \left[ \log\left(\frac{\sigma_2^2}{\sigma_1^2}\right) + \frac{\sigma_1^2 + (\mu_1 - \mu_2)^2}{\sigma_2^2} - 1 \right]$ 
            \end{itemize}} \\
            \bottomrule
        \end{tabular}
    \end{center}
\end{summary}
\newpage

\section{Algorithms}
\begin{summary}
    \begin{center}
        \begin{tabular}{llll}
            \toprule
            \textbf{Algorithm} & \textbf{Inputs} & \textbf{Outputs} & \textbf{Equations} \\
            \toprule
            PCA & $x$ & $x'$ & $x \underset{\text{Encode}}{\mapsto} z \underset{\text{Decode}}{\mapsto} x'$ \\
            & & & $\text{Encoder}(x) = Wx$, $\text{Decoder}(z) = W^{-1}z$ \\
            \multicolumn{4}{p{\linewidth}}{
            \begin{itemize}
                \item \textbf{Overview:} Linear transformation for dimension reduction 
            \end{itemize}} \\
            \midrule
            AE & $x$ & $x'$ & $x \underset{\text{Encode}}{\mapsto} z \underset{\text{Decode}}{\mapsto} x'$ \\
            & & & $f_{\text{enc}}(x), \; \text{Encoder}(x) = \text{Neural network}$, $f_{\text{dec}}(z), \; \text{Decoder}(z) = \text{Neural network}$ \\
            \multicolumn{4}{p{\linewidth}}{
            \begin{itemize}
                \item Bottleneck (latent space) forces network to learn compressed representation
                \item Encoder maps input to lower dimension, decoder reconstructs input
            \end{itemize}} \\
            \midrule
            VAE & $x$ & $x'$ & $x \underset{\text{Encode}}{\mapsto} z \underset{\text{Decode}}{\mapsto} x'$ \\
            & & & $f_{\text{enc}, \phi}(\mathbf{z} \mid \mathbf{x}) = \mathcal{N}(\mathbf{z} \mid \boldsymbol{\mu}(\mathbf{x}), \sigma^2(\mathbf{x}) \mathbf{I})$, $f_{\text{dec}}(\mathbf{z})$\\ 
            \multicolumn{4}{p{\linewidth}}{
            \begin{itemize}
                \item \textbf{Overview:} A probabilistic approach by learning smooth latent spaces by modelling probability distributions.
                \item $\boldsymbol{\mu}(\mathbf{x})$: Mean of the distribution, $\sigma^2(\mathbf{x})$: Variance of the distribution (independent)
                \item Encoder maps input to a distribution, decoder samples from it.
                \begin{itemize}
                    \item Encoder estimates parameters for a distribution in the latent space
                    \item Decoder samples from the distribution to generate an output sample
                \end{itemize}
                \item \textbf{Key:} Works on any data.
            \end{itemize}} \\
            \midrule
            RNN & $x_t,h_{t-1}$ & $y_{t},h_{t}$ & $h_t = \text{tanh}(\text{Linear} (h_{t-1}) + \text{Linear}(x_t))$ \\ 
            & & & $y_t = \text{MLP}(h_t)$ \\
            \multicolumn{4}{p{\linewidth}}{
            \begin{itemize}
                \item $x_t$: Input, $h_t$: Hidden state, $y_t$: Output
            \end{itemize}} \\
            \midrule
            GRU & $x_t,h_{t-1}$ & $y_t,h_t$ & $z_t = \text{sigmoid}(\text{Linear}(x_t) + \text{Linear}(h_{t-1}))$ \\
            & & & $r_t = \text{sigmoid}(\text{Linear}(x_t) + \text{Linear}(h_{t-1}))$ \\
            & & & $\tilde{h}(t) = \text{tanh}(\text{Linear}(x_t) + \text{Linear}(r_t \odot h_{t-1}))$ \\
            & & & $h_t = (1-z_t) \odot h_{t-1} + z_t \odot \tilde{h}_t$ \\
            \multicolumn{4}{p{\linewidth}}{
            \begin{itemize}
                \item $z_t$: Update gate, $r_t$: Reset gate
                \item $h_t$: Hidden state
            \end{itemize}} \\
            \midrule
            LSTM & $x_t,h_{t-1}$ & $h_t,c_t$ & $f_t = \text{sigmoid}(\text{Linear}(x_t) + \text{Linear}(h_{t-1}))$ \\
            & & & $i_t = \text{sigmoid}(\text{Linear}(x_t) + \text{Linear}(h_{t-1}))$ \\
            & & & $o_t = \text{sigmoid}(\text{Linear}(x_t) + \text{Linear}(h_{t-1}))$ \\
            & & & $\tilde{c}_t = \text{tanh}(\text{Linear}(x_t) + \text{Linear}(h_{t-1}))$ \\
            & & & $c_t = f_t \odot c_{t-1} + i_t \odot \tilde{c}_t$ \\
            & & & $h_t = o_t \odot \text{tanh}(c_t)$ \\
            \multicolumn{4}{p{\linewidth}}{
            \begin{itemize}
                \item $f_t$: Forget gate, $i_t$: Input/Update gate, $o_t$: Output gate
                \item $h_t$: Hidden state, $c_t$: Cell state
            \end{itemize}} \\
            \bottomrule
        \end{tabular}
    \end{center}
\end{summary}

\subsection{Geometric DL Blueprint for NNs}
\begin{summary}
    Unify various networks around symmetry. 
    ADD IMAGE
\end{summary}
\newpage

\section{Intro to ML}
% \begin{summary}
    \begin{itemize}
        \item Not responsible for proofs, but know when to use each algorithm.
    \end{itemize}
\end{summary}
\subsection{Setup}
\begin{definition} In a search problem, it is assumed that: 
    \begin{itemize}
        \item There is only one agent (us).
        \item For each state, $s \in S$, we have a discrete set of actions, $\mathcal{A}(s)$.
        \item The transition resulting from a move, $(s, a)$, is deterministic; the resulting state is $tr(s, a)$.
        \item $cst(s, a, tr(s, a))$ is our cost for the transition, $(s, a, tr(s, a))$.
        \item We want to realize a path that minimizes our cost.
    \end{itemize}
    
    A search problem may have no solutions, in which case, we define the solution as \texttt{NULL}.
\end{definition}

\subsection{Search Graphs}
\begin{definition}
    In a search graph (a graph representing a search problem):
    \begin{itemize}
        \item $S$ is defined by the vertices.
        \item $\mathcal{G}$ is a subset of the vertices.
        \item $s^{(0)}$ is some vertex.
        \item $tr(\cdot, \cdot)$ and $\mathcal{T}$ are defined by the edges.
        \item $cst(\cdot, \cdot, \cdot)$ is defined by the edge weights.
    \end{itemize}
\end{definition}

\subsection{Path Trees}
\begin{definition}
    A search algorithm explores a tree of possible paths. 
    \begin{itemize}
        \item In such a tree, each node represents the path from the root to itself.
        \begin{itemize}
            \item The node may also include other info (such as the path's origiin, cost, etc).
        \end{itemize}
    \end{itemize}
\end{definition}

\subsection{Search Algorithms}
\begin{definition}
    All search algorithms follow the template below:

\begin{lstlisting}
$\mathcal{O} \gets \{(\langle \rangle, 0)\}$ (*\hfill $\triangleright$ initialize a set of open nodes*) 
SEARCH($\mathcal{O}$)
\end{lstlisting}
\begin{itemize}
    \item $\langle \rangle$ is the empty path, and $0$ is the cost of the empty path.
\end{itemize}

\begin{lstlisting}
procedure SEARCH($\mathcal{O}$)
    if $\mathcal{O} = \emptyset$ then
        return NULL  (*\hfill $\triangleright$ the search algorithm failed to find a path to a goal*)
    $n \gets \textsc{Remove}(\mathcal{O})$ (*\hfill $\triangleright$ "explore" a node $n$*)
    if $\textsc{dst}(n) \in \mathcal{G}$ then
        return $n$ (*\hfill $\triangleright$ the search algorithm found a path to a goal*)
    for $n' \in \textsc{chl}(n)$ do
        $\mathcal{O} \gets \mathcal{O} \cup \{n'\}$ (*\hfill $\triangleright$ "expand" $n$ and "export" its children*)
    SEARCH($\mathcal{O}$)
\end{lstlisting}
\begin{itemize}
    \item Explore: Remove a node from the open set.
    \item Exapnd: Generate the children of the node.
    \item Export: Add the children to the open set.
\end{itemize}

\end{definition}

\begin{warning}
    The key difference is in the order that \textsc{Remove}($\cdot$) removes nodes.
\end{warning}

\subsubsection{Characteristics of a Search Algorithm}
\begin{definition}
    We want to choose \texttt{REMOVE(·)} so that the algorithm exhibits the following characteristics:

    \begin{center}
        \begin{tabular}{|p{3cm}|p{9cm}|}
        \hline
        \textbf{Characteristic} & \textbf{Description} \\ \hline
        Halting & Terminates after finitely many nodes explored \\ \hline
        Sound & Returned (possibly NULL) solution is correct \\ \hline
        Complete & Halting and sound when a non-NULL solution exists \\ \hline
        Optimal & Returns an optimal solution when multiple exist \\ \hline
        Time Efficient & Minimizes the nodes \textbf{explored}/expanded/exported \\ \hline
        Space Efficient & Minimizes the nodes simultaneously open \\ \hline
        \end{tabular}
    \end{center}
    \vspace{1em}   
    \begin{itemize}
        \item Will be using explored for time efficiency.
    \end{itemize} 
    \vspace{1em}

    The characteristics of the algorithm also depend on several properties of the path tree over which it searches. These properties include:
    \begin{itemize}
        \item Branching factor: $b$ ($b < \infty$), the maximum number of children a node can have.
        \item Depth: $d$, the length of the longest path.
        \item Length of the shortest solution: $l^*$
        \item Cost of the cheapest solution: $c^*$
        \item Cost of the cheapest edge: $\epsilon$ 
    \end{itemize}

    We want to choose \texttt{REMOVE($\cdot$)} so that the algorithm exhibits the aforementioned characteristics for as many path trees as possible.

\end{definition}

\subsubsection{Breadth First Search (BFS)}
\begin{definition}
    Explores the least-recently expanded open node first.
    \begin{center}
        \begin{tabular}{|p{3cm}|p{3cm}|}
        \hline
        \textbf{Property} & \textbf{Description} \\ \hline
        Halting & $d < \infty$ \newline non-NULL \\ \hline
        Sound & always \\ \hline
        Complete & always \\ \hline
        Optimal & constant cst \\ \hline
        Time & $b^{l^*}$ \\ \hline
        Space & $b^{l^* + 1}$ \\ \hline
        \end{tabular}
    \end{center}
\end{definition}

\subsubsection{Depth First Search (DFS)}
\begin{definition}
    Explores the most-recently expanded open node first.
    \begin{center}
        \begin{tabular}{|p{3cm}|p{3cm}|}
        \hline
        \textbf{Property} & \textbf{Description} \\ \hline
        Halting & $d < \infty$ \\ \hline
        Sound & always \\ \hline
        Complete & $d < \infty$ \\ \hline
        Optimal & never \\ \hline
        Time & $b^d$ \\ \hline
        Space & $bd$ \\ \hline
        \end{tabular}
    \end{center}    
\end{definition}

\subsubsection{Iterative Deepening DFS (IDDFS)}
\begin{definition}
    Same as DFS but with iterative deepening.
    \begin{center}
        \begin{tabular}{|p{3cm}|p{3cm}|}
        \hline
        \textbf{Property} & \textbf{Description} \\ \hline
        Halting & always \\ \hline
        Sound & always \\ \hline
        Complete & always \\ \hline
        Optimal & constant cst \\ \hline
        Time & $b^{l^*}$ \\ \hline
        Space & $bl^*$ \\ \hline
        \end{tabular}
    \end{center}    
\end{definition}

\subsubsection{Cheapest-First Search (CFS)}
\begin{definition}
    Explores the cheapest open node first.
    \begin{center}
        \begin{tabular}{|p{3cm}|p{3cm}|}
        \hline
        \textbf{Property} & \textbf{Description} \\ \hline
        Halting & $d < \infty$ \newline non-NULL \\ \hline
        Sound & yes \\ \hline
        Complete & $\epsilon > 0$ \\ \hline
        Optimal & $\epsilon > 0$ \\ \hline
        Time & $b^{c^*/\epsilon}$ \\ \hline
        Space & $b^{c^*/\epsilon + 1}$ \\ \hline
        \end{tabular}
    \end{center}    
\end{definition}

\subsection{Modifications to Search Algorithms}
\subsubsection{Depth-Limiting}
\begin{definition}
    Depth limit of $d_{\text{max}}$ to any search algorithm by modifying \texttt{SEARCH($\cdot$)} as follows:
\begin{lstlisting}
procedure SEARCHDL($\mathcal{O}$, $d_{\text{max}}$):
    if $\mathcal{O} = \emptyset$ then
        return NULL (*\hfill $\triangleright$ the search algorithm failed to find a path to a goal*)
    $n \leftarrow \text{REMOVE}(\mathcal{O})$ (*\hfill $\triangleright$ "explore" a node, $n$*)
    if dst($n$) $\in \mathcal{G}$ then
        return $n$ (*\hfill $\triangleright$ the search algorithm found a path to a goal*)
    for $n' \in \text{chl}(n)$ do (*\hfill $\triangleright$ "expand" $n$ and "export" its children*)
        if len($n'$) $\leq d_{\text{max}}$ then (*\hfill $\triangleright$ unless the child is too long*)
            $\mathcal{O} \leftarrow \mathcal{O} \cup \{n'\}$
    SEARCHDL($\mathcal{O}$, $d_{\text{max}}$)
\end{lstlisting}

\end{definition}

\subsubsection{Iterative Deepening}
\begin{definition}
    Iteratively increase the depth-limit, $d_{\max}$, to any search algorithm w/ depth-limiting, by placing \texttt{SEARCHDL($\cdot$)} in a wrapper, \texttt{SEARCHID($\cdot$)}:
\begin{lstlisting}
procedure SEARCHID():
    $n \leftarrow \text{NULL}$
    $d_{\text{max}} \leftarrow 0$
    (*$\triangleright$ while a solution has not been found, reset the open set, run the search algorithm, then increase the depth-limit*)
    while $n = \text{NULL}$ do
        $\mathcal{O} \leftarrow \{(\langle \rangle, 0)\}$
        $n \leftarrow \text{SEARCHDL}(\mathcal{O}, d_{\text{max}})$
        $d_{\text{max}} \leftarrow d_{\text{max}} + 1$
    return $n$
\end{lstlisting}
    
\end{definition}

\begin{warning}
    Increasing $d_{\text{max}}$ can be done in different ways.
\end{warning}

\subsubsection{Cost-Limiting}
\begin{definition}
    Cost limit of $c_{\text{max}}$ to any search algorithm by modifying \texttt{SEARCH($\cdot$)} as follows:

\begin{lstlisting}
procedure SEARCHCL($\mathcal{O}$, $c_{\text{max}}$):
    if $\mathcal{O} = \emptyset$ then
        return NULL (*\hfill $\triangleright$ the search algorithm failed to find a path to a goal*)
    $n \leftarrow \text{REMOVE}(\mathcal{O})$ (*\hfill $\triangleright$ "explore" a node, $n$*)
    if dst($n$) $\in \mathcal{G}$ then
        return $n$ (*\hfill $\triangleright$ the search algorithm found a path to a goal*)
    for $n' \in \text{chl}(n)$ do (*\hfill $\triangleright$ "expand" $n$ and "export" its children*)
        if cst($n'$) $\leq c_{\text{max}}$ then (*\hfill $\triangleright$ unless the child is too expensive*)
            $\mathcal{O} \leftarrow \mathcal{O} \cup \{n'\}$
    SEARCHCL($\mathcal{O}$, $c_{\text{max}}$)
\end{lstlisting}

\end{definition}

\subsubsection{Iterative-Inflating}
\begin{definition}
    Iteratively increase the cost limit, $c_{\text{max}}$, to any search algorithm with cost-limiting, by placing \texttt{SEARCHCL($\cdot$)} in a wrapper, \texttt{SEARCHII($\cdot$)}:

\begin{lstlisting}
procedure SEARCHII():
    $n \leftarrow \text{NULL}$
    $c_{\text{max}} \leftarrow 0$
    (*$\triangleright$ while a solution has not been found, reset the open set, run the search algorithm, then increase the cost-limit*)
    while $n = \text{NULL}$ do
        $\mathcal{O} \leftarrow \{(\langle \rangle, 0)\}$
        $n \leftarrow \text{SEARCHCL}(\mathcal{O}, c_{\text{max}})$
        $c_{\text{max}} \leftarrow c_{\text{max}} + \epsilon$
    return $n$
\end{lstlisting}

\end{definition}

\begin{warning}
    Increasing $c_{\text{max}}$ can be done in different ways.
\end{warning}

\subsubsection{Intra-Path Cycle Checking}
\begin{definition}
    Do not expand a path if it is cyclic. Modify \texttt{SEARCH($\cdot$)} as follows:

\begin{lstlisting}
procedure SEARCH($\mathcal{O}$):
    if $\mathcal{O} = \emptyset$ then
        return NULL
    $n \leftarrow \text{REMOVE}(\mathcal{O})$
    if dst($n$) $\in \mathcal{G}$ then
        return $n$
    for $n' \in \text{chl}(n)$ do (*\hfill $\triangleright$ "expand" $n$ and "export" its children*)
        if not CYCLIC($n'$) then (*\hfill $\triangleright$ unless the child is cyclic*)
            $\mathcal{O} \leftarrow \mathcal{O} \cup \{n'\}$
    SEARCH($\mathcal{O}$)
\end{lstlisting}
\begin{itemize}
    \item Optimately of an algorithm is preserved provided $\epsilon>0$.
\end{itemize}

\end{definition}

\subsubsection{Inter-Path Cycle Checking}
\begin{definition}
    We modify \texttt{SEARCH($\cdot$)} as follows:

\begin{lstlisting}
procedure SEARCH($\mathcal{O}$, $\mathcal{C}$):
    if $\mathcal{O} = \emptyset$ then
        return NULL
    $n \leftarrow \text{REMOVE}(\mathcal{O})$
    $\mathcal{C} \leftarrow \mathcal{C} \cup \{n\}$ (*\hfill $\triangleright$ add $n$ to the closed set*)
    if dst($n$) $\in \mathcal{G}$ then
        return $n$
    for $n' \in \text{chl}(n)$ do (*\hfill $\triangleright$ "expand" $n$ and "export" its children*)
        if $n' \notin \mathcal{C}$ then (*\hfill $\triangleright$ unless the child's destination is closed*)
            $\mathcal{O} \leftarrow \mathcal{O} \cup \{n'\}$
    SEARCH($\mathcal{O}$, $\mathcal{C}$)
\end{lstlisting}

and then call the algorithm as follows:

\begin{lstlisting}[mathescape=true, escapeinside={(*}{*)}, numbers=left, frame=single]
$\mathcal{O} \leftarrow \{(\langle \rangle, 0)\}$
$\mathcal{C} \leftarrow \emptyset$ (*\hfill $\triangleright$ initialize a set of closed vertices*)
SEARCH($\mathcal{O}$, $\mathcal{C}$)
\end{lstlisting}

\end{definition}

\subsection{Informed Search Algorithms}
\subsubsection{Estimated Cost}
\begin{definition}
    $\text{ecst}(\cdot)$, to estimate the total cost to a goal given a path, $p$, based on the following:
    \begin{itemize}
        \item Cost of path $p$: $\text{cst}(p)$
        \item Estimate of the extra cost needed to get to a goal from $\text{dst}(p)$: $\text{hur} : S \to \mathbb{R}_+$
        \begin{itemize}
            \item $\text{hur}(s)$ estimates the cost to get to $\mathcal{G}$ from $s$ and $\text{hur}(p)$ means $\text{hur}(\text{dst}(p))$.
        \end{itemize}
    \end{itemize}
\end{definition}

\begin{example}
    Some common choices for $\text{ecst}(\cdot)$ include:
    \begin{enumerate}
        \item $\text{ecst}(p) = \text{hur}(p)$; called nearest-first search (NFS)
        \item $\text{ecst}(p) = \text{cst}(p) + \text{hur}(p)$; called A$^*$ (A-star)
    \end{enumerate}
\end{example}

\subsection{Characteristics of an Informed Search Algorithm}
\begin{definition}
    \begin{enumerate}
        \item Heuristic: $\text{hur}(\cdot)$
        \item Cost estimation: $\text{ecst}(\cdot)$
    \end{enumerate}
\end{definition}
\subsubsection{Heuristics}


\subsubsection{Heuristic-First Search (HFS)}

\subsubsection{A-Star Search (A*)}

\subsubsection{Iterative Inflating A-Star Search (IIA*)}

\subsubsection{Designing Heuristics via Problem Relaxation}

\subsubsection{Combining Heuristics}

\subsection{Anytime Search Algorithms}

\subsection{Formulating a Search Problem}







\newpage

\section{MLP}
\subsection{2 RVs}
\begin{notes}
    RVs are neither random nor a variable. 
    \begin{equation*}
        \underline{Z} = (X,Y)
    \end{equation*}
    \customFigure[0.5]{../Images/L3_0.png}{Mapping of RVs}
\end{notes}

\subsection{Joint PMF/PDF}
\begin{definition}
    \begin{equation}
    P_{X,Y}(x, y) = P[X = x, Y = y]
    \end{equation}
    
    \begin{equation}
    f_{X,Y}(x, y) = \frac{\partial^2}{\partial x \partial y} F_{X,Y}(x, y)
    \end{equation}
    
    \begin{equation}
    P[(X, Y) \in A] = \int \int_{(x, y) \in A} f_{X,Y}(x, y) \, dx \, dy
    \end{equation}
\end{definition}

\begin{example} Jointly Gaussian RVs $X$ and $Y$ with ($\mu_1, \mu_2, \sigma_1^2, \sigma_2^2, \rho$)
    \[
    f_{X,Y}(x, y) = \frac{1}{2\pi \sigma_1 \sigma_2 \sqrt{1-\rho^2}} 
    \exp \left\{ 
    -\frac{1}{2(1-\rho^2)} 
    \left[ 
    \left(\frac{x-\mu_1}{\sigma_1}\right)^2 
    - 2\rho \left(\frac{x-\mu_1}{\sigma_1}\right) \left(\frac{y-\mu_2}{\sigma_2}\right) 
    + \left(\frac{y-\mu_2}{\sigma_2}\right)^2 
    \right] 
    \right\}
    \]
\end{example}

\subsection{Expectations}
\begin{definition}
    \[
    E[g(X, Y)] = \int_{-\infty}^{\infty} \int_{-\infty}^{\infty} g(x, y) f_{X,Y}(x, y) \, dx \, dy
    \]
\end{definition}

\begin{notes}
    \begin{itemize}
        \item $g(X,Y)$ is also an RV, but inside the integral or sum, you use $x$ and $y$ as dummy variables to vary through the values of the RVs.
    \end{itemize}
\end{notes}

\subsubsection{Correlation}
\begin{definition}
    \begin{equation}
        E[XY]
    \end{equation}
\end{definition}

\subsubsection{Covariance}
\begin{definition}
    \begin{equation}
        \text{Cov}[X, Y] = E[(X - \mu_X)(Y - \mu_Y)] = E[XY] - \mu_X \mu_Y = E[XY] - E[X]E[Y]
    \end{equation}
\end{definition}

\begin{notes}
    \begin{itemize}
        \item Mean shifted to 0.
    \end{itemize}
\end{notes}

\subsubsection{Correlation Coefficient}
\begin{definition}
    \begin{equation}
        \rho_{X,Y} = E \left[ \left(\frac{X - \mu_X}{\sigma_X} \right) \left( \frac{Y - \mu_Y}{\sigma_Y} \right) \right] = \frac{\text{Cov}[X, Y]}{\sigma_X \sigma_Y}
    \end{equation}
    \begin{itemize}
        \item $|\rho_{X,Y}| \leq 1$
    \end{itemize}
\end{definition}

\begin{notes}
    \begin{itemize}
        \item Mean shifted to 0 and normalized by the standard deviation.
    \end{itemize}
\end{notes}

\subsection{Marginal PMF/PDF}
\begin{definition}
    \begin{equation}
    P_X(x) = \sum_{j=1}^{\infty} P_{X,Y}(x, y_j), \quad P_Y(y) = \sum_{j=1}^{\infty} P_{X,Y}(x_j, y)
    \end{equation}
    
    \begin{equation}
    f_X(x) = \int_{-\infty}^{\infty} f_{X,Y}(x, y) \, dy, \quad f_Y(y) = \int_{-\infty}^{\infty} f_{X,Y}(x, y) \, dx
    \end{equation}
\end{definition}

\begin{notes}
    \begin{itemize}
        \item Total probability theorem is being used here.
    \end{itemize}
\end{notes}

\begin{example} Jointly Gaussian $X$ and $Y$:
    \begin{align*}
        f_X(x) &= \int_{-\infty}^{\infty} f_{X,Y}(x, y) \, dy \\
               &= \dots \quad (\text{completing the square}) \\
               &= \frac{1}{\sqrt{2\pi} \sigma_1} e^{-\frac{(x-\mu_1)^2}{2\sigma_1^2}}, \quad \text{marginally Gaussian}
    \end{align*}
    \begin{itemize}
        \item Gaussian RVs has a property that the PDF of a single variable is equal to the marginal Gaussian of two variables.
    \end{itemize}
\end{example}

\subsection{Conditional PMF/PDF}
\begin{definition}
    \begin{equation}
    P_{X|Y}(x|y) \triangleq P[X = x | Y = y] = \frac{P_{X,Y}(x, y)}{P_Y(y)}
    \end{equation}
    
    \begin{equation}
    f_{X|Y}(x|y) \triangleq \frac{f_{X,Y}(x, y)}{f_Y(y)}
    \end{equation}
\end{definition}

\subsection{Bayes' Rule}
\begin{definition}
    \begin{equation}
    P_{Y|X}(x|y) = \frac{P_{X,Y}(x, y)}{P_X(x)} = \frac{P_{X|Y}(x|y) P_Y(y)}{\sum_{j=1}^\infty P_{X,Y}(x, y_j) P_Y (y_j)}
    \end{equation}
    
    \begin{equation}
    f_{Y|X}(y|x) = \frac{f_{X,Y}(x, y)}{f_X(x)} = \frac{f_{X|Y}(x|y) f_Y(y)}{\int_{-\infty}^\infty f_{X|Y}(x|y') f_Y(y') \, dy'}
    \end{equation}  
\end{definition}

\subsection{Independent vs. Uncorrelated vs. Orthogonal}
\begin{definition} 
    \begin{enumerate}
        \item Independent:
        \begin{equation}
        f_{X|Y}(x|y) = f_X(x) \; \forall y
        \Leftrightarrow 
        f_{X,Y}(x, y) = f_X(x) f_Y(y) 
        \end{equation}
        \item Uncorrelated:
        \begin{equation}
        \text{Cov}[X, Y] = 0 \quad \Leftrightarrow \quad \rho_{X,Y} = 0
        \end{equation}
        \item Orthogonal:
        \begin{equation}
        E[XY] = 0
        \end{equation}
    \end{enumerate}
\end{definition}

\begin{theorem}
    If independent, then uncorrelated.
\end{theorem}

\begin{derivation}
    \begin{align*}
    \text{Independent} & \implies E[XY] = \int_{-\infty}^{\infty} \int_{-\infty}^{\infty} x y f_{X,Y}(x, y) \, dx \, dy \\
    &= \int_{-\infty}^{\infty} \int_{-\infty}^{\infty} x y f_X(x) f_Y(y) \, dx \, dy \\
    &= \left( \int_{-\infty}^{\infty} x f_X(x) \, dx \right) \left( \int_{-\infty}^{\infty} y f_Y(y) \, dy \right) \\
    &\implies E[XY] = E[X] E[Y] \\
    &\implies \text{Cov}[X, Y] = 0, \quad \text{uncorrelated} \\
    &\not\Leftarrow \text{in general.}
    \end{align*}
\end{derivation}

\begin{example} Jointly Gaussian RVs $X$ and $Y$: If uncorrelated, i.e. $\rho_{X,Y} = 0$, then $X$ and $Y$ are independent.
    \begin{align*}
    f_{X,Y}(x, y) &= \frac{1}{2\pi \sigma_1 \sigma_2} 
    \exp \left\{ 
    -\frac{1}{2} 
    \left[ 
    \left(\frac{x-\mu_1}{\sigma_1}\right)^2 
    + 
    \left(\frac{y-\mu_2}{\sigma_2}\right)^2 
    \right] 
    \right\} \\
    &= \frac{1}{\sqrt{2\pi} \sigma_1} e^{-\frac{(x-\mu_1)^2}{2\sigma_1^2}} 
    \cdot 
    \frac{1}{\sqrt{2\pi} \sigma_2} e^{-\frac{(y-\mu_2)^2}{2\sigma_2^2}} \\
    &= f_X(x) f_Y(y) \quad \text{independent}
    \end{align*}
\end{example}

\subsection{Conditional Expectation}
\begin{definition}
    \begin{equation}
        E[Y] = E[E[Y|X]]
    \end{equation}
    \begin{equation}
        E[h(Y)] = E[E[h(Y)|X]]
    \end{equation}
\end{definition}

\begin{notes}
    \begin{itemize}
        \item $E[E[Y|X]]$ is w.r.t. $X$.
        \item $E[Y|X]$ is w.r.t. $Y$.
    \end{itemize}
\end{notes}

\begin{derivation}
    \begin{align*}
    E[Y] &= \int_{-\infty}^\infty \int_{-\infty}^\infty y f_{X,Y}(x, y) \, dx \, dy \\
            &= \int_{-\infty}^\infty \int_{-\infty}^\infty y f_{Y|X}(y|x) f_X(x) \, dx \, dy \\
            &= \int_{-\infty}^\infty \left( \int_{-\infty}^\infty y f_{Y|X}(y|x) \, dy \right) f_X(x) \, dx \\
            &= \int_{-\infty}^\infty E[Y|X=x] f_X(x) \, dx \quad \text{(using the total probability theorem)} \\
            &= \int_{-\infty}^\infty g(x) f_X(x) \, dx \\
            &= E[g(X)] \\ 
            &= E[E[Y|X]].
    \end{align*}
\end{derivation}

\begin{example}
    \begin{enumerate}
        \item \textbf{Given:} An unknown voltage. \( X \sim \text{Uniform}(0,1) \). Measurement from a (bad) voltmeter: \( Y \sim \text{Uniform}(0, X) \).
    
        \begin{align*}
            f_X(x) &= 
            \begin{cases} 
                1, & 0 < x < 1 \\ 
                0, & \text{otherwise}
            \end{cases} \\
            f_{Y|X}(y|x) &= 
            \begin{cases} 
                \frac{1}{x}, & 0 < y < x \\ 
                0, & \text{otherwise}
            \end{cases}
        \end{align*}
        \begin{itemize}
            \item \textbf{Note:} Area under PDF is 1.
        \end{itemize}

        \customFigure[0.25]{../Images/L3_1.png}{Uniform Distribution of $X$}
        \customFigure[0.25]{../Images/L3_2.png}{Uniform Distribution of $Y$}

    
        \item \textbf{Expected Value (Average Reading of Bad Voltmeter):}
        \begin{align*}
            E[Y] &= E[E[Y|X]] \\
                 &= E\left[\frac{X}{2}\right] \quad \text{Since in the middle of 0 and x}\\
                 &= \frac{1}{2} \cdot E[X] \\ 
                 &= \frac{1}{2} \cdot \frac{1}{2} = \frac{1}{4} \quad \text{Since $E[X]$ (i.e. mean) is 0.5}
        \end{align*}
    
        \item \textbf{The Long Way:}
        \begin{align*}
            f_Y(y) &= \int_{-\infty}^{\infty} f_{Y|X}(y|x) f_X(x) \, dx \\
                   &= \int_{y}^1 f_{Y|X}(y|x) f_X(x) \, dx \\
                   &= \int_{y}^1 \frac{1}{x} \cdot 1 \, dx \\
                   &= -\ln y. \\
            E[Y] &= \int_{0}^1 y \cdot (-\ln y) \, dy = \dots = \frac{1}{4}
        \end{align*}
    
        \item \textbf{Question:} Suppose \( Y = \frac{1}{8} \). What is "best" given \( X \)? This will be the quesiton for the rest of the course.
    \end{enumerate}    
\end{example}
\newpage

\section{Neural Network Engineering}
\section{Learning Problems}
\begin{definition}
    In a learning problem, we assume that there is some (unknown) relationship, 
    \begin{equation*}
        f: \mathcal{X} \rightarrow \mathcal{Y}
    \end{equation*}
    s.t. $x \mapsto_f y$
    \vspace{1em}

    Find $h: \mathcal{X} \rightarrow \mathcal{Y}$ (hypothesis) s.t. $h \approx f$, given some data about $f$: 

    \begin{itemize}
        \item $\text{in}(\mathcal{D}) = \{x \text{ s.t. } (x,y) \in \mathcal{D}\}$
        \item $\text{out}(\mathcal{D}) = \{y \text{ s.t. } (x,y) \in \mathcal{D}\}$
    \end{itemize}
\end{definition}

\subsection{Classification vs. Regression Problems}
\begin{definition}
    \begin{itemize}
        \item \textbf{Classification Problems:} $\mathcal{X} \subseteq \mathbb{R}^n$ and $\mathcal{Y} \subseteq \mathbb{N}$
        \item \textbf{Regression Problems:} $\mathcal{X} \subseteq \mathbb{R}^n$ and $\mathcal{Y} \subseteq \mathbb{R}$
    \end{itemize}
\end{definition}

\subsection{Feature Spaces}
\begin{definition}
    It is often easier to learn relationships from high-level features (instead of the raw input).
\end{definition}

\subsection{Feasibility of Learning}
\begin{motivation}
    More than one function (hypothesis) may be consistent with the data.
\end{motivation}

\begin{notes}
    So it may appear that finding the correct one should be impossible. 
\end{notes}

\subsubsection{Probably Approximately Correct (PAC) Estimations}
\begin{example}
    Take $N$ i.i.d. samples (i.e. take out a ball from an urn, record its color, and put it back in).
    \begin{itemize}
        \item $\nu \rightarrow \mu$ (empirical distribution $\rightarrow$ true distribution) as $N \rightarrow \infty$
    \end{itemize}
\end{example}

\subsubsection{Hoeffding's Inequality}
\begin{definition}
    Let $\mu$ denote the probability of an event, and $\nu$ denote its relative frequency in a sample size $N$. Then, for any $\epsilon > 0$,
    \begin{equation}
        P(|\nu - \mu| > \epsilon) \leq 2e^{-2\epsilon^2N}
    \end{equation}
    \begin{itemize}
        \item $\nu$: Relative frequency in the sample (known)
        \item $\mu$: Probabillity of drawing a blue ball (unknown)
        \item $N \rightarrow \infty$: $\nu \rightarrow \mu$
        \item $\epsilon$: How close we want $\nu$ to be to $\mu$
        \item $\epsilon \rightarrow 0$: Probability will be 1
        \item $\epsilon \rightarrow \infty$: $\nu \rightarrow \mu$
        \item $\mu \overset{?}{\approx} \nu $: $\mu$ is probably approximately equal to $nu$.
    \end{itemize}
\end{definition}

\begin{warning}
    We can approximate the true distribution with high probability by taking a large enough sample size, NOT guaranteeing that we can find the true distribution.
    \begin{itemize}
        \item Don't need to know where this theorem comes from.
    \end{itemize}
\end{warning}

Consider determining the class of a randomly chosen target point. If we ask a K-ary question about the points in $\mathcal{D}$

\subsubsection{PAC Learning}

\subsection{Decision Trees}

\subsubsection{Structure of a Decision Tree}


\newpage

\section{Hyperparameter Optimization}
\begin{summary}
    \begin{itemize}
        \item What stategies can help a NN converge when training?
        \item What hyperparameters does a NN architecture have?
        \item How can we optimize parameters without gradients? 
        \item DL requires a lot of data, what can we do when data is scarce?
    \end{itemize}
\end{summary}

\section{Blackbox Optimization}
\begin{motivation}
    Also known as derivative free optimization, as the derivative is unknown, so you have to use derivative-free or heuristic methods. 
    \begin{equation*}
        x^* = \arg\min_{x \in X} f(x)
    \end{equation*}
    \customFigure[0.5]{../Images/L5_2.png}{}
\end{motivation}

\subsection{Parameters \& Hyperparameters}
\begin{definition}
    Distinction b/w model setting elements and tuning knobs.
    \begin{itemize}
        \item \textbf{Parameters:} Learnable parameters $(W,b)$
        \begin{itemize}
            \item Opt: Gradient Descent
        \end{itemize}
        \item \textbf{Hyperparameters:} Non-differentiable parameters (i.e. discrete)
        \begin{itemize}
            \item E.g. Number of layers, hidden dim, activation, normalization, dropout, ...
            \item Opt: Heuristics
        \end{itemize}
        \customFigure[0.5]{../Images/L5_0.png}{}
    \end{itemize}
\end{definition}
\newpage

\begin{summary}
    \begin{center}
        \begin{tabular}{l}
        \toprule
        \textbf{Types} \\
        \midrule
        \textbf{Grid Search} \\
        \multicolumn{1}{p{\linewidth}}{
        \begin{itemize}
            \item Exhaustive evaluation across a predefined set of values.
            \customFigure[0.3]{../Images/L5_3.png}{}
        \end{itemize}} \\
        \midrule
        \textbf{Coordinate Descent} \\
        \multicolumn{1}{p{\linewidth}}{
        \begin{itemize}
            \item Optimize each hyperparameter one at a time. 
            \customFigure[0.3]{../Images/L5_4.png}{}
        \end{itemize}} \\
        \midrule
        \textbf{Grad-Student Descent} \\
        \multicolumn{1}{p{\linewidth}}{
        \begin{itemize}
            \item Manual and ad-hoc, i.e. "follow your heart"
            \customFigure[0.3]{../Images/L5_5.png}{}
        \end{itemize}} \\
        \midrule
        \textbf{Random Search} \\
        \multicolumn{1}{p{\linewidth}}{
        \begin{itemize}
            \item Sampling hyperparameter configurations from defined distributions.
            \customFigure[0.3]{../Images/L5_6.png}{}
        \end{itemize}} \\
        \bottomrule
        \end{tabular}
    \end{center}
\end{summary}

\section{Bayesian Optimization}
\begin{definition}
    A principled sequential approach for efficient global optimization
    \vspace{1em}

    \textbf{Ingredients:}
    \begin{itemize}
        \item \textbf{Function, $f(x)$}: The numerical values we want to optimize.
        \item \textbf{Space to optimize, $X$}: Parameters or decisions or degrees of freedom to explore.
        \item \textbf{Bayesian model, $g(x)$}: Provides prediction ($\mu$) and uncertainty ($\sigma$).
        \item \textbf{Acquisition function, $A(\mu, \sigma)$}: A strategy to trade off exploration \& exploitation.
    \end{itemize}
\end{definition}

\subsection{Surrogate Model}
\begin{notes}
    Let's approximate the expensive function $f(x)$ with a cheaper function $g(x)$ to model prediction ($\mu$) and uncertainty ($\sigma$).
    \begin{itemize}
        \item Kernel models
        \item Gaussian processes (GP)
        \item Gradient boosted trees
        \item Neural networks
    \end{itemize}
\end{notes}

\subsubsection{Acquisition Function}
\begin{notes}
    Let's mix exploitation and exploration; sometimes, it pays off to explore areas where we have little information.
    \begin{itemize}
        \item Acquisition functions encapsulate the heuristic of what to sample next, how useful is unobserved data?
        \item E.g. Expected Improvement (EI), Probability of Improvement (PI), Upper Confidence Bound (UCB), ...
        \item $\mu$: exploitation 
        \item $\sigma$: exploration 
    \end{itemize}
\end{notes}

\subsection{BayesOpt Loop}
\begin{definition}
    Iterative process of modelling and sampling. 
    \begin{enumerate}
        \item Set a termination criteria (budget, iterations, maxima)
        \item Evaluate $f(x)$ on initial set of points (random)
        \item \textbf{Loop:} while criteria is not met:
        \begin{enumerate}
            \item Update surrogate model on all data
            \item Optimize acquisition function to find a maxima ($x_{\text{new}}$)
            \item Evaluate $f(x_{\text{new}}$)
        \end{enumerate}
    \end{enumerate}
\end{definition}

\begin{warning}
    \begin{itemize}
        \item Define your hyperparameter space (bounds, datatypes, etc.)
        \item Simplify it. 
        \item Use a platform to launch monitor and launch models. 
        \item \textbf{Libraries:} Optuna, Ray, BoTorch, Ax...
    \end{itemize}
\end{warning}

\begin{example}
    \begin{itemize}
        \item 2 random points
        \item Criteria: 10 evaluations.
        \item Normalize $f(x)$ to a smaller range b/c gradients are sensitive to scale.
        \item Repeat for 10 iterations:
        \begin{itemize}
            \item Pick points that maximizes acquistion function.
            \item Update surrogate model w/ the new point and its evaluation, i.e. $f(x_i)$ to get more certainty and better predictions.
        \end{itemize}
    \end{itemize}
    \vspace{1em}

    Look at L5.
\end{example}


\newpage

\section{Representations and VAE}
\subsection{Bayesian Network}
\begin{definition}
    Vertices represent random variables and edges represent dependencies between variables.
\end{definition}

\subsubsection{Junction}
\begin{definition}
    A \textbf{junction} consists of three vertices, $X_1$, $X_2$, and $X_3$, connected by two edges, $e_1$ and $e_2$:
    \begin{itemize}
        \item Both arrows pointing in one direction
        \item Both arrows pointing in opposite directions
        \item One arrow pointing in each direction
    \end{itemize}
\end{definition}

\begin{warning}
    Want to look for causal relationships. 
    Arrows, what's causing what, what's influencing what.
\end{warning}

\subsubsection{Causal Chain}
\begin{definition}
    A causal chain is a junction of the following form:
    \begin{itemize}
        \item $X_1$ and $X_2$ are dependent. $X_2$ is dependent on $X_1$. Vice versa. From a causal perspective, $X_1$ is influencing $X_2$. Subtle difference, just bc $X_1 \rightarrow X_2$.
        \item $X_2$ and $X_3$ are dependent.
        \item $X_1$ and $X_3$ are dependent. 
        \begin{itemize}
            \item Given $X_2$, $X_1$ and $X_3$ are independent. Why? $X_2$'s door closes when you know $X_2$, so $X_1$ and $X_3$ are independent.
        \end{itemize}
    \end{itemize}
\end{definition}

\begin{warning}
    $X_1$ is influeincing $X_2$ and $X_2$ is influencing $X_3$.
\end{warning}

\subsubsection{Common Cause}
\begin{definition}
    A common cause is a junction of the form: 
\end{definition}

\begin{notes}
    \begin{itemize}
        \item $X_1$ and $X_3$ are dependent. 
        \begin{itemize}
            \item Given $X_2$, $X_1$ and $X_3$ are independent. Why? $X_2$ whether you smoke or not, $X_1$ whether you have yellow teeth, $X_3$ whether you have lung cancer, if you don't know $X_2$, if they have yellow teeth, then they might smoke, then they might have lung cancer. If you know $X_2$, yellow teeth and lung cancer are independent b/c you already know if they smoke or not, and yellow teeth implies smoke, 
        \end{itemize}
    \end{itemize}
\end{notes}

\subsubsection{Common Effect}
\begin{definition}
    A common effect is a junction of the form:
    
\end{definition}

\begin{notes}
    \begin{itemize}
        \item $X_1$ and $X_3$ are independent. 
        \item Given $X_2$ or any of $X_2$'s descendents, $X_1$ and $X_3$ are dependent.
    \end{itemize}
\end{notes}

\begin{warning}
    Just b/c you don't know something about the middle variable, then it can be independent
\end{warning}

\begin{example}
    $X_2$ Grass being wet, $X_1$ raining, and $X_3$ sprinkler being on.
    \begin{itemize}
        \item If you know the grass is wet, you know that either the sprinkler is on or it's raining.
        \begin{itemize}
            \item If it didn't have the sprinkler on, then it must have rained.
            \item If it didn't rain, then the sprinkler must have been on.
            \item So this means that $X_1$ and $X_3$ are dependent given $X_2$.
        \end{itemize}
        \item If you don't know the grass is wet, then $X_1$ and $X_3$ are independent b/c you don't know if it rained or the sprinkler was on.
    \end{itemize}
\end{example}

\begin{example}
    \begin{enumerate}
        \item \textbf{Given:} Caveman is deciding whether to go hunt for meat. He must take into account several factors:
        \begin{itemize}
            \item Weather
            \item Possibility of over-exertion
            \item Possibility encountering lion
        \end{itemize}

        These factors can result in Cavemen's death. His decision will ultimately depend on the \textbf{chances} of his death.
        \item \textbf{Binary Variables:}
        \begin{itemize}
            \item $W = \{\text{Sun}, \text{Rainy}\}$: Weather
            \item $H$: Whether the Cavemen goes hunting or not.
            \item $L$: Whether the Cavemen encounters a lion or not.
            \item $T$: Whether the Cavement is tired or not.
            \item $D$: Whether the Cavemen dies or not
        \end{itemize}
        \item \textbf{Problem:} Cavemen must decide whether to go hunting or not. 
        \begin{itemize}
            \item He must consider the conditional probabilities (i.e. dependence) of each event.
        \end{itemize}
    \end{enumerate}
\end{example}

\begin{warning}
    Have to be discrete. 
\end{warning}


\newpage

\section{Software Development}
\subsection{Software Engineering}
\begin{summary}
    \begin{center}
        \begin{tabular}{ll}
            \toprule
            \textbf{Concept} & \textbf{Description} \\
            \midrule
            Code Readability & Maintainable code enables collaboration and future work. \\
            \multicolumn{2}{p{\linewidth}}{
                \centering
                \begin{minipage}{0.48\linewidth}
                    \centering
                    \customFigure[1.0]{../Images/L7_4.png}{Image 1 Description}
                \end{minipage}
                \hfill
                \begin{minipage}{0.48\linewidth}
                    \centering
                    \customFigure[1.0]{../Images/L7_1.png}{Image 2 Description}
                \end{minipage} 
            } \\
            \bottomrule
        \end{tabular}
    \end{center}
\end{summary}
\subsection{Code Readability Matters}
\begin{notes}
    \begin{itemize}
        \item \textbf{Overview:} Maintainable code enables collaboration and future work.
        \customFigure[0.5]{../Images/L7_1.png}{}
        \customFigure[0.5]{../Images/L7_4.png}{}
        \item \textbf{Naming:} Clear names enhance code understanding 
        \begin{itemize}
            \item \textbf{Functions:} verb\_do
            \item \textbf{Variables:} two\_three\_words
            \item \textbf{Classes:} CapitalizedWords
        \end{itemize}
        \customFigure[0.5]{../Images/L7_2.png}{}
    \end{itemize}
\end{notes}

\subsection{Style Guides}
\begin{notes}
    \begin{itemize}
        \item \textbf{Overview:} Ensure consistent code formatting. 
        \item \textbf{Why?} Provide a standardized set of rules for formatting code, prompting uniformity, and reducing cognitive load when reading code. 
        \item \textbf{Examples:} PEP8, Google Style Guide, etc.
    \end{itemize}
    \customFigure[0.5]{../Images/L7_3.png}{}
    \customFigure[0.5]{../Images/L7_5.png}{}
\end{notes}
\newpage

\section{Coding Mantras / Ideas}
\subsection{Zen of Python}
\begin{definition}
    \customFigure[0.5]{../Images/L7_0.png}{}
\end{definition}

\subsection{KISS: Keep It Simple and Straightforward}
\begin{notes}
    Advocate for solutions that are easy to understand and maintain, reduce unnecessary complexity.
    \customFigure[0.5]{../Images/L7_6.png}{}
\end{notes}

\subsection{YAGNI: You Aren't Gonna Need It}
\begin{notes}
    AKA premature optimization is the root of all evil. 
    \customFigure[0.5]{../Images/L7_7.png}{}
    \customFigure[0.5]{../Images/L7_8.png}{}
\end{notes}

\subsection{Principle of DRY and WET and DAMP}
\begin{notes}
    Don't Repeat Yourself; Write Everything Twice.
    \customFigure[0.5]{../Images/L7_9.png}{}
    \vspace{1em}

    Don't Abstract Methods Prematurely.
    \customFigure[0.5]{../Images/L7_10.png}{}
\end{notes}
\newpage

\section{Into The Weeds (Technical Details)}
\subsection{Python Data Containers Overview}
\begin{definition}
    Choosing the right container for the task:
    \begin{itemize}
        \item \textbf{list:} Ordered, mutable sequence; versatile for collections of items.
        \item \textbf{tuple:} Ordered, immutable sequence; suitable for fixed collections and data integrity.
        \item \textbf{set:} Unordered collection of unique elements; efficient for membership testing and removing duplicates.
        \item \textbf{dict:} Key-value mappings; ideal for efficient data lookup and representing structured information.
    \end{itemize}
    \vspace{1em}
    
    \textbf{Data Structures:}
    \begin{itemize}
        \item \textbf{collections.NamedTuple}
        \item \textbf{dataclasses.Dataclass}
    \end{itemize}
    \vspace{1em}

    \textbf{Tensors}
    \begin{itemize}
        \item \textbf{numpy.ndarray}
        \item \textbf{torch.Tensor}
        \item \textbf{jax.Array}
    \end{itemize}
\end{definition}

\subsection{List Comprehensions}
\begin{notes}
    Provide concise list creation.
    \customFigure[0.5]{../Images/L7_11.png}{}
\end{notes}

\subsection{Leveraging Set Data Structures: Set}
\begin{notes}
    Efficiently handle unique elements and set logic.
    \customFigure[0.5]{../Images/L7_12.png}{}
\end{notes}

\subsection{Dictionaries for Key-Value Pairs}
\begin{notes}
    Enable efficient data lookup by keys.
    \customFigure[0.5]{../Images/L7_13.png}{}
\end{notes}

\subsection{Itertools for Efficient Iteration}
\begin{notes}
    Provides tools for very common iteration patterns.
    \begin{itemize}
        \item e.g. permutations, chunked, chain, zip, filter, product, combinations, etc.
    \end{itemize}
    \customFigure[0.5]{../Images/L7_14.png}{}
    \customFigure[0.5]{../Images/L7_15.png}{}
\end{notes}

\subsection{Functools for Functional Tools}
\begin{notes}
    Need to manipulate a function and get a new function. 
    \begin{itemize}
        \item Caching with \texttt{functools.lru\_cache}: Cache optimizes performance by storing results (i.e. return a previous result when the input of a function has already been observed)
    \end{itemize}
    \customFigure[0.5]{../Images/L7_16.png}{}
    \customFigure[0.5]{../Images/L7_17.png}{}
\end{notes}

\subsection{Abstract Base Classes (ABCs)}
\begin{notes}
    ABCs enforce interfaces for robust design.
    \customFigure[0.5]{../Images/L7_18.png}{}
\end{notes}

\subsection{Dataclasses for Data Storage}
\begin{notes}
    Data classes simplify data-centric class creation.
    \customFigure[0.5]{../Images/L7_19.png}{}
\end{notes}

\subsection{Python Typing Provides Hints}
\begin{notes}
    Type hints enhance code clarity, communicate intent and detect errors.
    \customFigure[0.5]{../Images/L7_20.png}{} 
\end{notes}

\subsection{Tensor Typing: jaxtyping}
\begin{notes}
    Communicate expected tensor shapes and data types. 
    \customFigure[0.5]{../Images/L7_21.png}{}
\end{notes}

\subsection{pytest: Helps you write better programs}
\begin{notes}
    Simplifies testing for reliable code.
    \customFigure[0.5]{../Images/L7_22.png}{}
\end{notes}

\subsection{Ideas}
\begin{summary}
    \begin{center}
        \begin{tabular}{l}
            \toprule
            \textbf{Exercise} \\
            \toprule
            Write research ideas. Get a mentor to rate them \\
            \midrule
            Ask other researchers about their taste \\
            \midrule
            Read about history of research ("The Structure of Scientific Revolutions") \\
            \midrule
            De-risk your ideas: Proactive idea evaluation mitigates research risks (kill fast, learn fast) \\
            \multicolumn{1}{p{\linewidth}}{
            \begin{enumerate}
                \item Identify potential bottlenecks
                \item Prioritize and commit $X$ amt. of time to exploring them
                \item Decide if you should continue or pivot
            \end{enumerate}} \\
            \bottomrule
        \end{tabular}
    \end{center}
    \begin{itemize}
        \item \href{https://colah.github.io/notes/taste/}{Research Taste}
    \end{itemize}
\end{summary}

\begin{warning}
    \begin{itemize}
        \item Getting attached to one direction. 
        \item Lack of research knowledge / intimacy.
        \item Environment is not supportive of your interests.
    \end{itemize}
\end{warning}
\newpage

\subsection{Code / Experiments}
\begin{summary}
    \begin{center}
        \begin{tabular}{ll}
            \toprule
            \textbf{Tools} & \textbf{Links} \\
            \toprule
            Artifacts to create/track w/ experiments & \\
            \multicolumn{2}{p{\linewidth}}{
            \begin{itemize}
                \item Data, code (scripts / modules), models (weights, configurations), results, predictions, plots, meeting notes, papers, documentation, etc.
            \end{itemize}} \\
            \midrule
            Git (Version Control): Enables effective tracking and collaborative changes & \href{https://rogerdudler.github.io/git-guide/}{Git Guide} \\
            \multicolumn{2}{p{\linewidth}}{
            \begin{itemize}
                \item Tracking changes, collaboration, backup, revert.
                \item Add, commit -m "message", push, pull, merge, diff, revert, branch, checkout, log, status, etc.
                \item .gitignore: Ignore files, directories, or patterns
            \end{itemize}} \\
            \midrule
            GitHub (Collaborative Code Hosting): Facilitates sharing and collaboration on code & \href{https://github.com/git-guides}{GitHub} \\
            \multicolumn{2}{p{\linewidth}}{
            \begin{itemize}
                \item Collaboration, sharing, open science, project hosting.
            \end{itemize}} \\
            \midrule
            Cookiecutter (Project Template): Standardizes project structure & \href{https://cookiecutter.readthedocs.io/en/1.7.2/}{Cookiecutter} \\
            & \href{https://github.com/drivendataorg/cookiecutter-data-science}{Repo} \\
            \multicolumn{2}{p{\linewidth}}{
            \begin{itemize}
                \item Logical, flexible, and reasonably standardized project structure for doing and sharing data science work.
                \item File Structure
                \begin{itemize}
                    \item data/: 
                    \begin{itemize}
                        \item external/: Data from third party sources.
                        \item interim/: Intermediate data that has been transformed.
                        \item processed/: The final, canonical data sets for modeling.
                        \item raw/: The original, immutable data dump.
                    \end{itemize}
                    \item src/: Source code for use in this project.
                    \begin{itemize}
                        \item \_\_init\_\_.py: Makes src a Python module.
                        \item config.py: Configuration settings.
                        \item dataset.py: Code to load data.
                        \item features.py: Code to build features.
                        \item modeling/: Code to train models.
                        \begin{itemize}
                            \item \_\_init\_\_.py: Makes modeling a Python module.
                            \item predict.py: Code to make predictions.
                            \item train.py: Code to train models.
                        \end{itemize}
                        \item plots.py: Code to create plots.
                    \end{itemize}
                    \item docs/: Documentation for this project.
                    \item models/: Trained and serialized models, model predictions, or model summaries.
                    \item notebooks/: Jupyter notebooks.  
                    \item references/: Data dictionaries, manuals, and all other explanatory materials.
                    \item reports/: Generated analysis as HTML, PDF, LaTeX, etc.
                    \begin{itemize}
                        \item figures/: Generated graphics and figures to be used in reporting.
                    \end{itemize}
                    \item pyproject.toml: Project information and dependencies.
                    \item requirements.txt: The requirements file for reproducing the analysis environment.
                    \item setup.cfg: Configuration file for setting up the project.
                    \item LICENSE: 
                    \item Makefile: 
                    \item README.md: 
                \end{itemize}
            \end{itemize}} \\
            \bottomrule
        \end{tabular}
    \end{center}
\end{summary}
\newpage

\begin{summary}
    \begin{center}
        \begin{tabular}{ll}
            \toprule
            \textbf{Tools} & \textbf{Links} \\
            \toprule
            Cookiecutter (Project Template): Standardizes project structure & \href{https://cookiecutter-data-science.drivendata.org/opinions/}{Opinions} \\
            \multicolumn{2}{p{\linewidth}}{
            \begin{itemize}
                \item \textbf{Design Philosophy:} Prioritizes conventions and reasonable defaults to streamline project setup. Opinions:
                \begin{itemize}
                    \item Data analysis is a DAG:
                    \begin{enumerate}
                        \item Raw data
                        \item Compute features
                        \item Plot analysis on raw data 
                        \item Train model
                        \item Compute statistics on features
                    \end{enumerate}
                    \item Raw data is immutable (i.e. never change raw data)
                    \begin{itemize}
                        \item \textbf{Dos:}
                        \begin{itemize}
                            \item Pipeline code: Process raw data $\rightarrow$ final analysis.
                            \item Cache outputs: Serialize or cache intermediate steps. 
                            \item Reproducible results: Enable full reproduction from code and raw data only.
                        \end{itemize}
                        \item \textbf{Don'ts:}
                        \begin{itemize}
                            \item Never edit raw data: Avoid manual edits or format changes
                            \item Never overwrite raw data: Do not replace raw data with processed data.
                            \item Single raw data version: Maintain only one version of raw data.
                        \end{itemize}
                    \end{itemize}
                    \item Data should (mostly) not be kept in source control
                    \begin{itemize}
                        \item GitHub warns for files over 50MB and rejects files over 100MB.
                        \item Use s3, azcopy, gcloud, drive to store data (i.e. cloud services)
                        \item Use cloudpathlib to access cloud data in the same way as pathlib to access local data.
                    \end{itemize}
                    \item Notebooks are for exploration and communication, source files are for repetitions.
                    \item Refactor the good parts into source code (\href{https://cookiecutter-data-science.drivendata.org/opinions/\#notebooks-are-for-exploration-and-communication-source-files-are-for-repetition}{Refactor Example}).
                    \begin{itemize}
                        \item Don't write code to do the same task in multiple notebooks.
                    \end{itemize}
                    \item Keep your modelling organized \href{https://pytorch.org/tutorials/beginner/basics/saveloadrun_tutorial.html}{(PyTorch Example)}
                    \begin{itemize}
                        \item Predictions (csv), training log (csv), stats (txt), model config / hyperparameters (json)
                    \end{itemize}
                    \item Build from the environment up \href{https://github.com/mamba-org/mamba}{(Mamba)}
                    \begin{itemize}
                        \item Use mamba rather than conda for faster environment management.
                        \item Create a environment.yml file to manage dependencies.
                    \end{itemize}
                \end{itemize}
            \end{itemize}} \\
            \bottomrule
        \end{tabular}
    \end{center}
\end{summary}
\newpage

\subsection{Writing / Analyzing}
\begin{summary}
    \begin{center}
        \begin{tabular}{ll}
            \toprule
            \textbf{Exercise} & \textbf{Links} \\
            \toprule
            Writing Skeleton & \href{https://esajournals.onlinelibrary.wiley.com/doi/full/10.1002/bes2.1258}{Step-by-Step Guide to Undergraduate Writing} \\
            & \href{https://www.nature.com/articles/d41586-018-02404-4}{How to write a first-class paper (Nature)} \\
            & \href{https://www.ncbi.nlm.nih.gov/pmc/articles/PMC5037950/}{Preparing Manuscript: Scientific Writing for Publication} \\
            \multicolumn{2}{p{\linewidth}}{
            \begin{enumerate}
                \item Start w/ Figures (How would you want to telll the story?)
                \item Write the structure (Introduction, Methods, Experiments and Results, Discussion)
                \item 2-3 sentence pitch for your idea
                \item Bullet points inside of each section (What are you expecting to cover?)
                \item Fill in text, repeat.
            \end{enumerate}} \\
            \midrule
            Figures: Move quick, perfect later & \\
            \multicolumn{2}{p{\linewidth}}{
            \begin{itemize}
                \item Figure \#1: Tells problem in simple way (30s elevator pitch)
                \item Figure \#2-3: Conceptual or data centric
                \begin{itemize}
                    \item How are you solving the problem?
                    \item What does the data look like?
                \end{itemize}
                \item Figure \#4-8: Quantitative evidence
                \item \textbf{Recommendations:}
                \begin{itemize}
                    \item Napkin/whiteboard figures first
                    \item Make good enough version w/ code (svg, png) using matplotlib, seaborn, etc.
                    \item Finetune w/ InkScape, Illustrator, GIMP, etc
                \end{itemize}
                \item \textbf{Anatomy of a Figure Examples (L8):} 
                \begin{itemize}
                    \item Slide 44: Task, Slide 45: Model + EDA
                    \item Slide 46-47: Quantitative evidence, Slide 48: Different ways of telling same story (e.g. tables or plots)
                \end{itemize}
            \end{itemize}} \\
            \midrule
            Pick good, consistent colors & \href{https://colorbrewer2.org/#type=sequential&scheme=BuGn&n=3}{ColorBrewer}, \href{https://matplotlib.org/stable/gallery/color/colormap_reference.html}{Matplotlib Colormaps} \\
            & \href{http://seaborn.pydata.org/tutorial/aesthetics.html}{Seaborn Palettes}, \href{https://www.youtube.com/watch?v=ze08gwVPaXk}{NeurIPS 18 Visualization for ML tutorial} \\
            \multicolumn{2}{p{\linewidth}}{
            \begin{itemize}
                \item Be mindful of how colours can help tell a story, accessibility is also important.
            \end{itemize}}\\
            \midrule
            Pair-writing (Como Pair-Coding) & \href{https://en.wikipedia.org/wiki/Rubber_duck_debugging}{Rubber Duck Debugging}, \href{https://en.wikipedia.org/wiki/Pair_programming}{Pair Programming} \\
            & \href{http://sunnyday.mit.edu/16.355/williams.pdf}{Pair Writing}, \href{https://pds.blog.parliament.uk/2017/03/29/pair-writing/}{Pair Writing in Government} \\
            \multicolumn{2}{p{\linewidth}}{
            \begin{itemize}
                \item Working together $\rightarrow$ Help communicate thoughts adn put you in a diff. attitude.
            \end{itemize}}\\
            \midrule
            Communal writing (Social pressure $\rightarrow$ accountable) & \href{https://gsas.harvard.edu/academics/writing}{Harvard Writing Center} \\
            \multicolumn{2}{p{\linewidth}}{
            \begin{enumerate}
                \item Setup an objective, measurable goal. 
                \item Set a time for writing period, take breaks
                \item Share progress at the end of each session, share writing stuggles if needed.
                \item Reflect if there are some reasons why it is hard to write.
            \end{enumerate}}\\
            \midrule
            Writing: Focus on quick iterations & \\
            \multicolumn{2}{p{\linewidth}}{
            \begin{itemize}
                \item Google docs and Paperpile (Copy DOI, paste, click, done) $\overset{\text{Export w/ bibtex}}{\Longrightarrow}$ LaTex (Overleaf)
            \end{itemize}}\\
            \midrule
            Interactive Apps & \href{https://streamlit.io/}{Streamlit}, \href{https://www.gradio.app/}{Gradio} \\
            & \href{https://www.huggingface.co}{Hugging Face}, \href{https://www.hf.co/spaces}{Hugging Face Spaces} \\
            & \href{https://github.com/eliahuhorwitz/Academic-project-page-template}{Academic Project Page Template} \\
            \bottomrule
        \end{tabular}
    \end{center}
\end{summary}



\newpage

\section{Symmetries, Tabular Data, Sets}

\begin{enumerate}
    \item \textbf{Assume prior:} 
    \[
    p_\Theta(\theta) \text{ or } f_\Theta(\theta) \quad \text{(Bayesian)}
    \]
    Observations: \( \bar{X} = x \).

    \item \textbf{LMS Estimator:}
    \begin{align*}
        \hat{\theta} &= g(x) = \mathbb{E}[\Theta \mid \bar{X} = x] \\
        \text{or } \hat{\Theta} &= g(\bar{X}) = \mathbb{E}[\Theta \mid \bar{X}].
    \end{align*}

    \textbf{Note:}
    \begin{itemize}
        \item \textbf{MAP:} Use the most probable \( \theta \) given \( x \).
        \item \textbf{LMS:} Use the expected value (conditional on \( \bar{X} = x \)) of \( \Theta \), i.e., the "Conditional Expectation Estimator."
    \end{itemize}

    \item \textbf{Unbiasedness of LMS Estimator:}
    \begin{align*}
        \mathbb{E}[\hat{\Theta}] &= \mathbb{E}[\mathbb{E}[\Theta \mid \bar{X}]] = \mathbb{E}[\Theta], \\
        \implies \mathbb{E}[\hat{\Theta} - \Theta] &= 0.
    \end{align*}

    \item \textbf{LMS Estimator Minimizes Conditional MSE:}
    \[
    \mathbb{E}\big[(\Theta - \hat{\Theta})^2 \mid \bar{X} = x \big].
    \]
    \textbf{Proof:}
    \begin{enumerate}
        \item First, suppose no observations: \( \hat{\Theta} \) is a constant.
        \begin{align*}
            \hat{\Theta} &= \arg\min_c \mathbb{E}\big[(\Theta - c)^2\big], \\
            0 &= \frac{d}{dc} \big[ -2\mathbb{E}[\Theta] + 2c \big], \\
            c &= \mathbb{E}[\Theta].
        \end{align*}

        \item Alternate view:
        \begin{align*}
            \mathbb{E}\big[(\Theta - c)^2\big] &= \mathrm{Var}[\Theta] + (\mathbb{E}[\Theta] - c)^2.
        \end{align*}
        To minimize: Set bias \( \mathbb{E}[\Theta] - c \) to zero.

        \item Now, with observations \( \bar{X} = x \):
        \begin{align*}
            \mathbb{E}\big[(\Theta - g(x))^2 \mid \bar{X} = x\big] &= \mathrm{Var}[\Theta \mid \bar{X} = x] + (\mathbb{E}[\Theta \mid \bar{X} = x] - g(x))^2.
        \end{align*}
        To minimize: Set \( g(x) = \mathbb{E}[\Theta \mid \bar{X} = x] \).
    \end{enumerate}

    \item \textbf{Conclusion:}
    \[
    \hat{\Theta} = g(x) = \mathbb{E}[\Theta \mid \bar{X} = x].
    \]
\end{enumerate}

\begin{enumerate}
    \item \textbf{Example: Prior Coin Toss Problem}
    \begin{align*}
        \hat{\theta}_{\text{LMS}} &= \mathbb{E}[\Theta \mid X = k] \\
        &= \frac{k + \alpha}{n + \alpha + \beta}.
    \end{align*}

    \item \textbf{Example: Prior Voltage Problem}
    \begin{enumerate}
        \item \textbf{Setup:}
        \begin{itemize}
            \item Unknown voltage \( \Theta \).
            \item Prior: \( \Theta \sim \text{Uniform}[0, 1] \).
            \item Volt meter reading \( Y \) given \( \Theta \): \( Y \sim \text{Uniform}[0, \Theta] \).
            \item Independent measurements: \( Y_1, \dots, Y_n \) given \( \Theta \).
        \end{itemize}

        \item \textbf{Likelihood:}
        \begin{align*}
            f_{Y \mid \Theta}(\mathbf{y} \mid \theta) &= \prod_{i=1}^n f_{Y}(y_i \mid \theta) \\
            &= \frac{1}{\theta^n} \cdot 1(\theta \geq \max_i y_i).
        \end{align*}

        \item \textbf{Posterior:}
        \begin{align*}
            f_{\Theta \mid Y}(\theta \mid \mathbf{y}) &= \frac{\frac{1}{\theta^n} \cdot 1(\theta \geq \max_i y_i)}{f_Y(\mathbf{y})}.
        \end{align*}

        \item \textbf{Estimators:}
        \begin{itemize}
            \item Maximum Likelihood (ML): 
            \[
            \hat{\theta} = \max_{1 \leq i \leq n} y_i.
            \]

            \item LMS:
            \[
            \hat{\theta} = \mathbb{E}[\Theta \mid Y = y] = \int_{0}^\infty \theta f_{\Theta \mid Y}(\theta \mid y) d\theta.
            \]
        \end{itemize}

        \item \textbf{Derivation for LMS:}
        \begin{enumerate}
            \item Compute \( f_Y(y) \) for \( n = 1 \):
            \begin{align*}
                f_Y(y) &= \int_y^1 \frac{1}{\theta} d\theta \\
                &= \ln(\theta) \Big|_y^1 \\
                &= -\ln(y).
            \end{align*}

            \item Compute \( \hat{\theta} \) for \( n = 1 \):
            \begin{align*}
                \hat{\theta} &= \int_y^1 \frac{\theta \cdot 1(\theta \geq y)}{-\ln(y)} d\theta \\
                &= \frac{1}{-\ln(y)} \int_y^1 \theta d\theta \\
                &= \frac{1}{-\ln(y)} \cdot \frac{y^2 - 1}{2}.
            \end{align*}
        \end{enumerate}

        \item \textbf{Graphical Interpretation:}
        \begin{itemize}
            \item \( f_{\Theta \mid Y}(\theta \mid y) = \frac{1}{-\ln(y)} \cdot 1(\theta \geq y) \cdot 1(0 \leq \theta \leq 1) \).
            \item The MAP estimator corresponds to the most probable \( \theta \).
            \item The LMS estimator minimizes the mean squared error, representing the "safest" choice.
        \end{itemize}
    \end{enumerate}
\end{enumerate}
\newpage

\section{CNN}
\begin{summary}
    In a \textbf{POMDPs}, we assume that: 
    \begin{itemize}
        \item environment modelled using state space, $\mathcal{S}$
        \item single agent
        \item $S_t$ = state after transition $t$
        \item $A_t$ = action inducing transition $t$
        \item stochastic state transitions with memoryless property:
        \[
        S_T \perp S_0, A_1, \dots, A_{T-1}, S_{T-2} \mid S_{T-1}, A_T
        \]
        \item $R_t$ = reward for transition $t$, i.e., $(S_{T-1}, A_T, S_T)$
        \item $O_t$ = observation of $S_t$
    \end{itemize}
    \vspace{1em}

    \begin{center}
        \begin{tabular}{ll}
            \toprule
            \textbf{Name} & \textbf{Function:} \\
            \midrule
            Initial state distribution & $p_0(s) := \mathbb{P}[S_0 = s]$ \\
            \midrule
            Transition distribution & $p(s'|s,a) := \mathbb{P}[S_t = s' | A_t = a, S_{t-1} = s]$ \\
            \midrule
            Reward function & $r(s,a,s') :=$ reward for transition $(s, a, s')$ \\
            \midrule
            Policy for choosing actions & $\pi_t(a | o_0, \dots, o_t) := \mathbb{P}[A_t = a | O_0 = o_0, \dots, O_t = o_t]$ \\
            \midrule
            Measurement model & $m(o | s) := \mathbb{P}[O_t = o | S_t = s]$ \\
            \bottomrule
        \end{tabular}
    \end{center}
\end{summary}
\newpage

\subsection{Bayesian Network}
\begin{notes}
    $S_0, O_0, A_1, R_1, S_1, O_1, A_2, R_2, S_2, O_2, \dots$ form a Bayesian network:
    \customFigure[0.5]{../Images/L10_0.png}{}
\end{notes}

\begin{example}
    
\end{example}
\newpage

\section{RNN}
\begin{summary}
    In a \textbf{Multi-Agent problem}, we assume that:
    \begin{itemize}
        \item Set of states for environment is $\mathcal{S}$
        \item $P$ agents within environment. 
        \item For each state $s \in \mathcal{S}$: 
        \begin{itemize}
            \item possible actions for agent $i$ is $\mathcal{A}_i(s)$
            \item set of action profiles is $\mathcal{A}(s) = \prod_{i=1}^P \mathcal{A}_i(s)$
        \end{itemize}
        \item possible state-action pairs are $\mathcal{T} = \{(s,a) \text{ s.t. } s \in \mathcal{S}, a \in \mathcal{A}(s)\}$
        \item environment in some origin state, $s_0$ 
        \item environment destroyed after $N$ transitions 
        \item agent $j$ wants to find policy $\pi_j (a_j \mid s)$ so that $\mathbb{E}[r_j(p)]$ is maximized
        \item agents act independently given the environment's state: $\pi (a \mid s) = \prod_{j\in [P]} \pi_j (a_j \mid s)$
    \end{itemize}

    \begin{center}
        \begin{tabular}{ll}
            \toprule
            \textbf{Name} & \textbf{Function:} \\
            \midrule
            State transition given state-action pair defined by $\text{tr}: \mathcal{T} \to \mathcal{S}$ & $\text{tr}(s,a) = \text{state transition from $s$ under $a$}$ \\ 
            \midrule
            Reward to each agent, $i$ defined by $r_i$: $\mathcal{Q} \times \mathcal{S} \rightarrow \mathbb{R}_+$ & $r_i(s,a,\text{tr}(s,a)) = \text{rwd to agent $i$ for $(s,a,tr(s,a))$}$ \\
            \midrule
            State evolution of environment after $N$ transitions & $p = \langle (s_0,a^{(1)},s_{1}),\ldots,(s_{N-1},a^{(N)},s_{N})\rangle$ \\ 
            \multicolumn{2}{p{\linewidth}}{
            \begin{itemize}
                \item Given sequence of actions: $p.a = \langle a^{(1)},\ldots,a^{(n)}\rangle$
                \item $s_N = \tau (s_{n-1},a^{(n)})$
            \end{itemize}} \\
            \midrule
            reward to agent $i$ & $r_i(p) = \sum_{n=1}^N r_i (s_{n-1},a^{(n)}, s_n)$ \\
            \midrule
            expected-reward (value) of playing $a$ from $s$ for agent $j$ & $q_j (s,a) = r_j(s,a,\tau(s,a)) +$ \\
            & $\sum_{a' \in \mathcal{A}(\tau(s,a))} \pi(a' \mid \tau(s,a)) q_j(\tau(s,a),a')$ \\
            \multicolumn{2}{p{\linewidth}}{
                \begin{itemize}
                    \item $\mathcal{A}(s) = \emptyset$ if $s \in \mathcal{G}$
                \end{itemize}} \\
            \bottomrule            
        \end{tabular}
    \end{center}
\end{summary}

\subsection{Policy Equilibria}
\begin{notes}
    \begin{itemize}
        \item \textbf{No Regret:} $\pi$ is no-regret if $\pi_j$ maximizes $q_j$ when $\pi_{-j}$ is fixed. 
        \item If all agents play perfectly, then we expect
        \begin{equation*}
            \pi (a \mid s) = \begin{cases}
                1 & \text{if } a= a^*(s) \\
                0 & \text{otherwise}
            \end{cases}
        \end{equation*}
        \begin{itemize}
            \item $a_j^*(s) = \arg \max_{a_j \in \mathcal{A}_j(s)} q_j(s,a_j,a_{-j}^*)$ is the best action for agent $j$ given the other agents' policies.
        \end{itemize}
    \end{itemize}
\end{notes}

\begin{warning}
    No regret if it got the highest reward given the other players' action. 
\end{warning}
\newpage

\subsection{Single Action Games}
\begin{summary}
    In a \textbf{Single Action Game}, we assume that:
    \begin{itemize}
        \item $N=1$ (one-shot game)
        \item Initial state is $s_0 \in \mathcal{S}$
        \item Agent $j$ wants to find policy, $\pi_i (a_i \mid s_0)$ so $\mathbb{E}[r_i(p)]$ is maximized
    \end{itemize}
\end{summary}

\subsection{Actions (Deterministic)}
\begin{summary} Allow each agent to choose action deterministically.
    \begin{center}
        \begin{tabular}{ll}
            \toprule
            \textbf{Name} & \textbf{Function:} \\
            \midrule
            Action $j$ for agent $i$ & $[0 \cdots 0 \; 1 \; 0 \cdots 0]^T$ \\
            \midrule
            \multicolumn{2}{p{\linewidth}}{
                \begin{itemize}
                    \item One-hot vector of $M_i$ components, $\mathbf{e}_{i,j}$
                \end{itemize}} \\
            \midrule
            Agent $i$'s set of possible actions & $\mathcal{A}_i = \left\{a_i \in \{0,1\}^{M_i} \mid \sum_{j \in [M_i]} a_{i,j} = 1\right\}$ \\
            \multicolumn{2}{p{\linewidth}}{
                \begin{itemize}
                    \item Agent $i$'s chosen action with $a_i \in \mathcal{A}_i$
                \end{itemize}} \\
            \midrule
            Action profile is a tuple of actions & $a=(a_1,\ldots,a_P)$ \\
            \multicolumn{2}{p{\linewidth}}{
                \begin{itemize}
                    \item \textbf{Notational Convenience:} $a_{-i} = (a_1,\ldots,a_{i-1},a_{i+1},\ldots,a_P)$ so that $a=(a_i,a_{-i})$. 
                \end{itemize}} \\
            \midrule 
            Optimal action profile & $a^+ \text{ s.t. } \forall a \exists i \text{ s.t. } r_i(a) < r_i(a^+)$ \\
            \multicolumn{2}{p{\linewidth}}{
                \begin{itemize}
                    \item An action profile w.r.t. which any other action profile leaves at least one player worse off.
                \end{itemize}} \\
            \midrule 
            Set of optimal action profiles & $\text{aOpt} = \{a^+ \mid \forall a \exists i: r_i(a) < r_i(a^+)\}$ \\
            \midrule
            Best-action mapping, $\text{ba}_i$: $\mathcal{A}_{-i} \to \mathcal{A}_i$ & $\text{ba}_i (a_{-i}) = \arg \max_{a_i \in \mathcal{A}_i} r_i (a_i,a_{-i})$ \\
            & $ = \{a_i \in \mathcal{A}_i \mid r_i (a_i,a_{-i}) = \max_{a_i' \in \mathcal{A}_i} r_i (a_i',a_{-i})\}$ \\
            \midrule 
            Agent $i$ will \textbf{not regret} playing $a_i^*$ when others play $a_{-i}^*$ if & $r_i (a_i^*,a_{-i}^*) \geq r_i (a_i,a_{-i}^*) \; \forall a_i \in \mathcal{A}_i$ \\
            & or $a_i^* \in \text{ba}_i (a_{-i}^*)$ \\
            \midrule
            Action equilibria is any action, $a^*$ in which no agent regrets & $a_i^* \in \text{ba}_i (a_{-i}^*) \; \forall i \in [P]$ \\
            \midrule
            Existence of action equilibria & May not always exist, i.e., it may be that $\text{aEq} = \emptyset$ \\
            \bottomrule            
        \end{tabular}
    \end{center}
\end{summary}
\newpage

\subsection{Strategies (Probabilistic)}
\begin{summary} Allow each agent to choose action based on a distribution/strategy. 
    \begin{center}
        \begin{tabular}{ll}
            \toprule
            \textbf{Name} & \textbf{Function:} \\
            \midrule
            Stategy for agent $i$ & $[0.05 \cdots 0.2 \; 0.7 \; 0 \cdots 0.05]^T$ \\
            \multicolumn{2}{p{\linewidth}}{
                \begin{itemize}
                    \item Vector of $M_i$ components, that are non-negative and sum to 1
                \end{itemize}} \\
            \midrule
            Agent $i$'s set of possible strategies & $\Delta_i = \Delta^{M_i} = \left\{x_i \in [0,1]^{M_i}, \sum_{j \in [M_i]} x_{i,j} = 1 \right\}$ \\
            \multicolumn{2}{p{\linewidth}}{
                \begin{itemize}
                    \item Agent $i$'s chosen strategy with $x_i \in \Delta_i$
                \end{itemize}} \\
            \midrule
            Expected reward & $\bar{r}_i(x_1,\ldots,x_P) = \mathbb{E}[r_i(a)] = \sum_{a_i \in \mathcal{A}_i} \pi(a) r_i(a)$ \\
            \midrule
            Stategy profile is a tuple of strategies & $x=(x_1,\ldots,x_P)$ \\
            \multicolumn{2}{p{\linewidth}}{
                \begin{itemize}
                    \item \textbf{Notational Convenience:} $x_{-i} = (x_1,\ldots,x_{i-1},x_{i+1},\ldots,x_P)$ so that $x=(x_i,x_{-i})$. 
                \end{itemize}} \\
            \midrule
            No-regret strategies & $\bar{r}_i(x_i^*,x_{-i}^*) \geq \bar{r}_i(x_i,x_{-i}^*) \; \forall x_i \in \Delta_i$ \\
            & $x_i^* \in \text{bs}_i (x_{-i}^*)$ \\
            \multicolumn{2}{p{\linewidth}}{
                \begin{itemize}
                    \item Agent $i$ will not regret using $x_i^*$ when others use $x_{-i}^*$.
                \end{itemize}} \\
            \midrule
            Best strategy mapping $\text{bs}_i$: $\Delta_{-i} \to \Delta_i$ & $\text{bs}_i (x_{-i}) = \arg \max_{x_i \in \Delta_i} \bar{r}_i (x_i,x_{-i})$ \\
            & $ = \{x_i \in \Delta_i \mid \bar{r}_i (x_i,x_{-i}) = \max_{x_i' \in \Delta_i} \bar{r}_i (x_i',x_{-i})\}$ \\
            \midrule
            Strategy equilibria is any strategy, $x^*$ in which no agent regrets & $x_i^* \in \text{bs}_i (x_{-i}^*) \; \forall i \in [P]$ \\
            \midrule
            Joint best-strategy mapping $\text{bs}: \Delta \to \Delta$ & $\text{bs}(x) = (\text{bs}_1(x_{-1}),\ldots,\text{bs}_P(x_{-P}))$ \\
            \midrule
            Existence of strategy equilibria & Any strategy equilibrium, $x^*$ is a fixed pt. \\
            & of $x^* = \text{bs}(x^*)$ \\
            \multicolumn{2}{p{\linewidth}}{
                \begin{itemize}
                    \item Fixed pt. always exists. 
                \end{itemize}} \\
            \bottomrule            
        \end{tabular}
    \end{center}
\end{summary}

\subsubsection{Simplifying Games}
\begin{notes} May be able to reduce $M_i$ by eliminating useless actions/strategies: 
    \begin{itemize}
        \item \textbf{Equivalent Stategies:} $x_i^{(1)} \equiv x_i^{(2)}$
        \begin{equation*}
            \bar{r}_i(x_i^{(1)},x_{-i}) = \bar{r}_i(x_i^{(2)},x_{-i}) \; \forall x_{-i} 
        \end{equation*}
        \item \textbf{Dominated Strategies:} $x_i$ 
        \begin{equation*}
            \exists x_i' \text{ s.t. } \bar{r}_i(x_i',x_{-i}) \leq \bar{r}_i(x_i,x_{-i}) \; \forall x_{-i}
        \end{equation*}
    \end{itemize}
    Can remove dominated and equilvalent strategies w/o changing the game. 
\end{notes}
\newpage

\subsection{Examples}
\subsubsection{Finding Action Equilibria}
\begin{process} To find action equilibria:
    \begin{enumerate}
        \item For each $i$, compute $\text{ba}_i (a_{-i})$ for all $a_{-i}$
        \item Define $\text{bap}_i$ so that $\text{bap}_i = \{(a_i',a_{-i}), \; \forall a_i' \in \text{ba}_i (a_{-i}), \; \forall a_{-i} \in \mathcal{A}_{-i}\}$
        \item Action equilibria are then $\text{aEq} = \bigcap_{i\in [P]} \text{bap}_i$.
    \end{enumerate}
\end{process}

\begin{process}
    \begin{enumerate}
        \item Fix strategy (i.e. prob. 1) for other player, then find best move for current player by getting max reward. 
        \item See if there's intersection between best responses.
    \end{enumerate}
\end{process}

\begin{warning}
    \begin{itemize}
        \item Action equilibrium is a pure equilibrium (i.e. prob. 1 or 0)
        \begin{itemize}
            \item aEq doesn't mean that the actual outcome is fight,fight. 
            \item aEq doesn't mean it is socially optimal (i.e. equilibria may not be optimal). 
        \end{itemize}
        \item Mixed equilibrium is a probabilistic equilibrium (i.e. prob. $p$)
        \begin{itemize}
            \item Every action equilibrium is a mixed equilibrium, but not vice versa.
        \end{itemize}
    \end{itemize}
\end{warning}
\newpage

\begin{example}
    \begin{enumerate}
        \item \textbf{Given:} Consider a 2-player single-action game, in which each player has 2 actions. Let $a_1$ and $a_2$ be their chosen actions and $r_1(a_1, a_2), r_2(a_1, a_2)$ be the resulting rewards. 
            
            \[
            \begin{array}{c c|c|c}
            a_1 & a_2 & r_1(a_1, a_2) & r_2(a_1, a_2) \\
            \hline
            1 & 1 & 2 & 2 \\
            1 & 2 & 4 & 1 \\
            2 & 1 & 1 & 4 \\
            2 & 2 & 3 & 3 \\
            \end{array}
            \]
        \item \textbf{Solution:} 
        \begin{itemize}
            \item Convert:
            \[
            \begin{array}{c|c|c}
            & a_2 = 1 & a_2 = 2 \\
            \hline
            a_1 = 1 & (2,2) & (4,1) \\
            a_1 = 2 & (1,4) & (3,3) \\
            \end{array}
            \]
            \item \textbf{Fix} $a_2$, \textbf{choose }$a_1$:
            \begin{align*}
            \text{If } a_2 = 1 &\Rightarrow a_1 = 1 \Rightarrow (1,1) \\
            \text{If } a_2 = 2 &\Rightarrow a_1 = 1 \Rightarrow (1,2) \\
            &\Rightarrow \text{bap}_1 = \{(1,1), (1,2)\}
            \end{align*}

            \item \textbf{Fix} $a_1$, \textbf{choose }$a_2$:
            \begin{align*}
            \text{If } a_1 = 1 &\Rightarrow a_2 = 1 \Rightarrow (1,1) \\
            \text{If } a_1 = 2 &\Rightarrow a_2 = 1 \Rightarrow (2,1) \\
            &\Rightarrow \text{bap}_2 = \{(1,1), (2,2)\}
            \end{align*}

            \noindent
            \textbf{Intersection:}
            \[
                \text{aEq} = \text{bap}_1 \cap \text{bap}_2 = \{(1,1)\}
            \]
        \end{itemize}
        \item Therefore, the action equilibria is $(1,1)$.
    \end{enumerate}

\end{example}
\newpage

\begin{example}
    \begin{enumerate}
        \item \textbf{Given:} Suppose lion and cavemen both want meat. Each must decide whether to fight for the food or share it. 
        \customFigure[0.5]{../Images/L11_3.png}{}
        \vspace{-1em}
        \item \textbf{Problem:} Find the action equilibria 
        \item \textbf{Solution:}
        \begin{enumerate}
            \item \textbf{Best Action Profiles:}
            \begin{itemize}
                \item Cavemen: $\text{bap}_{\text{cavemen}} = \{(\text{Fight, Fight}), (\text{Fight, Share})\}$. Cavemen fights no matter what. 
                \begin{itemize}
                    \item If lion fights, then cavemen fights to get maximum reward in this scenario of $+1$. 
                    \item If lion shares, then caveman fights to get maximum reward in this scenario of $+4$.
                \end{itemize}
                \item Lion: $\text{bap}_{\text{lion}} = \{(\text{Fight, Fight}), (\text{Share, Fight})\}$. Lion fights no matter what.
                \begin{itemize}
                    \item If caveman fights, then lion fights to get maximum reward in this scenario of $+1$.
                    \item If caveman shares, then lion fights to get maximum reward in this scenario of $+4$.
                \end{itemize}
            \end{itemize}
            \item \textbf{Best Action Equilibria:} Intersection of the best action profiles.
            \begin{itemize}
                \item $\text{aEq} = \text{bap}_{\text{cavemen}} \cap \text{bap}_{\text{lion}} = \{(\text{Fight, Fight})\}$
            \end{itemize}
        \end{enumerate}
    \end{enumerate}
\end{example}

\begin{example}
    \begin{enumerate}
        \item \textbf{Given:} Suppose lion and cavemen both want meat. Each must decide whether to fight for the food or share it. 
        \customFigure[0.5]{../Images/L11_4.png}{}
        \vspace{-1em}
        \item \textbf{Problem:} Find the action equilibria 
        \item \textbf{Solution:}
        \begin{enumerate}
            \item \textbf{Best Action Profiles:}
            \begin{itemize}
                \item Cavemen: $\text{bap}_{\text{cavemen}} = \{(\text{Fight, Fight}), (\text{Share, Share})\}$. Cavemen fights no matter what. 
                \begin{itemize}
                    \item If lion fights, then cavemen fights to get maximum reward in this scenario of $+3$. 
                    \item If lion shares, then caveman shares to get maximum reward in this scenario of $+1$.
                \end{itemize}
                \item Lion: $\text{bap}_{\text{lion}} = \{(\text{Fight, Share}), (\text{Share, Fight})\}$. Lion fights no matter what.
                \begin{itemize}
                    \item If caveman fights, then lion shares to get maximum reward in this scenario of $+3$.
                    \item If caveman shares, then lion fights to get maximum reward in this scenario of $+1$.
                \end{itemize}
            \end{itemize}
            \item \textbf{Best Action Equilibria:} Intersection of the best action profiles.
            \begin{itemize}
                \item $\text{aEq} = \text{bap}_{\text{cavemen}} \cap \text{bap}_{\text{lion}} = \emptyset$
            \end{itemize}
        \end{enumerate}
    \end{enumerate}
\end{example}
\newpage

\subsubsection{Optimal Action Profiles}
\begin{process}
    \begin{enumerate}
        \item Switching action profiles will leave at least one player worse off.
        \begin{itemize}
            \item That is, no unilateral deviation by any player should result in both players being better off.
        \end{itemize}
    \end{enumerate}
\end{process} 

\begin{example}
    \begin{enumerate}
        \item \textbf{Given:}
         \[
        \begin{array}{c c|c|c}
        a_1 & a_2 & r_1(a_1, a_2) & r_2(a_1, a_2) \\
        \hline
        1 & 1 & 2 & 2 \\
        1 & 2 & 4 & 1 \\
        2 & 1 & 1 & 4 \\
        2 & 2 & 3 & 3 \\
        \end{array}
        \]
        
        \item \textbf{Solution:}
        \begin{itemize}
            \item \textbf{At action profile }$(1,1)$:
            \[
            \text{Switch to }(2,2): \text{ both players are better off} \Rightarrow \text{not optimal}
            \]
        
            \item \textbf{At action profile }$(1,2)$:
            \[
            \text{Switch to }(1,1), (2,1), (2,2): \text{ Player 1 will be worse off for all switches} \Rightarrow \text{optimal}
            \]
        
            \item \textbf{At action profile }$(2,1)$:
            \[
            \text{Switch to }(1,1), (1,2), (2,2): \text{ Player 2 will be worse off for all switches} \Rightarrow \text{optimal}
            \]
        
            \item \textbf{At action profile }$(2,2)$:
            \begin{align*}
                \text{Switch to }(1,1): &\text{ Both Player 1 and Player 2 will be worse off} \\
                \text{Switch to }(2,1): &\text{ Player 1 will be worse off} \\
                \text{Switch to }(1,2): &\text{ Player 2 will be worse off} \Rightarrow \text{optimal} 
            \end{align*}
            \item Therefore, $(1,2)$, $(2,1)$, and $(2,2)$ are optimal action profiles.
        \end{itemize}
    \end{enumerate}
\end{example}
\newpage

\begin{example}
    \begin{enumerate}
        \item \textbf{Given:} Suppose lion and cavemen both want meat. Each must decide whether to fight for the food or share it. 
        \customFigure[0.5]{../Images/L11_3.png}{}
        \item \textbf{Problem:} Find the optimal action profiles: 
        \item \textbf{Solution:}
        \begin{enumerate}
            \item $\text{aOpt} = \{(\text{Share, Share}), (\text{Fight, Share}), (\text{Share, Fight})\}$
            \begin{itemize}
                \item \textbf{(Fight, Fight) $\rightarrow$ (1, 1):} \\
                Both players could be better off by choosing (Share, Share), which gives (2, 2). \\
                Since there exists another outcome where no one is worse off and at least one player is better off, this is not an optimal action profile. \\
                $\Rightarrow$ \textbf{Not optimal}.
                
                \item \textbf{(Fight, Share) $\rightarrow$ (4, 0):} \\
                The caveman gets the highest possible payoff (4), but the lion gets 0. \\
                Any other outcome that gives the lion more than 0 will reduce the caveman’s payoff. \\
                So, every other option makes at least one player worse off. \\
                $\Rightarrow$ \textbf{Optimal}.
                
                \item \textbf{(Share, Fight) $\rightarrow$ (0, 4):} \\
                The lion gets the highest possible payoff (4), but the caveman gets 0. \\
                Any other outcome that gives the caveman more than 0 will reduce the lion’s payoff. \\
                So, every other option makes at least one player worse off. \\
                $\Rightarrow$ \textbf{Optimal}.
                
                \item \textbf{(Share, Share) $\rightarrow$ (2, 2):} \\
                Both players receive a fair and equal payoff. No other outcome makes both players better off at the same time. \\
                So, changing this would hurt at least one of them. \\
                $\Rightarrow$ \textbf{Optimal}.
            \end{itemize}                                 
        \end{enumerate}
    \end{enumerate}
\end{example}
\newpage

\subsubsection{Finding/Convergence Strategy Equilibria}
\begin{process} \textbf{Finding}
    \begin{enumerate}
        \item For each $i$, compute $\text{bs}_i (x_{-i})$ for all $x_{-i}$
        \item Define $\text{bsp}_i$ so that $\text{bsp}_i = \{(x_i',x_{-i}), \; \forall x_i' \in \text{bs}_i (x_{-i}), \; \forall x_{-i} \in \Delta_{-i}\}$
        \item Strategy equilibria are then $\text{sEq} = \bigcap_{i\in [P]} \text{bsp}_i$.
    \end{enumerate}
    \begin{itemize}
        \item Requires each agent, $j$, to know $\bar{r}_1,\ldots,\bar{r}_P$
    \end{itemize}
\end{process}

\begin{warning}
    \begin{itemize}
        \item If on the line, then don't move for the player, then it's optimal, so don't move it. 
    \end{itemize}
\end{warning}
\newpage

\begin{example}
    \begin{enumerate}
        \item 4-1, pg. 29-35
    \end{enumerate}
\end{example}

\begin{example}
    \begin{enumerate}
        \item 4-1, pg. 46-47
    \end{enumerate}
\end{example}

\begin{example}
    \begin{enumerate}
        \item \textbf{Given/Problem:} Find all equilibria of the following one-shot game or state that none exist.
        \vspace{1em}
            \begin{center}
            \begin{tabular}{ccc}
            \toprule
            & \textbf{B1 (y)} & \textbf{B2 (1-y)} \\
            \midrule
            \textbf{A1 (x)} & (5, 3) & (1, 0) \\
            \textbf{A2 (1-x)} & (0, 1) & (2, 4) \\
            \bottomrule
            \end{tabular}
            \end{center}
        \vspace{1em}
        \begin{itemize}
            \item (\#,\#) is the payoff to P1 and P2 respectively for a given action profile.
        \end{itemize}
        \item \textbf{Solution:}
        \begin{enumerate}
            \item \textbf{Define Probabilities:}
            \begin{itemize}
                \item Let $y$ be the probability that B1 plays action B1 so $1-y$ is the probability that B1 plays action B2.
                \item Let $x$ be the probability that A1 plays action A1 so $1-x$ is the probability that A1 plays action A2.
            \end{itemize}
            \item \textbf{Expected Rewards:} 
            \begin{itemize}
                \item P1: 
                \begin{align*}
                    E[x] = 5xy + 1x(1-y) + 0(1-x)y + 2(1-x)(1-y) &= 5xy + x - xy + 2 - 2x - 2y + 2xy \\
                    &= 5xy - xy + 2xy + x - 2x - 2y + 2 \\
                    &= 6xy - x - 2y + 2 \quad \text{simplify} \\
                    &= \underbrace{(6y - 1)}_{c}x + 2 - 2y \quad \text{linear in $x$} 
                \end{align*}
                \item P2:
                \begin{align*}
                    E[y] = 3xy + 0x(1-y) + 1(1-x)y + 4(1-x)(1-y) &= 3xy + 0 + y - xy + 4 - 4x - 4y + 4xy \\
                    &= 3xy - xy + 4xy + y - 4x - 4y + 4 \\
                    &= 6xy - 4x - 3y + 4 \quad \text{simplify} \\
                    &= \underbrace{(6x - 3)}_{c}y + 4 - 4x \quad \text{linear in $y$}
                \end{align*}
                \item \textbf{Note:} $E[x]$ is linear in $x$ and $E[y]$ is linear in $y$.
            \end{itemize}
            \item \textbf{Constrained Argmax Expected Rewards w.r.t $x \in [0,1]$ (since P1):} If it was cost, then minimize. Also don't care about constant term in $y$ since we are derivating w.r.t $x$.
            \begin{itemize}
                \item P1: 
                \begin{equation*}
                    \text{bs}_{\text{A}}(x) = \begin{cases}
                        1 & \text{if } y > \frac{1}{6} \text{ i.e. } c > 0 \text{ since positive want maximum positive}\\
                        & \\
                        [0,1] & \text{if } y=\frac{1}{6} \text{ i.e. }c = 0 \text{ doesn't matter since 0} \\
                        & \\
                        0 & \text{if } y < \frac{1}{6} \text{ i.e. } c < 0 \text{ since negative want maximum negative}
                    \end{cases}
                \end{equation*}
                \item P2:
                \begin{equation*}
                    \text{bs}_{\text{B}}(y) = \begin{cases}
                        1 & \text{if } x > \frac{3}{6} \text{ i.e. } c > 0 \text{ since positive want maximum positive}\\
                        & \\
                        [0,1] & \text{if } x=\frac{3}{6} \text{ i.e. }c = 0 \text{ doesn't matter since 0} \\
                        & \\
                        0 & \text{if } x < \frac{3}{6} \text{ i.e. } c < 0 \text{ since negative want maximum negative}
                    \end{cases}
                \end{equation*}
            \end{itemize}
            \item \textbf{Finding all equilibrium:} Lines on the graph represents where your reward is maximized. 
            \customFigure[0.5]{../Images/L11_9.png}{}
            \begin{itemize}
                \item \textbf{Case 1:} $x=0$ and $y=0$
                \begin{itemize}
                    \item $P(\text{P1 chooses A1}) = 0$ 
                    \item $P(\text{P1 chooses A2}) = 1$
                    \item $P(\text{P2 chooses B1}) = 0$
                    \item $P(\text{P2 chooses B2}) = 1$
                \end{itemize}
                \item \textbf{Case 2:} $x=1/2$ and $y=1/6$
                \begin{itemize}
                    \item $P(\text{P1 chooses A1}) = 1/2$
                    \item $P(\text{P1 chooses A2}) = 1/2$
                    \item $P(\text{P2 chooses B1}) = 1/6$
                    \item $P(\text{P2 chooses B2}) = 5/6$
                \end{itemize}
                \item \textbf{Case 3:} $x=1$ and $y=1$
                \begin{itemize}
                    \item $P(\text{P1 chooses A1}) = 1$
                    \item $P(\text{P1 chooses A2}) = 0$
                    \item $P(\text{P2 chooses B1}) = 1$
                    \item $P(\text{P2 chooses B2}) = 0$
                \end{itemize}
            \end{itemize}
            \item \textbf{Unstable Equilibrium:} P1 moves left and right b/c $x$ is associated with $x$-axis. P2 moves up and down b/c $y$ is associated with $y$-axis.
            \customFigure[0.5]{../Images/L11_10.png}{}
            \begin{itemize}
                \item Stability means that in a radius disc around the equilibrium, if you move a little bit, you will still be in the equilibrium (have to check all relevant quadrants)
                \begin{itemize}
                    \item If one quadrant is unstable, then don't need to check the other quadrants as the equilibrium point is unstable. 
                    \item Simulatenous (both players move at the same time) and sequential (one player moves first and the other player moves second) 
                \end{itemize}
                \item \textbf{Case 1:} $x=0$ and $y=0$ is stable
                \begin{itemize}
                    \item Q1: Always converges to $(0,0)$ since P1 moves left to red and P2 moves down to turquoise.
                \end{itemize}
                \item \textbf{Case 2:} $x=1/2$ and $y=1/6$ is unstable
                \begin{itemize}
                    \item Q1 (Top Left): P1 moves right to red and P2 moves up to turquoise $\implies (1,1)$
                    \item Q2 (Top Right): P1 moves right to red and P2 moves up to turquoise $\implies (1,1)$
                    \item Q3 (Bottom Left): P1 moves left to red and P2 moves down to turquoise $\implies (0,0)$
                    \item Q4 (Bottom Right): P1 moves left to red and P2 moves down to turquoise $\implies (0,0)$
                \end{itemize}
                \item \textbf{Case 3:} $x=1$ and $y=1$ is stable
                \begin{itemize}
                    \item Q1: Always converges to $(1,1)$ since P1 moves left to red and P2 moves down to turquoise.
                \end{itemize}
            \end{itemize}
        \end{enumerate}
    \end{enumerate}
\end{example}
\newpage

\subsubsection{Simplifying Games}
\begin{example}
    \begin{enumerate}
        \item \textbf{Given:} 
        \customFigure[0.5]{../Images/L11_8.png}{}
        \item \textbf{Problem:} Simplify the game 
        \item \textbf{Solution:}
        \begin{enumerate}
            \item 'Leave' action for Cavemen is dominated so it will never choose this action b/c it can get more reward by choosing Fight or Share. 
            \item As aresult, we can remove it from the game and have a 2 action, 2 player game, rather than a 2-3 action, 2 player game.
        \end{enumerate}
    \end{enumerate}
\end{example}


\newpage

\section{GNN}
\subsection{Graphs as General Structured Data}
\begin{definition}
    \begin{itemize}
        \item Nodes (Vertices): Attributes 
        \begin{itemize}
            \item e.g. Node identity, \# of neighbors, etc.
            \customFigure[0.25]{../Images/L12_0.png}{}
        \end{itemize}
        \item Edges (Link): Attributes and Directions
        \begin{itemize}
            \item e.g. edge identity, edge weight, etc.
            \customFigure[0.25]{../Images/L12_1.png}{}
        \end{itemize}
        \item Global (master node): Attributes
        \begin{itemize}
            \item e.g. \# of nodes, longest path, etc.
            \customFigure[0.25]{../Images/L12_2.png}{}
        \end{itemize}
    \end{itemize}
\end{definition}

\subsubsection{What Info Can Store in a Graph?}
\begin{notes} Add vector (embeddings) to each of the graph components.
    \begin{itemize}
        \item Nodes (Vertices): Embeddings
        \item Edges (Link): Attributes and Embeddings
        \begin{itemize}
            \item Undirected and directed edges
        \end{itemize}
        \item Global (master node): Embeddings
    \end{itemize}
    \customFigure[0.25]{../Images/L12_3.png}{}
\end{notes}
\newpage

\subsection{Examples of Graphs}
\begin{example}
    \begin{enumerate}
        \item Maps
        \item Code, algorithms, mathematical formulas, neural networks 
        \item Images: Pixels connected to their neighboring pixels ("Locality" inductive bias)
        \customFigure[0.5]{../Images/L12_4.png}{}
        \item Test: Chain of words or characters ("Sequential" inductive bias)
        \customFigure[0.5]{../Images/L12_5.png}{}
        \item Molecules: Atoms are nodes, covalent bonds are edges
        \customFigure[0.5]{../Images/L12_6.png}{}
        \item Social Networks: People can be nodes, and their interactions are edges. 
        \customFigure[0.5]{../Images/L12_7.png}{}
    \end{enumerate}
\end{example}
\newpage

\subsection{Graph Problems}
\begin{notes}
    \begin{itemize}
        \item Global tasks:
        \customFigure[0.5]{../Images/L12_8.png}{}
        \item Node tasks:
        \customFigure[0.5]{../Images/L12_9.png}{}
        \item Edge tasks:
        \customFigure[0.5]{../Images/L12_10.png}{}
    \end{itemize}
\end{notes}
\newpage

\subsubsection{Representing Graphs Numerically, Mathematically, Programmatically}
\begin{notes} Represent the connectivity of the graph.
    \begin{itemize}
        \item Numerically 
        \begin{itemize}
            \item Adjacency Matrix: $A \in \mathbb{R}^{N \times N}$
            \customFigure[0.5]{../Images/L12_11.png}{}
            \item Adjacency List: $A \in \mathbb{R}^{N \times E}$
            \customFigure[0.5]{../Images/L12_12.png}{}
        \end{itemize}
        \item Mathematically: Structured tensors $G = (X,E,A,U)$
        \item Programmatically: Graph data structure
        \customFigure[0.5]{../Images/L12_13.png}{}
    \end{itemize}
\end{notes}
\newpage

\subsection{GNN}
\begin{summary}
    \begin{center}
        \begin{tabular}{ll}
            \toprule
            \textbf{Concept} & \textbf{Description} \\
            \midrule
            \textbf{Prediction Methods} & \\
            Node to node & Take nodes and pass in our favorite classifer \\
            \multicolumn{2}{p{\linewidth}}{\begin{center}
                \customFigure[0.5]{../Images/L12_15.png}{}
                \vspace{-4em}
            \end{center}} \\
            Edge to node & Passing info from one attribute to another requires a pooling op. \\
            \multicolumn{2}{p{\linewidth}}{\begin{center}
                \customFigure[0.5]{../Images/L12_16.png}{}
                \vspace{-4em}
            \end{center}} \\
            Node to edges & Passing info from one attribute to another requires a pooling op. \\
            \multicolumn{2}{p{\linewidth}}{\begin{center}
                \customFigure[0.5]{../Images/L12_18.png}{}
                \vspace{-4em}
            \end{center}} \\
            \midrule
        \end{tabular}
    \end{center}
\end{summary}
\newpage

\begin{summary}
    \begin{center}
        \begin{tabular}{ll}
            \toprule
            \textbf{Concept} & \textbf{Description} \\
            \midrule
            \textbf{Pooling from Edges to Nodes} & Route info b/w diff. parts and aggregate them. \\
            \multicolumn{2}{p{\linewidth}}{\begin{center}
                \customFigure[0.5]{../Images/L12_17.png}{}
                \vspace{-4em}
            \end{center}} \\
            \midrule
            \textbf{Prediction Pipeline} & $\textbf{Graph} \rightarrow \text{Transform} \rightarrow \text{Predict on attribute of interest}$ \\
            \multicolumn{2}{p{\linewidth}}{\begin{center}
                \customFigure[0.5]{../Images/L12_19.png}{}
                \vspace{-4em}
            \end{center}} \\
        \end{tabular}
    \end{center}
\end{summary}
\newpage

\begin{summary}
    \begin{center}
        \begin{tabular}{ll}
            \toprule
            \textbf{Concept} & \textbf{Description} \\
            \midrule
            \textbf{Message passing} & Pool information, aggregate it, transform it, and update it \\
            \multicolumn{2}{p{\linewidth}}{\begin{center}
                \customFigure[0.5]{../Images/L12_20.png}{}
                \vspace{-4em}
            \end{center}} \\
            Propogation pattern of message passing & After a few layers we get increasingly more complex patterns of info. \\
            \multicolumn{2}{p{\linewidth}}{\begin{center}
                \customFigure[0.5]{../Images/L12_21.png}{}
                \vspace{-4em}
            \end{center}} \\
            \midrule
            \textbf{Conditioning Info} & Many ways of adding context to a specific part of a graph \\
            \multicolumn{2}{p{\linewidth}}{\begin{center}
                \customFigure[0.5]{../Images/L12_23.png}{}
                \vspace{-4em}
            \end{center}} \\
            \midrule
        \end{tabular}
    \end{center}
\end{summary}
\newpage

\subsection{Graph Topics}
\begin{summary}
    \begin{center}
        \begin{tabular}{ll}
            \toprule
            \textbf{Concept} & \textbf{Description} \\
            \midrule
            Learning subgraph representations & Learning a function that works on portions of a graph \\
            \multicolumn{2}{p{\linewidth}}{\begin{center}
                \customFigure[0.5]{../Images/L12_25.png}{}
                \vspace{-4em}
            \end{center}} \\
            \midrule
            Batching in graphs & Very context dependent, no general solution \\
            \multicolumn{2}{p{\linewidth}}{\begin{center}
                \customFigure[0.5]{../Images/L12_26.png}{}
                \vspace{-4em}
            \end{center}} \\
            \midrule
        \end{tabular}
    \end{center}
\end{summary}
\newpage

\begin{summary}
    \begin{center}
        \begin{tabular}{ll}
            \toprule
            \textbf{Concept} & \textbf{Description} \\
            \midrule
            Heterogeneous graphs & Different types of nodes and edges \\
            \multicolumn{2}{p{\linewidth}}{\begin{center}
                \customFigure[0.5]{../Images/L12_27.png}{}
                \vspace{-4em}
            \end{center}} \\
            \midrule
            Interpretability & Many ways to extract information from graphs \\
            \multicolumn{2}{p{\linewidth}}{\begin{center}
                \customFigure[0.5]{../Images/L12_28.png}{}
                \vspace{-4em}
            \end{center}} \\
            \midrule
            Attention & Transformers can be viewed as a GNN on a fully connected graph \\
            \multicolumn{2}{p{\linewidth}}{\begin{center}
                \customFigure[0.5]{../Images/L12_29.png}{}
                \vspace{-4em}
            \end{center}} \\
            \midrule
        \end{tabular}
    \end{center}
\end{summary}

\subsection{Examples}

\subsubsection{Message passing / GraphNets on a small graph}
\begin{example}
    \customFigure[0.5]{../Images/L12_24.png}{}
\end{example}


\end{document}