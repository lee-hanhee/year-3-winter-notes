\documentclass{article}
\usepackage{style}
\title{ECE324 Cheatsheet}
\author{Hanhee Lee}
\lhead{ECE324}
\rhead{Hanhee Lee}

\begin{document}

\section{Problem 1: Description of the Problem, Subscribe a ML/NN Based Solution}
\begin{process}

\end{process}

\begin{definition}
    
\end{definition}

\begin{example}
    
\end{example}

\section{Problem 2: Prescribe a strategy to optimize the NN.}
\begin{process}

\end{process}

\begin{definition}
    
\end{definition}

\begin{example}
    
\end{example}

\section{Problem 3: Explain step by step inference for basic algorithms (MLP, CNN, GNN, attention mechanism) in terms of numpy or basic tensor operations.}
\begin{process}

\end{process}

\begin{definition}
    
\end{definition}

\begin{example}
    
\end{example}

\section{Problem 4: Be able to explain why such a solution might work or fail.} 
\begin{process}

\end{process}

\begin{definition}
    
\end{definition}

\begin{example}
    
\end{example}
\newpage

\begin{center}
    \section*{L2: A tour of learning algorithms, representations, and neural networks}
\end{center}
\begin{summary}
    \begin{itemize}
        \item Not responsible for proofs, but know when to use each algorithm.
    \end{itemize}
\end{summary}
\subsection{Setup}
\begin{definition} In a search problem, it is assumed that: 
    \begin{itemize}
        \item There is only one agent (us).
        \item For each state, $s \in S$, we have a discrete set of actions, $\mathcal{A}(s)$.
        \item The transition resulting from a move, $(s, a)$, is deterministic; the resulting state is $tr(s, a)$.
        \item $cst(s, a, tr(s, a))$ is our cost for the transition, $(s, a, tr(s, a))$.
        \item We want to realize a path that minimizes our cost.
    \end{itemize}
    
    A search problem may have no solutions, in which case, we define the solution as \texttt{NULL}.
\end{definition}

\subsection{Search Graphs}
\begin{definition}
    In a search graph (a graph representing a search problem):
    \begin{itemize}
        \item $S$ is defined by the vertices.
        \item $\mathcal{G}$ is a subset of the vertices.
        \item $s^{(0)}$ is some vertex.
        \item $tr(\cdot, \cdot)$ and $\mathcal{T}$ are defined by the edges.
        \item $cst(\cdot, \cdot, \cdot)$ is defined by the edge weights.
    \end{itemize}
\end{definition}

\subsection{Path Trees}
\begin{definition}
    A search algorithm explores a tree of possible paths. 
    \begin{itemize}
        \item In such a tree, each node represents the path from the root to itself.
        \begin{itemize}
            \item The node may also include other info (such as the path's origiin, cost, etc).
        \end{itemize}
    \end{itemize}
\end{definition}

\subsection{Search Algorithms}
\begin{definition}
    All search algorithms follow the template below:

\begin{lstlisting}
$\mathcal{O} \gets \{(\langle \rangle, 0)\}$ (*\hfill $\triangleright$ initialize a set of open nodes*) 
SEARCH($\mathcal{O}$)
\end{lstlisting}
\begin{itemize}
    \item $\langle \rangle$ is the empty path, and $0$ is the cost of the empty path.
\end{itemize}

\begin{lstlisting}
procedure SEARCH($\mathcal{O}$)
    if $\mathcal{O} = \emptyset$ then
        return NULL  (*\hfill $\triangleright$ the search algorithm failed to find a path to a goal*)
    $n \gets \textsc{Remove}(\mathcal{O})$ (*\hfill $\triangleright$ "explore" a node $n$*)
    if $\textsc{dst}(n) \in \mathcal{G}$ then
        return $n$ (*\hfill $\triangleright$ the search algorithm found a path to a goal*)
    for $n' \in \textsc{chl}(n)$ do
        $\mathcal{O} \gets \mathcal{O} \cup \{n'\}$ (*\hfill $\triangleright$ "expand" $n$ and "export" its children*)
    SEARCH($\mathcal{O}$)
\end{lstlisting}
\begin{itemize}
    \item Explore: Remove a node from the open set.
    \item Exapnd: Generate the children of the node.
    \item Export: Add the children to the open set.
\end{itemize}

\end{definition}

\begin{warning}
    The key difference is in the order that \textsc{Remove}($\cdot$) removes nodes.
\end{warning}

\subsubsection{Characteristics of a Search Algorithm}
\begin{definition}
    We want to choose \texttt{REMOVE(·)} so that the algorithm exhibits the following characteristics:

    \begin{center}
        \begin{tabular}{|p{3cm}|p{9cm}|}
        \hline
        \textbf{Characteristic} & \textbf{Description} \\ \hline
        Halting & Terminates after finitely many nodes explored \\ \hline
        Sound & Returned (possibly NULL) solution is correct \\ \hline
        Complete & Halting and sound when a non-NULL solution exists \\ \hline
        Optimal & Returns an optimal solution when multiple exist \\ \hline
        Time Efficient & Minimizes the nodes \textbf{explored}/expanded/exported \\ \hline
        Space Efficient & Minimizes the nodes simultaneously open \\ \hline
        \end{tabular}
    \end{center}
    \vspace{1em}   
    \begin{itemize}
        \item Will be using explored for time efficiency.
    \end{itemize} 
    \vspace{1em}

    The characteristics of the algorithm also depend on several properties of the path tree over which it searches. These properties include:
    \begin{itemize}
        \item Branching factor: $b$ ($b < \infty$), the maximum number of children a node can have.
        \item Depth: $d$, the length of the longest path.
        \item Length of the shortest solution: $l^*$
        \item Cost of the cheapest solution: $c^*$
        \item Cost of the cheapest edge: $\epsilon$ 
    \end{itemize}

    We want to choose \texttt{REMOVE($\cdot$)} so that the algorithm exhibits the aforementioned characteristics for as many path trees as possible.

\end{definition}

\subsubsection{Breadth First Search (BFS)}
\begin{definition}
    Explores the least-recently expanded open node first.
    \begin{center}
        \begin{tabular}{|p{3cm}|p{3cm}|}
        \hline
        \textbf{Property} & \textbf{Description} \\ \hline
        Halting & $d < \infty$ \newline non-NULL \\ \hline
        Sound & always \\ \hline
        Complete & always \\ \hline
        Optimal & constant cst \\ \hline
        Time & $b^{l^*}$ \\ \hline
        Space & $b^{l^* + 1}$ \\ \hline
        \end{tabular}
    \end{center}
\end{definition}

\subsubsection{Depth First Search (DFS)}
\begin{definition}
    Explores the most-recently expanded open node first.
    \begin{center}
        \begin{tabular}{|p{3cm}|p{3cm}|}
        \hline
        \textbf{Property} & \textbf{Description} \\ \hline
        Halting & $d < \infty$ \\ \hline
        Sound & always \\ \hline
        Complete & $d < \infty$ \\ \hline
        Optimal & never \\ \hline
        Time & $b^d$ \\ \hline
        Space & $bd$ \\ \hline
        \end{tabular}
    \end{center}    
\end{definition}

\subsubsection{Iterative Deepening DFS (IDDFS)}
\begin{definition}
    Same as DFS but with iterative deepening.
    \begin{center}
        \begin{tabular}{|p{3cm}|p{3cm}|}
        \hline
        \textbf{Property} & \textbf{Description} \\ \hline
        Halting & always \\ \hline
        Sound & always \\ \hline
        Complete & always \\ \hline
        Optimal & constant cst \\ \hline
        Time & $b^{l^*}$ \\ \hline
        Space & $bl^*$ \\ \hline
        \end{tabular}
    \end{center}    
\end{definition}

\subsubsection{Cheapest-First Search (CFS)}
\begin{definition}
    Explores the cheapest open node first.
    \begin{center}
        \begin{tabular}{|p{3cm}|p{3cm}|}
        \hline
        \textbf{Property} & \textbf{Description} \\ \hline
        Halting & $d < \infty$ \newline non-NULL \\ \hline
        Sound & yes \\ \hline
        Complete & $\epsilon > 0$ \\ \hline
        Optimal & $\epsilon > 0$ \\ \hline
        Time & $b^{c^*/\epsilon}$ \\ \hline
        Space & $b^{c^*/\epsilon + 1}$ \\ \hline
        \end{tabular}
    \end{center}    
\end{definition}

\subsection{Modifications to Search Algorithms}
\subsubsection{Depth-Limiting}
\begin{definition}
    Depth limit of $d_{\text{max}}$ to any search algorithm by modifying \texttt{SEARCH($\cdot$)} as follows:
\begin{lstlisting}
procedure SEARCHDL($\mathcal{O}$, $d_{\text{max}}$):
    if $\mathcal{O} = \emptyset$ then
        return NULL (*\hfill $\triangleright$ the search algorithm failed to find a path to a goal*)
    $n \leftarrow \text{REMOVE}(\mathcal{O})$ (*\hfill $\triangleright$ "explore" a node, $n$*)
    if dst($n$) $\in \mathcal{G}$ then
        return $n$ (*\hfill $\triangleright$ the search algorithm found a path to a goal*)
    for $n' \in \text{chl}(n)$ do (*\hfill $\triangleright$ "expand" $n$ and "export" its children*)
        if len($n'$) $\leq d_{\text{max}}$ then (*\hfill $\triangleright$ unless the child is too long*)
            $\mathcal{O} \leftarrow \mathcal{O} \cup \{n'\}$
    SEARCHDL($\mathcal{O}$, $d_{\text{max}}$)
\end{lstlisting}

\end{definition}

\subsubsection{Iterative Deepening}
\begin{definition}
    Iteratively increase the depth-limit, $d_{\max}$, to any search algorithm w/ depth-limiting, by placing \texttt{SEARCHDL($\cdot$)} in a wrapper, \texttt{SEARCHID($\cdot$)}:
\begin{lstlisting}
procedure SEARCHID():
    $n \leftarrow \text{NULL}$
    $d_{\text{max}} \leftarrow 0$
    (*$\triangleright$ while a solution has not been found, reset the open set, run the search algorithm, then increase the depth-limit*)
    while $n = \text{NULL}$ do
        $\mathcal{O} \leftarrow \{(\langle \rangle, 0)\}$
        $n \leftarrow \text{SEARCHDL}(\mathcal{O}, d_{\text{max}})$
        $d_{\text{max}} \leftarrow d_{\text{max}} + 1$
    return $n$
\end{lstlisting}
    
\end{definition}

\begin{warning}
    Increasing $d_{\text{max}}$ can be done in different ways.
\end{warning}

\subsubsection{Cost-Limiting}
\begin{definition}
    Cost limit of $c_{\text{max}}$ to any search algorithm by modifying \texttt{SEARCH($\cdot$)} as follows:

\begin{lstlisting}
procedure SEARCHCL($\mathcal{O}$, $c_{\text{max}}$):
    if $\mathcal{O} = \emptyset$ then
        return NULL (*\hfill $\triangleright$ the search algorithm failed to find a path to a goal*)
    $n \leftarrow \text{REMOVE}(\mathcal{O})$ (*\hfill $\triangleright$ "explore" a node, $n$*)
    if dst($n$) $\in \mathcal{G}$ then
        return $n$ (*\hfill $\triangleright$ the search algorithm found a path to a goal*)
    for $n' \in \text{chl}(n)$ do (*\hfill $\triangleright$ "expand" $n$ and "export" its children*)
        if cst($n'$) $\leq c_{\text{max}}$ then (*\hfill $\triangleright$ unless the child is too expensive*)
            $\mathcal{O} \leftarrow \mathcal{O} \cup \{n'\}$
    SEARCHCL($\mathcal{O}$, $c_{\text{max}}$)
\end{lstlisting}

\end{definition}

\subsubsection{Iterative-Inflating}
\begin{definition}
    Iteratively increase the cost limit, $c_{\text{max}}$, to any search algorithm with cost-limiting, by placing \texttt{SEARCHCL($\cdot$)} in a wrapper, \texttt{SEARCHII($\cdot$)}:

\begin{lstlisting}
procedure SEARCHII():
    $n \leftarrow \text{NULL}$
    $c_{\text{max}} \leftarrow 0$
    (*$\triangleright$ while a solution has not been found, reset the open set, run the search algorithm, then increase the cost-limit*)
    while $n = \text{NULL}$ do
        $\mathcal{O} \leftarrow \{(\langle \rangle, 0)\}$
        $n \leftarrow \text{SEARCHCL}(\mathcal{O}, c_{\text{max}})$
        $c_{\text{max}} \leftarrow c_{\text{max}} + \epsilon$
    return $n$
\end{lstlisting}

\end{definition}

\begin{warning}
    Increasing $c_{\text{max}}$ can be done in different ways.
\end{warning}

\subsubsection{Intra-Path Cycle Checking}
\begin{definition}
    Do not expand a path if it is cyclic. Modify \texttt{SEARCH($\cdot$)} as follows:

\begin{lstlisting}
procedure SEARCH($\mathcal{O}$):
    if $\mathcal{O} = \emptyset$ then
        return NULL
    $n \leftarrow \text{REMOVE}(\mathcal{O})$
    if dst($n$) $\in \mathcal{G}$ then
        return $n$
    for $n' \in \text{chl}(n)$ do (*\hfill $\triangleright$ "expand" $n$ and "export" its children*)
        if not CYCLIC($n'$) then (*\hfill $\triangleright$ unless the child is cyclic*)
            $\mathcal{O} \leftarrow \mathcal{O} \cup \{n'\}$
    SEARCH($\mathcal{O}$)
\end{lstlisting}
\begin{itemize}
    \item Optimately of an algorithm is preserved provided $\epsilon>0$.
\end{itemize}

\end{definition}

\subsubsection{Inter-Path Cycle Checking}
\begin{definition}
    We modify \texttt{SEARCH($\cdot$)} as follows:

\begin{lstlisting}
procedure SEARCH($\mathcal{O}$, $\mathcal{C}$):
    if $\mathcal{O} = \emptyset$ then
        return NULL
    $n \leftarrow \text{REMOVE}(\mathcal{O})$
    $\mathcal{C} \leftarrow \mathcal{C} \cup \{n\}$ (*\hfill $\triangleright$ add $n$ to the closed set*)
    if dst($n$) $\in \mathcal{G}$ then
        return $n$
    for $n' \in \text{chl}(n)$ do (*\hfill $\triangleright$ "expand" $n$ and "export" its children*)
        if $n' \notin \mathcal{C}$ then (*\hfill $\triangleright$ unless the child's destination is closed*)
            $\mathcal{O} \leftarrow \mathcal{O} \cup \{n'\}$
    SEARCH($\mathcal{O}$, $\mathcal{C}$)
\end{lstlisting}

and then call the algorithm as follows:

\begin{lstlisting}[mathescape=true, escapeinside={(*}{*)}, numbers=left, frame=single]
$\mathcal{O} \leftarrow \{(\langle \rangle, 0)\}$
$\mathcal{C} \leftarrow \emptyset$ (*\hfill $\triangleright$ initialize a set of closed vertices*)
SEARCH($\mathcal{O}$, $\mathcal{C}$)
\end{lstlisting}

\end{definition}

\subsection{Informed Search Algorithms}
\subsubsection{Estimated Cost}
\begin{definition}
    $\text{ecst}(\cdot)$, to estimate the total cost to a goal given a path, $p$, based on the following:
    \begin{itemize}
        \item Cost of path $p$: $\text{cst}(p)$
        \item Estimate of the extra cost needed to get to a goal from $\text{dst}(p)$: $\text{hur} : S \to \mathbb{R}_+$
        \begin{itemize}
            \item $\text{hur}(s)$ estimates the cost to get to $\mathcal{G}$ from $s$ and $\text{hur}(p)$ means $\text{hur}(\text{dst}(p))$.
        \end{itemize}
    \end{itemize}
\end{definition}

\begin{example}
    Some common choices for $\text{ecst}(\cdot)$ include:
    \begin{enumerate}
        \item $\text{ecst}(p) = \text{hur}(p)$; called nearest-first search (NFS)
        \item $\text{ecst}(p) = \text{cst}(p) + \text{hur}(p)$; called A$^*$ (A-star)
    \end{enumerate}
\end{example}

\subsection{Characteristics of an Informed Search Algorithm}
\begin{definition}
    \begin{enumerate}
        \item Heuristic: $\text{hur}(\cdot)$
        \item Cost estimation: $\text{ecst}(\cdot)$
    \end{enumerate}
\end{definition}
\subsubsection{Heuristics}


\subsubsection{Heuristic-First Search (HFS)}

\subsubsection{A-Star Search (A*)}

\subsubsection{Iterative Inflating A-Star Search (IIA*)}

\subsubsection{Designing Heuristics via Problem Relaxation}

\subsubsection{Combining Heuristics}

\subsection{Anytime Search Algorithms}

\subsection{Formulating a Search Problem}







\newpage

\begin{center}
    \section*{L3: Multilayer Perceptrons (MLPs)}
\end{center}
\subsection{2 RVs}
\begin{notes}
    RVs are neither random nor a variable. 
    \begin{equation*}
        \underline{Z} = (X,Y)
    \end{equation*}
    \customFigure[0.5]{../Images/L3_0.png}{Mapping of RVs}
\end{notes}

\subsection{Joint PMF/PDF}
\begin{definition}
    \begin{equation}
    P_{X,Y}(x, y) = P[X = x, Y = y]
    \end{equation}
    
    \begin{equation}
    f_{X,Y}(x, y) = \frac{\partial^2}{\partial x \partial y} F_{X,Y}(x, y)
    \end{equation}
    
    \begin{equation}
    P[(X, Y) \in A] = \int \int_{(x, y) \in A} f_{X,Y}(x, y) \, dx \, dy
    \end{equation}
\end{definition}

\begin{example} Jointly Gaussian RVs $X$ and $Y$ with ($\mu_1, \mu_2, \sigma_1^2, \sigma_2^2, \rho$)
    \[
    f_{X,Y}(x, y) = \frac{1}{2\pi \sigma_1 \sigma_2 \sqrt{1-\rho^2}} 
    \exp \left\{ 
    -\frac{1}{2(1-\rho^2)} 
    \left[ 
    \left(\frac{x-\mu_1}{\sigma_1}\right)^2 
    - 2\rho \left(\frac{x-\mu_1}{\sigma_1}\right) \left(\frac{y-\mu_2}{\sigma_2}\right) 
    + \left(\frac{y-\mu_2}{\sigma_2}\right)^2 
    \right] 
    \right\}
    \]
\end{example}

\subsection{Expectations}
\begin{definition}
    \[
    E[g(X, Y)] = \int_{-\infty}^{\infty} \int_{-\infty}^{\infty} g(x, y) f_{X,Y}(x, y) \, dx \, dy
    \]
\end{definition}

\begin{notes}
    \begin{itemize}
        \item $g(X,Y)$ is also an RV, but inside the integral or sum, you use $x$ and $y$ as dummy variables to vary through the values of the RVs.
    \end{itemize}
\end{notes}

\subsubsection{Correlation}
\begin{definition}
    \begin{equation}
        E[XY]
    \end{equation}
\end{definition}

\subsubsection{Covariance}
\begin{definition}
    \begin{equation}
        \text{Cov}[X, Y] = E[(X - \mu_X)(Y - \mu_Y)] = E[XY] - \mu_X \mu_Y = E[XY] - E[X]E[Y]
    \end{equation}
\end{definition}

\begin{notes}
    \begin{itemize}
        \item Mean shifted to 0.
    \end{itemize}
\end{notes}

\subsubsection{Correlation Coefficient}
\begin{definition}
    \begin{equation}
        \rho_{X,Y} = E \left[ \left(\frac{X - \mu_X}{\sigma_X} \right) \left( \frac{Y - \mu_Y}{\sigma_Y} \right) \right] = \frac{\text{Cov}[X, Y]}{\sigma_X \sigma_Y}
    \end{equation}
    \begin{itemize}
        \item $|\rho_{X,Y}| \leq 1$
    \end{itemize}
\end{definition}

\begin{notes}
    \begin{itemize}
        \item Mean shifted to 0 and normalized by the standard deviation.
    \end{itemize}
\end{notes}

\subsection{Marginal PMF/PDF}
\begin{definition}
    \begin{equation}
    P_X(x) = \sum_{j=1}^{\infty} P_{X,Y}(x, y_j), \quad P_Y(y) = \sum_{j=1}^{\infty} P_{X,Y}(x_j, y)
    \end{equation}
    
    \begin{equation}
    f_X(x) = \int_{-\infty}^{\infty} f_{X,Y}(x, y) \, dy, \quad f_Y(y) = \int_{-\infty}^{\infty} f_{X,Y}(x, y) \, dx
    \end{equation}
\end{definition}

\begin{notes}
    \begin{itemize}
        \item Total probability theorem is being used here.
    \end{itemize}
\end{notes}

\begin{example} Jointly Gaussian $X$ and $Y$:
    \begin{align*}
        f_X(x) &= \int_{-\infty}^{\infty} f_{X,Y}(x, y) \, dy \\
               &= \dots \quad (\text{completing the square}) \\
               &= \frac{1}{\sqrt{2\pi} \sigma_1} e^{-\frac{(x-\mu_1)^2}{2\sigma_1^2}}, \quad \text{marginally Gaussian}
    \end{align*}
    \begin{itemize}
        \item Gaussian RVs has a property that the PDF of a single variable is equal to the marginal Gaussian of two variables.
    \end{itemize}
\end{example}

\subsection{Conditional PMF/PDF}
\begin{definition}
    \begin{equation}
    P_{X|Y}(x|y) \triangleq P[X = x | Y = y] = \frac{P_{X,Y}(x, y)}{P_Y(y)}
    \end{equation}
    
    \begin{equation}
    f_{X|Y}(x|y) \triangleq \frac{f_{X,Y}(x, y)}{f_Y(y)}
    \end{equation}
\end{definition}

\subsection{Bayes' Rule}
\begin{definition}
    \begin{equation}
    P_{Y|X}(x|y) = \frac{P_{X,Y}(x, y)}{P_X(x)} = \frac{P_{X|Y}(x|y) P_Y(y)}{\sum_{j=1}^\infty P_{X,Y}(x, y_j) P_Y (y_j)}
    \end{equation}
    
    \begin{equation}
    f_{Y|X}(y|x) = \frac{f_{X,Y}(x, y)}{f_X(x)} = \frac{f_{X|Y}(x|y) f_Y(y)}{\int_{-\infty}^\infty f_{X|Y}(x|y') f_Y(y') \, dy'}
    \end{equation}  
\end{definition}

\subsection{Independent vs. Uncorrelated vs. Orthogonal}
\begin{definition} 
    \begin{enumerate}
        \item Independent:
        \begin{equation}
        f_{X|Y}(x|y) = f_X(x) \; \forall y
        \Leftrightarrow 
        f_{X,Y}(x, y) = f_X(x) f_Y(y) 
        \end{equation}
        \item Uncorrelated:
        \begin{equation}
        \text{Cov}[X, Y] = 0 \quad \Leftrightarrow \quad \rho_{X,Y} = 0
        \end{equation}
        \item Orthogonal:
        \begin{equation}
        E[XY] = 0
        \end{equation}
    \end{enumerate}
\end{definition}

\begin{theorem}
    If independent, then uncorrelated.
\end{theorem}

\begin{derivation}
    \begin{align*}
    \text{Independent} & \implies E[XY] = \int_{-\infty}^{\infty} \int_{-\infty}^{\infty} x y f_{X,Y}(x, y) \, dx \, dy \\
    &= \int_{-\infty}^{\infty} \int_{-\infty}^{\infty} x y f_X(x) f_Y(y) \, dx \, dy \\
    &= \left( \int_{-\infty}^{\infty} x f_X(x) \, dx \right) \left( \int_{-\infty}^{\infty} y f_Y(y) \, dy \right) \\
    &\implies E[XY] = E[X] E[Y] \\
    &\implies \text{Cov}[X, Y] = 0, \quad \text{uncorrelated} \\
    &\not\Leftarrow \text{in general.}
    \end{align*}
\end{derivation}

\begin{example} Jointly Gaussian RVs $X$ and $Y$: If uncorrelated, i.e. $\rho_{X,Y} = 0$, then $X$ and $Y$ are independent.
    \begin{align*}
    f_{X,Y}(x, y) &= \frac{1}{2\pi \sigma_1 \sigma_2} 
    \exp \left\{ 
    -\frac{1}{2} 
    \left[ 
    \left(\frac{x-\mu_1}{\sigma_1}\right)^2 
    + 
    \left(\frac{y-\mu_2}{\sigma_2}\right)^2 
    \right] 
    \right\} \\
    &= \frac{1}{\sqrt{2\pi} \sigma_1} e^{-\frac{(x-\mu_1)^2}{2\sigma_1^2}} 
    \cdot 
    \frac{1}{\sqrt{2\pi} \sigma_2} e^{-\frac{(y-\mu_2)^2}{2\sigma_2^2}} \\
    &= f_X(x) f_Y(y) \quad \text{independent}
    \end{align*}
\end{example}

\subsection{Conditional Expectation}
\begin{definition}
    \begin{equation}
        E[Y] = E[E[Y|X]]
    \end{equation}
    \begin{equation}
        E[h(Y)] = E[E[h(Y)|X]]
    \end{equation}
\end{definition}

\begin{notes}
    \begin{itemize}
        \item $E[E[Y|X]]$ is w.r.t. $X$.
        \item $E[Y|X]$ is w.r.t. $Y$.
    \end{itemize}
\end{notes}

\begin{derivation}
    \begin{align*}
    E[Y] &= \int_{-\infty}^\infty \int_{-\infty}^\infty y f_{X,Y}(x, y) \, dx \, dy \\
            &= \int_{-\infty}^\infty \int_{-\infty}^\infty y f_{Y|X}(y|x) f_X(x) \, dx \, dy \\
            &= \int_{-\infty}^\infty \left( \int_{-\infty}^\infty y f_{Y|X}(y|x) \, dy \right) f_X(x) \, dx \\
            &= \int_{-\infty}^\infty E[Y|X=x] f_X(x) \, dx \quad \text{(using the total probability theorem)} \\
            &= \int_{-\infty}^\infty g(x) f_X(x) \, dx \\
            &= E[g(X)] \\ 
            &= E[E[Y|X]].
    \end{align*}
\end{derivation}

\begin{example}
    \begin{enumerate}
        \item \textbf{Given:} An unknown voltage. \( X \sim \text{Uniform}(0,1) \). Measurement from a (bad) voltmeter: \( Y \sim \text{Uniform}(0, X) \).
    
        \begin{align*}
            f_X(x) &= 
            \begin{cases} 
                1, & 0 < x < 1 \\ 
                0, & \text{otherwise}
            \end{cases} \\
            f_{Y|X}(y|x) &= 
            \begin{cases} 
                \frac{1}{x}, & 0 < y < x \\ 
                0, & \text{otherwise}
            \end{cases}
        \end{align*}
        \begin{itemize}
            \item \textbf{Note:} Area under PDF is 1.
        \end{itemize}

        \customFigure[0.25]{../Images/L3_1.png}{Uniform Distribution of $X$}
        \customFigure[0.25]{../Images/L3_2.png}{Uniform Distribution of $Y$}

    
        \item \textbf{Expected Value (Average Reading of Bad Voltmeter):}
        \begin{align*}
            E[Y] &= E[E[Y|X]] \\
                 &= E\left[\frac{X}{2}\right] \quad \text{Since in the middle of 0 and x}\\
                 &= \frac{1}{2} \cdot E[X] \\ 
                 &= \frac{1}{2} \cdot \frac{1}{2} = \frac{1}{4} \quad \text{Since $E[X]$ (i.e. mean) is 0.5}
        \end{align*}
    
        \item \textbf{The Long Way:}
        \begin{align*}
            f_Y(y) &= \int_{-\infty}^{\infty} f_{Y|X}(y|x) f_X(x) \, dx \\
                   &= \int_{y}^1 f_{Y|X}(y|x) f_X(x) \, dx \\
                   &= \int_{y}^1 \frac{1}{x} \cdot 1 \, dx \\
                   &= -\ln y. \\
            E[Y] &= \int_{0}^1 y \cdot (-\ln y) \, dy = \dots = \frac{1}{4}
        \end{align*}
    
        \item \textbf{Question:} Suppose \( Y = \frac{1}{8} \). What is "best" given \( X \)? This will be the quesiton for the rest of the course.
    \end{enumerate}    
\end{example}
\newpage

\begin{center}
    \section*{L4: Neural Network Engineering}
\end{center}
\section{Learning Problems}
\begin{definition}
    In a learning problem, we assume that there is some (unknown) relationship, 
    \begin{equation*}
        f: \mathcal{X} \rightarrow \mathcal{Y}
    \end{equation*}
    s.t. $x \mapsto_f y$
    \vspace{1em}

    Find $h: \mathcal{X} \rightarrow \mathcal{Y}$ (hypothesis) s.t. $h \approx f$, given some data about $f$: 

    \begin{itemize}
        \item $\text{in}(\mathcal{D}) = \{x \text{ s.t. } (x,y) \in \mathcal{D}\}$
        \item $\text{out}(\mathcal{D}) = \{y \text{ s.t. } (x,y) \in \mathcal{D}\}$
    \end{itemize}
\end{definition}

\subsection{Classification vs. Regression Problems}
\begin{definition}
    \begin{itemize}
        \item \textbf{Classification Problems:} $\mathcal{X} \subseteq \mathbb{R}^n$ and $\mathcal{Y} \subseteq \mathbb{N}$
        \item \textbf{Regression Problems:} $\mathcal{X} \subseteq \mathbb{R}^n$ and $\mathcal{Y} \subseteq \mathbb{R}$
    \end{itemize}
\end{definition}

\subsection{Feature Spaces}
\begin{definition}
    It is often easier to learn relationships from high-level features (instead of the raw input).
\end{definition}

\subsection{Feasibility of Learning}
\begin{motivation}
    More than one function (hypothesis) may be consistent with the data.
\end{motivation}

\begin{notes}
    So it may appear that finding the correct one should be impossible. 
\end{notes}

\subsubsection{Probably Approximately Correct (PAC) Estimations}
\begin{example}
    Take $N$ i.i.d. samples (i.e. take out a ball from an urn, record its color, and put it back in).
    \begin{itemize}
        \item $\nu \rightarrow \mu$ (empirical distribution $\rightarrow$ true distribution) as $N \rightarrow \infty$
    \end{itemize}
\end{example}

\subsubsection{Hoeffding's Inequality}
\begin{definition}
    Let $\mu$ denote the probability of an event, and $\nu$ denote its relative frequency in a sample size $N$. Then, for any $\epsilon > 0$,
    \begin{equation}
        P(|\nu - \mu| > \epsilon) \leq 2e^{-2\epsilon^2N}
    \end{equation}
    \begin{itemize}
        \item $\nu$: Relative frequency in the sample (known)
        \item $\mu$: Probabillity of drawing a blue ball (unknown)
        \item $N \rightarrow \infty$: $\nu \rightarrow \mu$
        \item $\epsilon$: How close we want $\nu$ to be to $\mu$
        \item $\epsilon \rightarrow 0$: Probability will be 1
        \item $\epsilon \rightarrow \infty$: $\nu \rightarrow \mu$
        \item $\mu \overset{?}{\approx} \nu $: $\mu$ is probably approximately equal to $nu$.
    \end{itemize}
\end{definition}

\begin{warning}
    We can approximate the true distribution with high probability by taking a large enough sample size, NOT guaranteeing that we can find the true distribution.
    \begin{itemize}
        \item Don't need to know where this theorem comes from.
    \end{itemize}
\end{warning}

Consider determining the class of a randomly chosen target point. If we ask a K-ary question about the points in $\mathcal{D}$

\subsubsection{PAC Learning}

\subsection{Decision Trees}

\subsubsection{Structure of a Decision Tree}


\newpage

\begin{center}
    \section*{L5 Optimizing Hyperparameters}
\end{center}
\begin{summary}
    \begin{itemize}
        \item What stategies can help a NN converge when training?
        \item What hyperparameters does a NN architecture have?
        \item How can we optimize parameters without gradients? 
        \item DL requires a lot of data, what can we do when data is scarce?
    \end{itemize}
\end{summary}

\section{Blackbox Optimization}
\begin{motivation}
    Also known as derivative free optimization, as the derivative is unknown, so you have to use derivative-free or heuristic methods. 
    \begin{equation*}
        x^* = \arg\min_{x \in X} f(x)
    \end{equation*}
    \customFigure[0.5]{../Images/L5_2.png}{}
\end{motivation}

\subsection{Parameters \& Hyperparameters}
\begin{definition}
    Distinction b/w model setting elements and tuning knobs.
    \begin{itemize}
        \item \textbf{Parameters:} Learnable parameters $(W,b)$
        \begin{itemize}
            \item Opt: Gradient Descent
        \end{itemize}
        \item \textbf{Hyperparameters:} Non-differentiable parameters (i.e. discrete)
        \begin{itemize}
            \item E.g. Number of layers, hidden dim, activation, normalization, dropout, ...
            \item Opt: Heuristics
        \end{itemize}
        \customFigure[0.5]{../Images/L5_0.png}{}
    \end{itemize}
\end{definition}
\newpage

\begin{summary}
    \begin{center}
        \begin{tabular}{l}
        \toprule
        \textbf{Types} \\
        \midrule
        \textbf{Grid Search} \\
        \multicolumn{1}{p{\linewidth}}{
        \begin{itemize}
            \item Exhaustive evaluation across a predefined set of values.
            \customFigure[0.3]{../Images/L5_3.png}{}
        \end{itemize}} \\
        \midrule
        \textbf{Coordinate Descent} \\
        \multicolumn{1}{p{\linewidth}}{
        \begin{itemize}
            \item Optimize each hyperparameter one at a time. 
            \customFigure[0.3]{../Images/L5_4.png}{}
        \end{itemize}} \\
        \midrule
        \textbf{Grad-Student Descent} \\
        \multicolumn{1}{p{\linewidth}}{
        \begin{itemize}
            \item Manual and ad-hoc, i.e. "follow your heart"
            \customFigure[0.3]{../Images/L5_5.png}{}
        \end{itemize}} \\
        \midrule
        \textbf{Random Search} \\
        \multicolumn{1}{p{\linewidth}}{
        \begin{itemize}
            \item Sampling hyperparameter configurations from defined distributions.
            \customFigure[0.3]{../Images/L5_6.png}{}
        \end{itemize}} \\
        \bottomrule
        \end{tabular}
    \end{center}
\end{summary}

\section{Bayesian Optimization}
\begin{definition}
    A principled sequential approach for efficient global optimization
    \vspace{1em}

    \textbf{Ingredients:}
    \begin{itemize}
        \item \textbf{Function, $f(x)$}: The numerical values we want to optimize.
        \item \textbf{Space to optimize, $X$}: Parameters or decisions or degrees of freedom to explore.
        \item \textbf{Bayesian model, $g(x)$}: Provides prediction ($\mu$) and uncertainty ($\sigma$).
        \item \textbf{Acquisition function, $A(\mu, \sigma)$}: A strategy to trade off exploration \& exploitation.
    \end{itemize}
\end{definition}

\subsection{Surrogate Model}
\begin{notes}
    Let's approximate the expensive function $f(x)$ with a cheaper function $g(x)$ to model prediction ($\mu$) and uncertainty ($\sigma$).
    \begin{itemize}
        \item Kernel models
        \item Gaussian processes (GP)
        \item Gradient boosted trees
        \item Neural networks
    \end{itemize}
\end{notes}

\subsubsection{Acquisition Function}
\begin{notes}
    Let's mix exploitation and exploration; sometimes, it pays off to explore areas where we have little information.
    \begin{itemize}
        \item Acquisition functions encapsulate the heuristic of what to sample next, how useful is unobserved data?
        \item E.g. Expected Improvement (EI), Probability of Improvement (PI), Upper Confidence Bound (UCB), ...
        \item $\mu$: exploitation 
        \item $\sigma$: exploration 
    \end{itemize}
\end{notes}

\subsection{BayesOpt Loop}
\begin{definition}
    Iterative process of modelling and sampling. 
    \begin{enumerate}
        \item Set a termination criteria (budget, iterations, maxima)
        \item Evaluate $f(x)$ on initial set of points (random)
        \item \textbf{Loop:} while criteria is not met:
        \begin{enumerate}
            \item Update surrogate model on all data
            \item Optimize acquisition function to find a maxima ($x_{\text{new}}$)
            \item Evaluate $f(x_{\text{new}}$)
        \end{enumerate}
    \end{enumerate}
\end{definition}

\begin{warning}
    \begin{itemize}
        \item Define your hyperparameter space (bounds, datatypes, etc.)
        \item Simplify it. 
        \item Use a platform to launch monitor and launch models. 
        \item \textbf{Libraries:} Optuna, Ray, BoTorch, Ax...
    \end{itemize}
\end{warning}

\begin{example}
    \begin{itemize}
        \item 2 random points
        \item Criteria: 10 evaluations.
        \item Normalize $f(x)$ to a smaller range b/c gradients are sensitive to scale.
        \item Repeat for 10 iterations:
        \begin{itemize}
            \item Pick points that maximizes acquistion function.
            \item Update surrogate model w/ the new point and its evaluation, i.e. $f(x_i)$ to get more certainty and better predictions.
        \end{itemize}
    \end{itemize}
    \vspace{1em}

    Look at L5.
\end{example}


\newpage

\begin{center}
    \section*{L6: Representation Learning and Variational Autoencoders}
\end{center}
\subsection{Bayesian Network}
\begin{definition}
    Vertices represent random variables and edges represent dependencies between variables.
\end{definition}

\subsubsection{Junction}
\begin{definition}
    A \textbf{junction} consists of three vertices, $X_1$, $X_2$, and $X_3$, connected by two edges, $e_1$ and $e_2$:
    \begin{itemize}
        \item Both arrows pointing in one direction
        \item Both arrows pointing in opposite directions
        \item One arrow pointing in each direction
    \end{itemize}
\end{definition}

\begin{warning}
    Want to look for causal relationships. 
    Arrows, what's causing what, what's influencing what.
\end{warning}

\subsubsection{Causal Chain}
\begin{definition}
    A causal chain is a junction of the following form:
    \begin{itemize}
        \item $X_1$ and $X_2$ are dependent. $X_2$ is dependent on $X_1$. Vice versa. From a causal perspective, $X_1$ is influencing $X_2$. Subtle difference, just bc $X_1 \rightarrow X_2$.
        \item $X_2$ and $X_3$ are dependent.
        \item $X_1$ and $X_3$ are dependent. 
        \begin{itemize}
            \item Given $X_2$, $X_1$ and $X_3$ are independent. Why? $X_2$'s door closes when you know $X_2$, so $X_1$ and $X_3$ are independent.
        \end{itemize}
    \end{itemize}
\end{definition}

\begin{warning}
    $X_1$ is influeincing $X_2$ and $X_2$ is influencing $X_3$.
\end{warning}

\subsubsection{Common Cause}
\begin{definition}
    A common cause is a junction of the form: 
\end{definition}

\begin{notes}
    \begin{itemize}
        \item $X_1$ and $X_3$ are dependent. 
        \begin{itemize}
            \item Given $X_2$, $X_1$ and $X_3$ are independent. Why? $X_2$ whether you smoke or not, $X_1$ whether you have yellow teeth, $X_3$ whether you have lung cancer, if you don't know $X_2$, if they have yellow teeth, then they might smoke, then they might have lung cancer. If you know $X_2$, yellow teeth and lung cancer are independent b/c you already know if they smoke or not, and yellow teeth implies smoke, 
        \end{itemize}
    \end{itemize}
\end{notes}

\subsubsection{Common Effect}
\begin{definition}
    A common effect is a junction of the form:
    
\end{definition}

\begin{notes}
    \begin{itemize}
        \item $X_1$ and $X_3$ are independent. 
        \item Given $X_2$ or any of $X_2$'s descendents, $X_1$ and $X_3$ are dependent.
    \end{itemize}
\end{notes}

\begin{warning}
    Just b/c you don't know something about the middle variable, then it can be independent
\end{warning}

\begin{example}
    $X_2$ Grass being wet, $X_1$ raining, and $X_3$ sprinkler being on.
    \begin{itemize}
        \item If you know the grass is wet, you know that either the sprinkler is on or it's raining.
        \begin{itemize}
            \item If it didn't have the sprinkler on, then it must have rained.
            \item If it didn't rain, then the sprinkler must have been on.
            \item So this means that $X_1$ and $X_3$ are dependent given $X_2$.
        \end{itemize}
        \item If you don't know the grass is wet, then $X_1$ and $X_3$ are independent b/c you don't know if it rained or the sprinkler was on.
    \end{itemize}
\end{example}

\begin{example}
    \begin{enumerate}
        \item \textbf{Given:} Caveman is deciding whether to go hunt for meat. He must take into account several factors:
        \begin{itemize}
            \item Weather
            \item Possibility of over-exertion
            \item Possibility encountering lion
        \end{itemize}

        These factors can result in Cavemen's death. His decision will ultimately depend on the \textbf{chances} of his death.
        \item \textbf{Binary Variables:}
        \begin{itemize}
            \item $W = \{\text{Sun}, \text{Rainy}\}$: Weather
            \item $H$: Whether the Cavemen goes hunting or not.
            \item $L$: Whether the Cavemen encounters a lion or not.
            \item $T$: Whether the Cavement is tired or not.
            \item $D$: Whether the Cavemen dies or not
        \end{itemize}
        \item \textbf{Problem:} Cavemen must decide whether to go hunting or not. 
        \begin{itemize}
            \item He must consider the conditional probabilities (i.e. dependence) of each event.
        \end{itemize}
    \end{enumerate}
\end{example}

\begin{warning}
    Have to be discrete. 
\end{warning}


\newpage

% \subsection{Recap}
% \begin{summary}
%     \begin{itemize}
%         \item What is a MLP? Vector-in vector-out optimizable, learnable transformation of data.
%         \item What is an inductive bias and why might they be useful? Set of assumptions that the learner puts on a model for a task, makes an algorithm learn one pattern over another.
%         \begin{itemize}
%             \item Let certain patterns be learnable (restricting the hypothesis space).
%             \item Last layer in the GLM: Restricting output values to 0 and 1. 
%         \end{itemize}
%         \item What is the difference between hyperparameters and parameters? Hyperparameters are set before training (usually discrete), parameters are learned during training (continuous to learn).
%         \item How do we optimize all parameters in a model? Back propagation.
%     \end{itemize}
% \end{summary}

\begin{center}
    \section*{L7: Software Developemnt, Python Programming}
\end{center}
\subsection{Problem Setup}
\begin{definition}
    Given a Bayesian network, $\mathcal{B} = (\mathcal{V}, \mathcal{E})$, where $\mathcal{V} = \{X_1, \dots, X_{|\mathcal{V}|}\}$, we want to find the value of:
    \[
    \operatorname{pr}(\mathbf{Q} \mid \mathbf{E}) := \operatorname{pr}(Q_1, \dots, Q_{|\mathbf{Q}|} \mid E_1, \dots, E_{|\mathbf{E}|}) = \frac{\sum_{\mathcal{V} \setminus (\mathbf{Q} \cup \mathbf{E})} p(X_1, \dots, X_{|\mathcal{V}|})}
    {\sum_{\mathcal{V} \setminus \mathbf{E}} p(X_1, \dots, X_{|\mathcal{V}|})}
    \]
    \[
    \operatorname{pr}(\mathbf{Q} \mid \mathbf{E}) \propto 
    \sum_{\mathcal{V} \setminus (\mathbf{Q} \cup \mathbf{E})} 
    \left( p(X_1) \prod_{i \neq 1} p(X_i \mid \operatorname{pts}(X_i)) \right)
    \]


    \begin{itemize}
        \item $\mathbf{Q} = \{Q_1, \dots, Q_{|\mathbf{Q}|}\}$: Query variables
        \item $\mathbf{E} = \{E_1, \dots, E_{|\mathbf{E}|}\} \subseteq \mathcal{V}$: Evidence variables
        \item $\mathbf{Q} \cap \mathbf{E} = \emptyset$.
    \end{itemize}
\end{definition}

\begin{warning}
    \begin{itemize}
        \item Denominator: Normalization constant (assuming $\mathbf{E}$ is fixed)
        \item Therefore, only need to compute numerator (w/o specifying $\mathbf{Q}$), which we can then normalize w.r.t. $\mathbf{Q}$
    \end{itemize}
\end{warning}

\subsubsection{Joint Distribution in a Bayesian Network}
\begin{derivation}
    For any joint distribution, the following factorization holds:
    \begin{equation*}
        p(X_1,\ldots,X_{|p|})  = p(X_1) \prod_{i \neq 1} p(X_i \mid X_1,\ldots, X_{i-1})
    \end{equation*}

    \textbf{Bayesian Network Conditions:} If 
    \begin{itemize}
        \item at least 1 variable will be an orphan (i.e. no parents)
        \item no variable is both ancestor and descendant of another. 
    \end{itemize}
    \vspace{1em}

    then this allows us to order $X_1,\ldots,X_{|\mathcal{V}|}$,  so that if $X_j$ is a descendent of $X_i$, then for any $j > i$, 
    \begin{equation*}
        \text{pts}(X_i) \subseteq \{X_1,\ldots,X_{i-1}\} \text{ and } X_1,\ldots,X_{i-1} \notin \text{des}(X_i)
    \end{equation*}

    Therefore, using the consequence of dependence separation, then 
    \begin{equation*}
        p(X_1,\ldots,X_{|\mathcal{V}|}) = p(X_1) \prod_{i \neq 1} p(X_i \mid \text{pts}(X_i))
    \end{equation*}
\end{derivation}

\subsection{Method 1: Bayesian Network Inference}

\subsubsection{Markov Blanket}
\begin{definition} 
    The \textbf{Markov blanket} of a variable $X$, denoted $\operatorname{mbk}(X)$, consists of the following variables:
    \begin{itemize}
        \item $X$'s children
        \item $X$'s parents
        \item The other parents of $X$'s children, excluding $X$ itself.
    \end{itemize}
    which is when a variable, $X$, is "eliminated", the resulting factor's scope is the Markov blanket of $X$.
\end{definition}

\subsubsection{Graphical Interpretation}
\begin{notes}
    Pictorially, eliminating $X$ is equivalent to replacing all hyper-edges that include $X$ with their union minus $X$, and then removing $X$.
\end{notes}

\subsubsection{Elimination Ordering}
\begin{definition}
    The order that the variables are eliminated.
    \begin{itemize}
        \item This creates a sequence of hyper-graphs that depend on the elimination ordering.
    \end{itemize}
\end{definition}

\subsubsection{Elimination Width}
\begin{definition}
    The \textbf{elimination width} of a sequence of hyper-graphs is the \# of variables in the hyper-edge within the sequence with the most variables.
\end{definition}

\subsubsection{Heuristics for Elimination Ordering}
\begin{definition}
    Choose the elimination ordering to minimize the elimination width using the following heuristics:
    \begin{enumerate}
        \item Eliminate variable with the fewest parents.
        \item Eliminate variable with the smallest domain for its parents, where
        \[
        |\operatorname{dom}(\operatorname{pts}(X))| = \prod_{Z \in \operatorname{pnt}(X)} |\operatorname{dom}(Z)|.
        \]
        \item Eliminate variable with the smallest Markov blanket.
        \item Eliminate variable with the smallest domain for its Markov blanket, where
        \[
        |\operatorname{dom}(\operatorname{mbk}(X))| = \prod_{Z \in \operatorname{embk}(X)} |\operatorname{dom}(Z)|.
        \]
    \end{enumerate}
\end{definition}

\begin{warning}
    Choosing the variable with the smallest domain for its Markov blanket is the most effective heuristic.
\end{warning}
\newpage

\subsection{Method 2: Inference via Sampling}
\begin{definition}
    Generate a large \# of samples and then approximate as:
    \[
    p(\mathbf{Q} \mid \mathbf{E}) \approx \frac{\# \text{ of samples w/ } \mathbf{Q} \text{ and } \mathbf{E}}{\# \text{ of samples w/ } \mathbf{E}}.
    \]
    \begin{itemize}
        \item As $\# \text{ of samples} \to \infty$, the approximation becomes exact.
    \end{itemize}
\end{definition}

\subsubsection{Inference via Sampling with Likelihood Weighting}
\begin{motivation}
    Most of the samples are wasted since they are not consistent with the evidence.
\end{motivation}

\begin{definition}
    Generate a large \# of samples and then approximate as:
    \[
    p(\mathbf{Q} \mid \mathbf{E}) \approx \frac{\text{weight of samples w/ } \mathbf{Q} \text{ and } \mathbf{E}}{\text{weight of samples w/ } \mathbf{E}}.
    \]
    \begin{itemize}
        \item Weight for each sample: Probability of forcing the evidence, i.e. probability of the evidence given the sample.
    \end{itemize}
\end{definition}
\newpage

\subsection{Canonical Problems:}
% \begin{example}
%     \begin{enumerate}
%         \item \textbf{Given:} Caveman is deciding whether to go hunt for meat. He must take into account several factors:
%         \begin{itemize}
%             \item Weather
%             \item Possibility of over-exertion
%             \item Possibility encountering lion
%         \end{itemize}

%         These factors can result in Cavemen's death. His decision will ultimately depend on the \textbf{chances} of his death.
%         \item \textbf{Binary Variables:}
%         \begin{itemize}
%             \item $W = \{\text{Sun}, \text{Rainy}\}$: Weather
%             \item $H$: Whether the Cavemen goes hunting or not.
%             \item $L$: Whether the Cavemen encounters a lion or not.
%             \item $T$: Whether the Cavement is tired or not.
%             \item $D$: Whether the Cavemen dies or not
%         \end{itemize}
%         \item \textbf{Problem:} Cavemen must decide whether to go hunting or not. 
%         \begin{itemize}
%             \item He must consider the conditional probabilities (i.e. dependence) of each event.
%         \end{itemize}
%     \end{enumerate}
% \end{example}

% \begin{warning}
%     Have to be discrete. 
% \end{warning}
% \newpage

\subsubsection{Bayesian Inference via Variable Elimination}
\begin{process}
    \begin{enumerate}
        \item Given Bayesian network w/ variables and their conditional probabilities.
        \item Find the probability of the query variable given the evidence variable, $p(\mathbf{Q} \mid \mathbf{E})$.
        \item Use $p(\mathbf{Q} \mid \mathbf{E}) = \frac{\sum_{\mathcal{V} \setminus (\mathbf{Q} \cup \mathbf{E})} p(X_1, \dots, X_{|\mathcal{V}|})}{\sum_{\mathcal{V} \setminus \mathbf{E}} p(X_1, \dots, X_{|\mathcal{V}|})}$.
        \item Determine $p(X_1) \prod_{i \neq 1} p(X_i \mid \operatorname{pts}(X_i))$ using the Bayesian network. 
        \item Write out the summation of the numerator in an order using heuristics to determine elimination ordering. 
        \item Start with inner summation and work outwards.
        \item Calculate the probability of the query variable(s) given the evidence variable(s).
    \end{enumerate}
\end{process}

\begin{warning}
    \begin{itemize}
        \item Write the complement probability to make life easier. (HIGHLY RECOMMENDED)
        \item To determine the conditional probability summation of a variable, look at its parents (inward arrows)
        \item Inner sum must have all probabilities with that variable in it that you are summing over. 
        \item TO determine $g_?$, look at which variable you aren't summing over and also aren't query/evidence variables, then it will be a fn of the remaining variables.
    \end{itemize}
\end{warning}
\newpage

\begin{example} 
    \begin{enumerate}
        \item \textbf{Given:}
        \customFigure[0.5]{../Images/L6_9.png}{}
        \begin{center}
            \begin{tabular}{ll}
                \toprule
                \textbf{Variables} & \textbf{Values} \\
                \midrule
                $W$ & $P(\text{Sunny}) = 0.5 \mid P(\text{Rainy}) = 0.5$ \\
                \midrule
                $H$ & $P(h) = 0.5 \mid P(\lnot h) = 0.5$ \\
                \midrule
                $T$ & $P(t \mid \text{Sunny}, h) = 1 \mid P(t \mid \text{Sunny}, \lnot h) = 0.5 \mid P(t \mid \text{Rainy}, h) = 0.25 \mid P(t \mid \text{Rainy}, \lnot h) = 0$ \\
                & $P(\lnot t \mid \text{Sunny}, h) = 0 \mid P(\lnot t \mid \text{Sunny}, \lnot h) = 0.5 \mid P(\lnot t \mid \text{Rainy}, h) = 0.75 \mid P(\lnot t \mid \text{Rainy}, \lnot h) = 1$ \\
                \midrule
                $L$ & $P(l \mid h) = 1 \mid P(l \mid \lnot h) = 0.75$ \\
                & $P(\lnot l \mid h) = 0 \mid P(\lnot l \mid \lnot h) = 0.25$ \\
                \midrule
                $D$ & $P(d \mid t,l) = 0.75 \mid P(d \mid t,\lnot l) = 1 \mid P(d \mid \lnot t,l) = 0 \mid P(d \mid \lnot t,\lnot l) = 0$ \\
                & $P(\lnot d \mid t,l) = 0.25 \mid P(\lnot d \mid t,\lnot l) = 0 \mid P(\lnot d \mid \lnot t,l) = 1 \mid P(\lnot d \mid \lnot t,\lnot l) = 1$ \\
                \bottomrule
            \end{tabular}
        \end{center}
        \item \textbf{Problem:} $p(d \mid h)$? 
        \item \textbf{Soln:}
        \begin{enumerate}
            \item $p(d \mid h) = \frac{p(d,h)}{p(h)} = \frac{\sum_{W,T,L} p(W,h,T,L,d)}{\sum_{W,T,L,D} p(W,h,T,L,d)}$ by definition of query and evidence equations.
            \item $p(W,h,T,L,D) = p(h) p(W) p(L \mid h) p(t \mid W,h) p(D \mid T,L)$ by Bayesian network and $p(X_1, \dots, X_{|\mathcal{V}|}) = p(X_1) \prod_{i \neq 1} p(X_i \mid \operatorname{pts}(X_i))$.
        \end{enumerate}
        \begin{center}
            \begin{tabular}{l}
                \toprule
                \textbf{Summation} \\
                \toprule
                \multicolumn{1}{p{\linewidth}}{
                \begin{center}
                    $\text{Numerator}: \underbrace{p(h) \sum_L p(L \mid h) \underbrace{\sum_T p(D \mid T,L) \underbrace{\sum_W p(W) p(T \mid W,h)}_{g_1(T)}}_{g_2(L,D)}}_{g_3(D)}$
                \end{center}} \\
                \toprule
                \multicolumn{1}{p{\linewidth}}{
                \begin{center}
                    $g_1(T) = p(\text{Sunny}) p(T \mid \text{Sunny},h) + p(\text{Rainy}) p(T \mid \text{Rainy},h)$
                \end{center}} \\
                $g_1(t) = p(\text{Sunny}) p(t \mid \text{Sunny},h) + p(\text{Rainy}) p(t \mid \text{Rainy},h) = 0.5 \cdot 1 + 0.5 \cdot 0.25 = 0.625$ \\
                $g_1(\lnot t) =  p(\text{Sunny}) p(\lnot t \mid \text{Sunny},h) + p(\text{Rainy}) p(\not t \mid \text{Rainy},h) = 0.5 \cdot 0 + 0.5 \cdot 0.75 = 0.375$ \\
                \midrule
                \multicolumn{1}{p{\linewidth}}{
                \begin{center}
                    $g_2(L,D) = p(D \mid t,L) g_1(t) + p(D \mid \lnot t, L) g_1(\lnot t)$ 
                \end{center}} \\
                $g_2(l,d) = p(d \mid t,l) g_1(t) + p(d \mid \lnot t, l) g_1(\lnot t) = 0.75 \cdot 0.625 + 0 \cdot 0.375 = 0.46875$ \\
                $g_2(l,\lnot d) = p(\lnot d \mid t,l) g_1(t) + p(\lnot d \mid \lnot t, l) g_1(\lnot t) = 0.25 \cdot 0.625 + 1 \cdot 0.375 = 0.53125$ \\
                $g_2(\lnot l,d) = p(d \mid t,\lnot l) g_1(t) + p(d \mid \lnot t, \lnot l) g_1(\lnot t) = 1 \cdot 0.625 + 0 \cdot 0.375 = 0.625$ \\
                $g_2(\lnot l,\lnot d) = p(\lnot d \mid t,\lnot l) g_1(t) + p(\lnot d \mid \lnot t, \lnot l) g_1(\lnot t) = 0 \cdot 0.625 + 1 \cdot 0.375 = 0.375$ \\
                \midrule
                \multicolumn{1}{p{\linewidth}}{
                \begin{center}
                    $g_3(D) = p(h) p(l \mid h) g_2(l,D) + p(h) p(\lnot l \mid h) g_2(\lnot l, D)$ 
                \end{center}} \\
                $g_3(d) = p(h) p(l \mid h) g_2(l,d) + p(h) p(\lnot l \mid h) g_2(\lnot l,d) = (0.5)(1)(0.46875) + (0.5)(0)(0.625) = 0.234375$ \\
                $g_3(\lnot d) = p(h) p(l \mid h) g_2(l,\lnot d) + p(h) p(\lnot l \mid h) g_2(\lnot l,\lnot d) = (0.5)(1)(0.53125) + (0.5)(0)(0.375) = 0.265625$ \\
                \bottomrule
            \end{tabular}
        \end{center}
        \vspace{1em}

        \begin{equation*}
            p(d \mid h) = \frac{g_3(d)}{g_3(d) + g_3(\lnot d)} = \frac{0.234375}{0.234375 + 0.265625} = \frac{0.234375}{0.5} = 0.46875
        \end{equation*}
    \end{enumerate}
\end{example}
\newpage

\begin{example}
    \begin{center}
        \begin{tabular}{l}
            \toprule
            \textbf{Summation} \\
            \toprule
            \multicolumn{1}{p{\linewidth}}{
            \begin{center}
                $\text{Numerator}: \underbrace{p(h) \sum_L p(L \mid h) \underbrace{\sum_W p(W) \underbrace{\sum_T p(T \mid W,h) p(D \mid T,L)}_{g_1(W,D,L)}}_{g_2(D,L)}}_{g_3(D)}$
            \end{center}} \\
            \toprule 
            \multicolumn{1}{p{\linewidth}}{
            \begin{center}
                $\text{Numerator}: \underbrace{p(h) \sum_W p(W) \underbrace{\sum_T p(T \mid W,h) \underbrace{\sum_L p(L \mid h) p(D \mid T,L)}_{g_1(D,T)}}_{g_2(D,W)}}_{g_3(D)}$
            \end{center}} \\
            \toprule
            \multicolumn{1}{p{\linewidth}}{
            \begin{center}
                $\text{Numerator}: \underbrace{p(h) \sum_W p(W) \underbrace{\sum_L p(L \mid h) \underbrace{\sum_T p(T \mid W,h) p(D \mid T,L)}_{g_1(W,D,L)}}_{g_2(W,D)}}_{g_3(D)}$
            \end{center}} \\
            \toprule
            \multicolumn{1}{p{\linewidth}}{
            \begin{center}
                $\text{Numerator}: \underbrace{p(h) \sum_T p(T \mid W,h) \underbrace{\sum_W p(W) \underbrace{\sum_L p(L \mid h) p(D \mid T,L)}_{g_1(D,T)}}_{g_2(D,T)}}_{g_3(D)}$
            \end{center}} \\
            \toprule
            \multicolumn{1}{p{\linewidth}}{
            \begin{center}
                $\ldots$
            \end{center}} \\
            \bottomrule
        \end{tabular}
    \end{center}
\end{example}
\newpage

\begin{example}
    \begin{enumerate}
        \item \textbf{Given:}
        \customFigure[0.5]{../Images/L7_1.png}{}
        \item \textbf{Problem:} Compute $\Pr(s = 1 \mid t = 1)$ if $\Pr(G = 1) = 0.3$, $\Pr(E = 1) = 0.4$, and the conditional probability tables for $S, Y,$ and $T$ are given below.
        \item \textbf{Solution:}
        \begin{enumerate}
            \item Derivation of $p(s=1 \mid t=1)$: 
            \begin{align*}
                p(s=1 \mid t=1) &= \frac{p(s=1,t=1)}{p(t=1)} \\
                &= \frac{P(s=1,t=1)}{\sum_S P(S,t=1)} \\
                &= \frac{\sum_{E,G,Y} P(E,G,Y,s=1,t=1)}{\sum_S \sum_{E,G,Y} P(E,G,Y,S,t=1)}
            \end{align*}
            \item Summation Term: 
            \begin{align*}
                & \sum_{E,G,Y} p(E) p(G) p(S \mid G,E) p(Y \mid S) p(t=1 \mid S) \\
                & \underbrace{p(t=1 \mid S) \sum_E p(E) \underbrace{\sum_G p(G) p(S \mid G,E) \underbrace{\sum_Y p(Y \mid S)}_{g_1(S)}}_{g_2(E,S)}}_{g_3(S)} \quad \text{one possible ordering}
            \end{align*}
            \begin{itemize}
                \item Conditional probability and individual probabilities come from Bayesian network, and set $t,s=1$ due to the query and evidence variables.
            \end{itemize}
            \item Choose: 
            \[
            \underbrace{p(t=1 \mid S) \sum_E p(E) \underbrace{\sum_G p(G) p(S \mid G,E) \underbrace{\sum_Y p(Y \mid S)}_{g_1(S)}}_{g_2(E,S)}}_{g_3(S)}
            \]
            \item $g_1(S)$: 
            \begin{align*}
                g_1(S) &= p(Y=1 \mid S) + p(Y=0 \mid S) \\
                &= \begin{cases}
                    0.1 + 0.9 = 1 \quad \text{if } S=0 \\
                    0.8 + 0.2 = 1 \quad \text{if } S=1 \\
                \end{cases} 
            \end{align*}
            \item $g_2(E,S)$: 
            \begin{align*}
                g_2(E,S) &= p(G=1) p(S \mid G=1,E) g_1(S) + p(G=0) p(S \mid G=0,E) g_1(S) \\
                g_2(E,S) &= p(G=1) p(S \mid G=1,E) + p(G=0) p(S \mid G=0,E) \\
                &= \begin{cases}
                    0.3(0.5) + 0.7(0.9) \quad \text{if } E=0, S=0 \\
                    0.3(0.5) + 0.7(0.1) \quad \text{if } E=0, S=1 \\
                    0.3(0.3) + 0.7(0.6) \quad \text{if } E=1, S=0 \\
                    0.3(0.7) + 0.7(0.4) \quad \text{if } E=1, S=1 \\
                \end{cases} \\
                &= \begin{cases}
                    0.78 \quad \text{if } E=0, S=0 \\
                    0.22 \quad \text{if } E=0, S=1 \\
                    0.51 \quad \text{if } E=1, S=0 \\
                    0.49 \quad \text{if } E=1, S=1 \\
                \end{cases}
            \end{align*}
            \item $g_3(S)$: 
            \begin{align*}
                g_3(S) &= p(t=1 \mid S) p(E=1) g_2(E=1,S) + p(t=1 \mid S) p(E=0) g_2(E=0,S) \\
                &= \begin{cases}
                    0.2(0.4)(0.51) + 0.2(0.6)(0.78) \quad \text{if } S=0 \\
                    0.2(0.4)(0.49) + 0.2(0.6)(0.22) \quad \text{if } S=1 \\
                \end{cases} \\
                &= \begin{cases}
                    0.1344 \quad \text{if } S=0 \\
                    0.2952 \quad \text{if } S=1 \\
                \end{cases} 
            \end{align*}
            \item $p(s=1 \mid t=1)$: 
            \begin{align*}
                p(s=1 \mid t=1) &= \frac{g_3(1)}{g_3(0) + g_3(1)} \\
                &= \frac{0.2952}{0.2952 + 0.1344} \\
                &= 0.6875
            \end{align*}
        \end{enumerate}
    \end{enumerate}

    % \begin{center}
    %     \begin{tabular}{l}
    %         \toprule
    %         \textbf{Summation} \\
    %         \toprule
    %         \multicolumn{1}{p{\linewidth}}{
    %         \begin{center}
    %             $\text{Numerator}: \underbrace{p(t=1 \mid s=1) \sum_E p(E) \underbrace{\sum_G p(G) p(s=1 \mid G,E) \underbrace{\sum_Y p(Y \mid s=1)}_{g_1}}_{g_2}}_{g_3}$
    %         \end{center}} \\
    %         \toprule
    %         \multicolumn{1}{p{\linewidth}}{
    %         \begin{center}
    %             $g_1 = p(Y=1 \mid S=1) + p(Y=0 \mid S=1) = 0.9 + 0.1 = 1$
    %         \end{center}} \\
    %         \midrule
    %         \multicolumn{1}{p{\linewidth}}{
    %         \begin{center}
    %             $g_2(E) = (p(g=1) p(s=1 \mid g=1, E) + p(g=0) p(s=1 \mid g=0, E)) g_1$
    %         \end{center}} \\
    %         $g_2(e=1,s=1) = 0.3(0.7) + 0.7(0.4) = 0.49$ \\
    %         $g_2(e=0,s=1) = 0.3(0.5) + 0.7(0.1) = 0.22$ \\
    %         $g_2(e=1,s=0) = 0.3(0.3) + 0.7(0.6) = 0.51$ \\
    %         $g_2(e=0,s=0) = 0.3(0.5) + 0.7(0.9) = 0.78$ \\
    %         \midrule
    %         \multicolumn{1}{p{\linewidth}}{
    %         \begin{itemize}
    %             \item $g_3(t=1 \mid s=1) = 0.9 p(e=1) g_2(e=1) + 0.9 p(e=0) g_2(e=0) = 0.9(0.4)(0.49) + 0.9(0.6)(0.22) = 0.2952$ 
    %             \item $g_3(t=1 \mid s=0) = 0.2 p(e=1) g_2(e=1) + 0.2 p(e=0) g_2(e=0) = 0.2(0.4)(0.51) + 0.2(0.6)(0.78) = 0.1344$
    %         \end{itemize}} \\
    %         \bottomrule
    %     \end{tabular}
    % \end{center}

    % \begin{equation*}
    %     p(s=1 \mid t=1) = \frac{g_3}{\sum_S p(t=1,S)} = \frac{0.2952}{0.2952 + 0.1344} = \frac{0.2952}{0.4296} = 0.6875
    % \end{equation*}
\end{example}
\newpage

\begin{example} 
    \begin{enumerate}
        \item \textbf{Given:} Consider the following Bayesian network, where $A,B,C,D$ are binary R.V. over $\{0,1\}$
        \customFigure[0.5]{../Images/FE23_3.png}{}
        \customFigure[0.5]{../Images/FE23_2.png}{}
        \item \textbf{Problem:} Find $P(A=0 \mid C=0)$ and $P(D=1 \mid C=0)$.
        \item \textbf{Solution:}
        \begin{enumerate}
            \item Derivation of \( P(D=1 \mid C=0) \): 
            \begin{align*}
                P(D=1 \mid C=0) &= \frac{P(D=1,C=0)}{P(C=0)} \quad \text{by definition}\\
                &= \frac{P(D=1,C=0)}{\sum_d P(D=d,C=0)} \quad \text{marginalize over $D$}\\
                &= \frac{\sum_{A,B} P(A,B,C=0,D=1)}{\sum_d \sum_{A,B} P(A,B,C=0,D=d)} \quad \text{equation in problem setup} 
            \end{align*}
            \begin{itemize}
                \item Summing over the variables that are not in the query and evidence variables.
            \end{itemize}
            
            \item Summation Term: 
            \begin{align*}
                \sum_{A,B} & P(A) P(B \mid A) P(C=0 \mid A) P(D=d \mid B,C=0) \quad \text{Bayesian network} \\
                \sum_A & P(A) P(C=0 \mid A) \sum_B P(B \mid A) P(D=d \mid B,C=0) \quad \text{(1st ordering)} \\
                \sum_B & P(D=d \mid B,C=0) \sum_A P(A) P(B \mid A) P(C=0 \mid A) \quad \text{(2nd ordering)}
            \end{align*}
            
            \item Choose: 
            \[
            \underbrace{\sum_B P(D=d \mid B,C=0) \underbrace{\sum_A P(A) P(B \mid A) P(C=0 \mid A)}_{g_1(B)}}_{g_2(d)}
            \]
            
            \item \( g_1(B) \):
            \begin{align*}
                g_1(B) &= P(A=0)P(B \mid A=0) P(C=0 \mid A=0) + P(A=1)P(B \mid A=1) P(C=0 \mid A=1) \\
                &= \begin{cases}
                    0.9(0.6)(0.8) + 0.1(0.5)(0.6) \quad \text{if } B=0 \\
                    0.9(0.4)(0.8) + 0.1(0.5)(0.6) \quad \text{if } B=1
                \end{cases} \\
                &= \begin{cases}
                    0.462 \quad \text{if } B=0 \\
                    0.318 \quad \text{if } B=1
                \end{cases} 
            \end{align*}
            
            \item \( g_2(d) \): 
            \begin{align*}
                g_2(d) &= P(D=d \mid B=0, C=0) g_1(B=0) + P(D=d \mid B=1, C=0)g_1(B=1) \\
                &= \begin{cases}
                    0.5(0.462) + 0.5(0.318) \quad \text{if } d=0 \\
                    0.5(0.462) + 0.5(0.318) \quad \text{if } d=1
                \end{cases} \\
                &= \begin{cases}
                    0.39 \quad \text{if } d=0 \\
                    0.39 \quad \text{if } d=1
                \end{cases} 
            \end{align*}
            
            \item \( P(D=1 \mid C=0) = \frac{g_2(1)}{g_2(0) + g_2(1)} = \frac{0.39}{0.39 + 0.39} = 0.5 \)
        \end{enumerate}   
        \item \textbf{Solution 2:}
        \begin{enumerate}
            \item Derivation of \( P(A=0 \mid C=0) \):
            \begin{align*}
                P(A=0 \mid C=0) &= \frac{P(A=0,C=0)}{P(C=0)} \\
                &= \frac{P(A=0,C=0)}{\sum_a P(A=a,C=0)} \\
                &= \frac{\sum_{B,D} P(A=0,B,C=0,D)}{\sum_a \sum_{B,D} P(A=a,B,C=0,D)}
            \end{align*}
            \item Summation Term:
            \begin{align*}
                \sum_{B,D} & P(A=a) P(B \mid A=a) P(C=0 \mid A=a) P(D \mid B,C=0) \quad \text{Bayesian network} \\
                P(C=0 \mid A=a) \sum_B & P(B \mid A=a) P(A=a \mid B,C=0) \sum_D P(D \mid B,C=0) \quad \text{(1st ordering)} \\
                P(C=0 \mid A=a) \sum_D & P(D \mid B,C=0) \sum_B P(B \mid A=a) P(A=a \mid B,C=0) \quad \text{(2nd ordering)}
            \end{align*}
            \item Choose:
            \[
            P(C=0 \mid A=a) \underbrace{\sum_B P(B \mid A=a) P(A=a \mid B,C=0) \underbrace{\sum_D P(D \mid B,C=0)}_{g_1(B)}}_{g_2(A)}
            \]
            \item Same as before.
        \end{enumerate}     
    \end{enumerate}
\end{example}
\newpage

\subsubsection{Hypergraph}
\begin{process} Process of eliminating a variable. 
    \begin{enumerate}
        \item Create a Hyper-graph by creating a node for each variable. 
        \item Create hyper-edges (factors) by circling the nodes based on of its parents (i.e. arrows pointing into a variable). If no parents, circle itself.
        \item Select a variable $v$ that we are summing over. 
        \begin{enumerate}
            \item Circle all the variables that have $v$ in their hyperedge into one big hyperedge (i.e. union of hyper-edges).
            \item Eliminate $v$ by removing the node. 
            \item Calculate the factor by multiplying the support of the variables in the union of hyperedges.
        \end{enumerate}
        \item Repeat the process for all other $v$. 
        \item Select the smallest factor to eliminate first.
        \item Repeat until all variables are eliminated to determine the best ordering of elimination. 
        \begin{itemize}
            \item The first eliminated variable will be the inner sum. 
        \end{itemize}
    \end{enumerate}
\end{process}
\newpage

\begin{example}
    \customFigure[0.5]{../Images/L6_11.png}{}
    \begin{itemize}
        \item Since these are all binary variables, we are selecting the factor with the least number of variables to eliminate first.
    \end{itemize}
    \customFigure[0.75]{../Images/L7_0.png}{}
\end{example}
\newpage

\begin{example}
    \begin{enumerate}
        \item \textbf{Given:} Bayesian network 
        \customFigure[0.5]{../Images/FE23_4.png}{}
        with cardinality of the support of each variable (i.e. number of values each variable can take on) as follows:
        \begin{itemize}
            \item $A$: $2^4$
            \item $B$: $2^2$
            \item $C$: $2^{12}$
            \item $D$: $2^2$
            \item $E$: $2^3$ 
            \item $F$: $2^6$
        \end{itemize}
        Suppose elimination ordering is chosen so that the next variable eliminated is the one that results in the smallest factor (breaking ties alphabetically).
        \item \textbf{Problem 1:} How many variables must be eliminated to compute $P(A,F \mid C)$?
        \item \textbf{Solution 1:}
        \begin{enumerate}
            \item Since $A$, $F$ are query, and $C$ is evidence, we must eliminate $B$, $D$, and $E$, so 3 variables must be eliminated.
        \end{enumerate}
        \item \textbf{Problem 2:} What is the first variable to be eliminated to compute $P(F \mid A)$? 
        \item \textbf{Solution 2:}
        \begin{enumerate}
            \item Try eliminating all variables that aren't query or evidence and count \# of variables in union of hyperedges.
            \customFigure[0.5]{../Images/FE23_5.png}{}
            \begin{enumerate}
                \item Eliminate $B$: Hyperunion is ACDF $\rightarrow$ $2^4 \cdot 2^{12} \cdot 2^2 \cdot 2^6 = 2^{24}$
                \item Eliminate $C$: Hyperunion is ABDEF $\rightarrow$ $2^4 \cdot 2^2 \cdot 2^2 \cdot 2^3 \cdot 2^6 = 2^{17}$
                \item Eliminate $D$: Hyperunion is BCF $\rightarrow$ $2^2 \cdot 2^{12} \cdot 2^6 = 2^{20}$
                \item Eliminate $E$: Hyperunion is AC $\rightarrow$ $\boxed{2^4 \cdot 2^{12} = 2^{16}}$
            \end{enumerate}
            \item Choose $E$ as the first variable to be eliminated because it has the lowest support in its hyperunion.
        \end{enumerate}
        \item \textbf{Problem 3:} What is the second variable to be eliminated to compute $P(F \mid A)$?
        \item \textbf{Solution 3:}
        \begin{enumerate}
            \item Try eliminating all variable except $F,A,E$.
            \customFigure[0.5]{../Images/FE23_6.png}{}
            \begin{enumerate}
                \item Eliminate $B$: Hyperunion is ACDF $\rightarrow$ $2^4 \cdot 2^{12} \cdot 2^2 \cdot 2^6 = 2^{24}$
                \item Eliminate $C$: Hyperunion is ABDF $\rightarrow$ $\boxed{2^4 \cdot 2^2 \cdot 2^2 \cdot 2^6 = 2^{14}}$
                \item Eliminate $D$: Hyperunion is BCF $\rightarrow$ $2^2 \cdot 2^{12} \cdot 2^6 = 2^{20}$
            \end{enumerate}
            \item Choose $C$ as the second variable to be eliminated because it has the lowest support in its hyperunion.
        \end{enumerate}
    \end{enumerate}
\end{example}
\newpage

\subsubsection{Inference via Sampling}
\begin{process}
    \begin{enumerate}
        \item Given samples 
        \item Calculate number of samples w/ the query and evidence variables.
        \item Calculate number of samples w/ the evidence variables.
        \item Approximate the probability of the query variable given the evidence variable by dividing the \# of samples w/ the query and evidence variables by the \# of samples w/ the evidence variables.
    \end{enumerate}
\end{process}

\begin{example}
    \begin{enumerate}
        \item \textbf{Given:} Samples 
        \customFigure[0.5]{../Images/L6_12.png}{}
        \item \textbf{Problem:} Find the probability of $p(d \mid h)$.
        \item \textbf{Soln:} $p(d \mid h) \approx \frac{\# \text{ of samples w/ } d \text{ and } h}{\# \text{ of samples w/ } h} = \frac{3}{5} = 0.6$.
    \end{enumerate}
\end{example}
\newpage

\subsubsection{Probability Review}
\begin{example}
    \begin{enumerate}
        \item[(a)] [1 pts] Assume \( A, B, C \) are random variables where \( A \perp B \), \( B \perp C \), and \( C \perp A \). Which of the following expressions are equivalent to \( P(A, B) \)?
        
        \begin{itemize}
            \item[ ] \( \boxed{\sum_c P(A, B, C = c)} \): Marginalizing over $C$
            \item[ ] \( \sum_c P(A|C = c) P(B|C = c) \)
            \item[ ] \(\boxed{\sum_c P(A) P(B|A) P(C = c | A, B)} \): Writing out in terms of $\sum_c P(A, B, C = c)$
            \item[ ] \( \boxed{P(A) P(B)} \): $A$ and $B$ are independent.
            \item[ ] None of the above
        \end{itemize}

        \item[(b)] [1 pts] Let \( A, B, C \) be random variables with \( A \not\perp B \). Is it possible for \( A \perp B | C \)? 
        
        \begin{itemize}
            \item[ ] $\boxed{\text{Yes}}$: Not being indepedent doesn't imply conditional independence.
            \item[ ] $\text{No}$
        \end{itemize}

        \item[(c)] [1 pts] Let \( \mathcal{V} \) denote the set of variables in a Bayesian network. Suppose \( X \in \mathcal{V} \), and let \texttt{pts}(\( X \)), \texttt{chl}(\( X \)), \texttt{ans}(\( X \)), and \texttt{des}(\( X \)) represent the parents, children, ancestors, and descendants of \( X \). Provide a general independence rule of the form: 
        \[
        X \perp \mathcal{V} \setminus \text{des}(X) \; \mid \; \texttt{pts}(X)
        \]
        You may use the set operations, \( \cap, \cup \) and/or \( \setminus \) if necessary.

        \item[(d)] [1 pts] Assume \( A, B, C, \) and \( D \) are binary random variables, and \( E \) is a trinary random variable over \( \{1,2,3\} \). What is \( \sum_{A,B} P(A|B, E = 1) \)?

        \begin{itemize}
            \item[ ] $P(A=0 \mid B=0, E=1) + P(A=1 \mid B=0, E=1) = 1$ 
            \item[ ] $P(A=0 \mid B=1, E=1) + P(A=1 \mid B=1, E=1) = 1$
            \item[] Therefore, $2$.
        \end{itemize}
    \end{enumerate}
\end{example}

\newpage

\begin{center}
    \section*{L8: Code to Paper}
\end{center}
\subsection{Markov Chains (MCs)}

\subsubsection{Setup}
\begin{summary}
    In a \textbf{Markov Chain}, we assume that:
    \begin{itemize}
        \item there are no agents
        \item state transitions occur automatically
        \item $S_t$ is the state \textit{after} transition $t$
        \item the state transition process is stochastic and memoryless:
        \[
        S_t \perp S_0, \dots, S_{t-2} \mid S_{t-1}
        \]
        \begin{itemize}
            \item $S_t$ is independent of all previous states given $S_{t-1}$
        \end{itemize}
    \end{itemize}
    \vspace{1em}

    \begin{center}
        \begin{tabular}{ll}
            \toprule
            \textbf{Name} & \textbf{Function:} \\
            \midrule
            initial state distribution & $p_0(s) := \mathbb{P}[S_0 = s]$ \\
            \midrule
            transition distribution & $p(s'|s) := \mathbb{P}[S_{t+1} = s' | S_t = s]$ \\
            \midrule 
            Prob. that state of the env. after $T$ transitions is $s$ & $p_T(s) := \mathbb{P}[S_T = s]$ \\
            & $\quad \quad \; \; \; \;= \sum_{s'} p_{T-1}(s') p(s|s')$ \\
            \bottomrule            
        \end{tabular}
    \end{center}
\end{summary}

\subsubsection{Bayesian Network}
\begin{definition}
    $S_0,S_1,S_2,\ldots$ form a \textbf{Bayesian Network}:
    \customFigure[0.5]{../Images/L8_0.png}{}
\end{definition}
\newpage

\subsection{Markov Reward Processes (MRPs)}
\subsubsection{Setup}
\begin{summary}
    In a \textbf{Markov Reward Process}, we assume that:
    \begin{itemize}
        \item there is one agent
        \item state transitions occur automatically (i.e. agent has no control over actions)
        \item $S_t$ is the state \textit{after} transition $t$
        \item the state transition process is stochastic and memoryless:
        \[
        S_t \perp S_0, \dots, S_{t-2} \mid S_{t-1}
        \]
        \begin{itemize}
            \item $S_t$ is independent of all previous states given $S_{t-1}$
        \end{itemize}
        \item $R_t$ is the reward for transition $t$, i.e., $(S_{t-1}, \varnothing, S_t)$
    \end{itemize}
    \vspace{1em}

    \begin{center}
        \begin{tabular}{ll}
            \toprule
            \textbf{Name} & \textbf{Function:} \\
            \midrule
            Initial state distribution & $p_0(s) := \mathbb{P}[S_0 = s]$ \\
            \midrule
            Transition distribution & $p(s'|s) := \mathbb{P}[S_{t+1} = s' | S_t = s]$ \\
            \midrule
            Reward function & $r(s, s') := \text{reward for transition } (s, \varnothing, s')$ \\
            \midrule
            Discount factor & $\gamma \in [0,1]$ \\
            \midrule
            Return after $T$ transitions & $U_T = \sum_{t=1}^{T} \gamma^{t-1} R_t$ \\
            & $\quad \; \;= U_{T-1} + \gamma^{T-1} R_T$ \\
            \multicolumn{2}{p{\linewidth}}{
            \begin{itemize}
                \item The (possibly discounted) sum of the rewards after $T$ transitions
            \end{itemize}} \\
            \midrule
            Expected return after $T$ transitions & $\mathbb{E}[U_T] = \mathbb{E}[U_{T-1}] + \gamma^{T-1} \mathbb{E} [R_t]$ \\
            &  $\quad \quad \; \; \; = \mathbb{E}[U_{T-1}] + \gamma^{T-1} \sum_{s,s'} p_{T-1}(s) p(s'|s) r(s, s')$ \\
            \multicolumn{2}{p{\linewidth}}{
            \begin{itemize}
                \item $p_{T-1}(s) p(s'|s)$: Prob. $s \to s'$
                \item $r(s, s')$: rwd $s \to s'$
                \item $\mathbb{E}[U_0] := 0$.
            \end{itemize}} \\
            \midrule
            \bottomrule            
        \end{tabular}
    \end{center}
\end{summary}

\subsubsection{Bayesian Network}
\begin{definition}
    $S_0,R_1,S_1,R_2,S_2,\ldots$ form a \textbf{Bayesian Network}:
    \customFigure[0.5]{../Images/L8_1.png}{}
\end{definition}
\newpage

\subsection{Markov Decision Processes (MDPs)}
\subsubsection{Setup}
\begin{summary}
    In a \textbf{Markov Decision Process (MDP)}, we assume that:
    \begin{itemize}
        \item there is one agent
        \item state transitions occur manually (after each action)
        \item $S_t$ is the state \textit{after} transition $t$
        \item $A_t$ is the action inducing transition $t$
        \item the state transition process is stochastic and memoryless:
        \[
        S_t \perp S_0, A_1, \dots, S_{t-2}, A_{t-1} \mid S_{t-1}, A_t
        \]
        \begin{itemize}
            \item $S_t$ is independent of all previous states and actions given $S_{t-1}$ and $A_t$
        \end{itemize}
        \item $R_t$ is the reward for transition $t$, i.e., $(S_{t-1}, A_t, S_t)$
    \end{itemize}
\end{summary}
\newpage

\begin{summary}
    \begin{center}
        \begin{tabular}{ll}
            \toprule
            \textbf{Name} & \textbf{Function:} \\
            \midrule
            initial state distribution & $p_0(s) := \mathbb{P}[S_0 = s]$ \\
            \midrule
            transition distribution & $p(s'|s, a) := \mathbb{P}[S_t = s' | A_t = a, S_{t-1} = s]$ \\
            \midrule
            reward function & $r(s, a, s') := \text{reward for transition } (s, a, s')$ \\
            \midrule
            a time-invariant policy for choosing actions & $\pi(a|s) := \mathbb{P}[A_t = a | S_t = s]$ \\
            \midrule
            Maximum number of transitions & $T_{\max}$ \\
            \multicolumn{2}{p{\linewidth}}{
            \begin{itemize}
                \item A Markov Decision Process can be either:
                \begin{itemize}
                    \item \textbf{Finite}: $T_{\max}$ is finite
                    \item \textbf{Infinite}: $T_{\max}$ is infinite
                    \begin{itemize}
                        \item For infinite MDPs, we must have $\gamma < 1$.
                    \end{itemize}
                \end{itemize}
            \end{itemize}} \\
            \midrule
            Prob. that state of the env. after $T$ transitions is $s$ & $p_T(s) = \sum_{a,s'} p_{T-1}(s) \pi(a|s') p(s|s',a)$ \\
            \multicolumn{2}{p{\linewidth}}{
            \begin{itemize}
                \item $p_{T-1}(s)$: Prob. $s'$ at $T$-$1$
                \item $pi(a|s')$: Action $a$ from $s'$
                \item $p(s|s',a)$: Prob. $s$ given $s',a$
            \end{itemize}} \\
            \midrule
            Expected return after $T$ transitions & $\mathbb{E}_{\pi}[U_T] \text{=} \mathbb{E}_{\pi}[U_{T-1}] + \gamma^{T-1} \mathbb{E}_{\pi}[R_t]$ \\
            \multicolumn{2}{p{\linewidth}}{
            \begin{itemize}
                \item $\mathbb{E}_\pi [R_t] = \sum_{s,a,s'} p_{T-1}(s) \pi(a \mid s) p(s' \mid s, a) r(s, a, s')$ 
                \item $\mathbb{E}_\pi [U_0] = 0$.
            \end{itemize}} \\
            \midrule
            Future return after $\tau$ transitions & $G_\tau = \sum_{t = \tau + 1}^T \gamma^{t - (\tau + 1)} R_t$ \\
            & $\quad \; \;= R_{\tau + 1} + \gamma G_{\tau + 1}$ \\
            \multicolumn{2}{p{\linewidth}}{
            \begin{itemize}
                \item Starting at $\tau + 1$ for the future return. 
            \end{itemize}} \\
            \midrule
            Expected future return after $\tau$ transitions given $S_\tau \text{=} s$ & $\mathbb{E}_{\pi}[G_{\tau} \mid S_{\tau} \text{=} s] \text{=} \mathbb{E}_{\pi}[R_{\tau+1} \mid S_{\tau} \text{=} s] + \gamma \mathbb{E}_{\pi}[G_{\tau+1} \mid S_{\tau} \text{=} s]$ \\
            & $\text{=} \sum_{a, s'} \pi(a \mid s) p(s' \mid s, a) \left( r(s, a, s') + \gamma \mathbb{E}_{\pi}[G_{\tau+1} \mid S_{\tau+1} \text{=} s'] \right)$ \\
            \multicolumn{2}{p{\linewidth}}{
            \begin{itemize}
                \item $\mathbb{E}_{\pi}[G_{T_{\max}} \mid S_{T_{\max}} = s] = 0$.
                \item $\mathbb{E}_{\pi}[R_{\tau+1} \mid S_{\tau} = s] = \sum_{a, s'} \pi(a \mid s) p(s' \mid a, s) r(s, a, s')$
                \begin{itemize}
                    \item $\pi(a \mid s) p(s' \mid a, s)$: Prob. of getting to $s'$ from $s$ w/ action $a$
                    \item $r(s, a, s')$: Reward of getting to $s'$ from $s$ w/ action $a$
                \end{itemize}
                \item $\mathbb{E}_{\pi}[G_{\tau+1} \mid S_{\tau} = s] = \sum_{a, s'} \pi(a \mid s) p(s' \mid a, s) \mathbb{E}_{\pi}[G_{\tau+1} \mid S_{\tau+1} = s']$
                \begin{itemize}
                    \item $\pi(a \mid s) p(s' \mid a, s)$: Prob. of getting to $s'$ from $s$ w/ action $a$
                    \item $\mathbb{E}_{\pi}[G_{\tau+1} \mid S_{\tau+1} = s']$: Expected future return at $\tau + 1$ from $s'$ at $\tau + 1$.
                    \item $\sum_{a, s'}$: Sum over all possible future states and current actions to get expected future return at $\tau + 1$ from $s$ at $\tau$.
                \end{itemize}
            \end{itemize}} \\
            \bottomrule            
        \end{tabular}
    \end{center}
\end{summary}
\newpage

\begin{summary}
    \begin{center}
        \begin{tabular}{ll}
            \toprule
            \textbf{Name} & \textbf{Function:} \\
            \midrule
            Optimal policy & $\pi^*(\cdot \mid s) = \arg\max_{\pi(\cdot \mid s)} \mathbb{E}_{\pi} [G_{\tau} \mid S_{\tau} = s]$ \\
            \multicolumn{2}{p{\linewidth}}{
            \begin{itemize}
                \item Choose $\pi(\cdot \mid s)$ to maximize the expected future return given $S_{\tau} = s$.
            \end{itemize}} \\
            \midrule 
            Value function & $v_{\pi}(s, \tau) := \mathbb{E}_{\pi}[G_{\tau} \mid S_{\tau} = s]$ \\
            & $\quad \quad \quad \; \; \; = \sum_{a, s'} \pi(a \mid s) p(s' \mid s, a) \left( r(s, a, s') + \gamma v_{\pi}(s', \tau+1) \right)$ \\
            \multicolumn{2}{p{\linewidth}}{
            \begin{itemize}
                \item $v(s, T) = 0$ for all $s$.
            \end{itemize}} \\
            \midrule
            Optimal value function & $v^*(s, \tau) = \max_a \sum_{s'} p(s' \mid a, s) \left( r(s, a, s') + \gamma v^*(s', \tau+1) \right)$ \\
            \multicolumn{2}{p{\linewidth}}{
            \begin{itemize}
                \item Taking the optimal $\pi(\cdot \mid s)$ (i.e., $\pi^*$) gives the recurrence above.
                \item $v^*(s, T) = 0$ for all $s$.
            \end{itemize}} \\
            \midrule 
            Expected Return & $\mathbb{E}_\pi [U_{T_{\max}}] = \sum_s \mathbb{E}_\pi [G_0 \mid S_0 = s] p_0(s)$ \\
            & $\quad \quad \; \; \; = \sum_s v_\pi(s, 0) p_0(s)$ \\
            \multicolumn{2}{p{\linewidth}}{
            \begin{itemize}
                \item $G_0 = U_{T_{\max}}$
            \end{itemize}} \\
            \midrule 
            Optimal Expected Return & \\
            \bottomrule            
        \end{tabular}
    \end{center}
\end{summary}

\subsubsection{Bayesian Network}
\begin{definition}
    $S_0,A_1,R_1,S_1,A_2,R_2,S_2,\ldots$ form a \textbf{Bayesian Network}:
    \customFigure[0.5]{../Images/L8_2.png}{}
\end{definition}


\newpage

No order and unique elements
\begin{itemize}
    \item No order: Permutation invariance 
    \item Algorithm: Deep sets (MLP)
\end{itemize}

MLPs seperate between linear and non-linear, pooling

Symmetry in the data. 

MLPs are equivariant symmetry operations

Geometric DL Blueprint

What's special about image data? Rotation, translation, scaling. 

\begin{summary}
    \begin{itemize}
        \item Invariance: Enforce symmetry
        \item Equivariance: 
    \end{itemize}
\end{summary}


\begin{center}
    \section*{L10: Convolutional Neural Networks}
\end{center}

\begin{summary}
    \begin{itemize}
        \item Deep Set: MLP $\rightarrow$ Pool $\rightarrow$ MLP.
    \end{itemize}
\end{summary}

\begin{example}
    \textbf{Given:}
    \textbf{Soln.}
    \begin{enumerate}
        \item Define the function w/ 3 arguments.
        \item Set size: Colllection of data. 
    \end{enumerate}

\end{example}



\end{document}