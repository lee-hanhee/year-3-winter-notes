\documentclass[5pt]{extarticle} % Note the extarticle document class
\usepackage[margin=0.2in]{geometry} % Set 0.3-inch margins
\usepackage{multicol}              % For multi-column support
\usepackage{lipsum}                % Dummy text generator (optional)
\usepackage{amssymb} % math symbols
\setlength{\parskip}{0ex}
\setlength\parindent{0pt} % no indent
%% optional packages -- documentation at ctan.org
\usepackage{graphicx}  % image handline
\usepackage{amsmath}   % enhanced equation environments
\usepackage{tikz}      % block diagrams
\usetikzlibrary{positioning}  % allow relative positioning of tikz elements
\usepackage{pgfplots}  % package for plots, based on tikz
\usepackage{hyperref}
\usepackage{paracol}             % Import paracol package
\usepackage{float} % for H in figure

\definecolor{darkgreen}{rgb}{0.0, 0.5, 0.0} % Define the dark green color

\newcommand{\customFigure}[3][]{
    \vspace{-2em}
    \begin{figure}[H]
        \centering
        \includegraphics[width=#1\textwidth]{#2}
    \end{figure}
    \vspace{-2em}
}

\begin{document}

\begin{paracol}{3}
    {\tiny
    \textcolor{red}{\textbf{Notation:}} $P_{X \mid Y} (x \mid y) = P[X=x \mid Y=y]$ \\
    *Subscript indicates the RV, and the value indicates the realization.  

    \textcolor{red}{\textbf{Intro:}} \\
    \textcolor{blue}{\textbf{Random Experiment:}} An outcome for each run.  

    \textcolor{blue}{\textbf{Sample Space $\Omega$:}} Set of all possible outcomes.

    \textcolor{blue}{\textbf{Event:}} Measurable subsets of $\Omega$.

    \textcolor{blue}{\textbf{Prob. of Event A:}} $P(A) = \frac{\text{Number of outcomes in A}}{\text{Number of outcomes in } \Omega}$

    \textcolor{blue}{\textbf{Axioms:}} (1) $P(A) \geq 0 \; \forall A \in \Omega$, (2) $P(\Omega) = 1$, \\
    (3) If $A \cap B = \emptyset$, then $P(A \cup B) = P(A) + P(B) \; \forall A, B \in \Omega$

    \textcolor{blue}{\textbf{Cond. Prob.}} $P(A|B) = \frac{P(A \cap B)}{P(B)}$ \\
    *Prob. measured on new sample space $B$. \\ 
    *$P(A \cap B) = P(A|B) P(B) = P(B|A) P(A)$

    \textcolor{darkgreen}{\textbf{Independence:}} $P(A|B) = P(A) \Leftrightarrow P(A \cap B) = P(A) P(B)$

    \textcolor{blue}{\textbf{Total Prob. Thm:}} If $H_1, H_2, \ldots, H_n$ form a partition of $\Omega$, then $P(A) = \sum_{i=1}^n P(A|H_i) P(H_i)$.

    \textcolor{blue}{\textbf{Bayes' Rule:}} $P(H_k|A) \text{=} \frac{P(H_k \cap A)}{P(A)} \text{=} \frac{P(A|H_k) P(H_k)}{\sum_{i=1}^n P(A|H_i) P(H_i)}$ \\
    *Posteriori: $P(H_k|A)$, Likelihood: $P(A|H_k)$, Prior: $P(H_k)$ 

    \textcolor{red}{\textbf{1 RV:}} \\
    \textcolor{blue}{\textbf{Cumulative Distribution Fn (CDF):}} $F_X(x) = P[X \leq x]$ 

    \textcolor{blue}{\textbf{Prob. Mass Fn (PMF):}} $P_X(x_j) = P[X = x_j] \; j=1,2,\ldots$

    \textcolor{blue}{\textbf{Prob. Density Fn (PDF):}} $f_X(x) = \frac{d}{dx} F_X(x)$ \\
    *$P[a \leq X \leq b] = \int_a^b f_X(x) \, dx$ 

    \textcolor{blue}{\textbf{Exp.:}} $E[h(X)] \text{=} \int_{-\infty}^{\infty} h(x) f_X(x) dx$ \\
    $E[h(X)]= \sum_{k=-\infty}^{\infty} h(k) P_X(x_i \text{=} k)$ 

    \textcolor{darkgreen}{\textbf{Variance:}} $\sigma_X^2 = \text{Var}[X] = E[(X - E[X])^2] = E[X^2] - E[X]^2$

    \textcolor{darkgreen}{\textbf{Cond. Exp.:}} $E[X|A] = \int_{-\infty}^{\infty} x f_X(x|A) \, dx$ 

    \textcolor{red}{\textbf{2 RVs:}}  \\
    \textcolor{blue}{\textbf{Joint PMF:}} $P_{X,Y}(x, y) = P[X = x, Y = y]$

    \textcolor{blue}{\textbf{Joint PDF:}} $f_{X,Y}(x, y) = \frac{\partial^2}{\partial x \partial y} F_{X,Y}(x, y)$ \\
    *$P[(X, Y) \in A] = \int \int_{(x, y) \in A} f_{X,Y}(x, y) \, dx \, dy$

    \textcolor{blue}{\textbf{Exp.:}} $E[g(X, Y)] = \int_{-\infty}^{\infty} \int_{-\infty}^{\infty} g(x, y) f_{X,Y}(x, y) \, dx \, dy$

    \textcolor{darkgreen}{\textbf{Correlation:}} $E[XY]$

    \textcolor{darkgreen}{\textbf{Covar.:}} $\text{Cov}[X, Y] \text{=} E[(X - \mu_X)(Y - \mu_Y)] \text{=} E[XY] - E[X] E[Y]$

    \textcolor{darkgreen}{\textbf{Corr. Coeff.:}} $\rho_{X,Y} \text{=} E\left[ \left( \frac{X - \mu_X}{\sigma_X} \right) \left( \frac{Y - \mu_Y}{\sigma_Y} \right) \right] = \frac{\text{Cov}[X, Y]}{\sigma_X \sigma_Y}$ \\
    *$-1 \leq \rho_{X,Y} \leq 1$

    \textcolor{blue}{\textbf{Marginal PMF:}} $P_X(x) = \sum_{j=1}^\infty P_{X,Y}(x, y_j) \mid P_Y(y)$

    \textcolor{blue}{\textbf{Marginal PDF:}} $f_X(x) = \int_{-\infty}^{\infty} f_{X,Y}(x, y) \, dy \mid f_Y(y)$

    \textcolor{blue}{\textbf{Cond. PMF:}} $P_{X|Y}(x|y) = \frac{P_{X,Y}(x, y)}{P_Y(y)} \mid P_{Y|X}(y|x)$

    \textcolor{blue}{\textbf{Cond. PDF:}} $f_{X|Y}(x|y) = \frac{f_{X,Y}(x, y)}{f_Y(y)} \mid f_{Y|X}(y|x)$

    \textcolor{blue}{\textbf{Bayes' Rule}} \\ 
    $f_{Y|X}(y|x) \text{=} \frac{f_{X,Y} (x,y)}{f_X (x)} \text{=} \frac{f_{X|Y} (x|y) f_Y (y)}{\int_{-\infty}^{\infty} f_{X|Y} (x|y') f_Y (y') \, dy'}$ \\
    *$P_{Y|X}(y|x) = \frac{P_{X,Y}(x, y)}{P_X(x)} = \frac{P_{X|Y}(x|y) P_Y(y)}{\sum_{j=1}^\infty P_{X|Y}(x|y_j) P_Y(y_j)}$

    \textcolor{blue}{\textbf{Ind.:}} $f_{X|Y}(x|y) = f_X(x) \; \forall y \Leftrightarrow f_{X,Y}(x, y) = f_X(x) f_Y(y) $ 
    
    \textcolor{darkgreen}{\textbf{Thm:}} If independent, then uncorrelated unless Guassian.

    \textcolor{blue}{\textbf{Uncorrelated:}} $\text{Cov}[X, Y] = 0 \Leftrightarrow \rho_{X,Y} = 0$

    \textcolor{blue}{\textbf{Orthogonal:}} $E[XY] = 0$

    \textcolor{blue}{\textbf{Cond. Exp.:}} $E[Y] = E[E[Y|X]]$ or $E[E[h(Y)|X]]$ \\
    *$E[E[Y|X]]$ w.r.t. $X \mid E[Y|X]$ w.r.t. $Y$. 

    \textcolor{red}{\textbf{Estimation:}} Estimate unknown parameter $\theta$ from $n$ i.i.d. measurements $X_1, X_2, \ldots, X_n$, $\hat{\Theta}(\underline{X}) = g(X_1, X_2, \ldots, X_n)$

    \textcolor{blue}{\textbf{Estimation Error:}} $\hat{\Theta}(\underline{X}) - \theta$. 

    \textcolor{blue}{\textbf{Unbiased:}} $\hat{\Theta}(\underline{X})$ is unbiased if $E[\hat{\Theta}(\underline{X})] = \theta$. \\
    *\textbf{Asymptotically unbiased:} $\lim_{n \to \infty} E[\hat{\Theta}(\underline{X})] = \theta$.

    \textcolor{blue}{\textbf{Consistent:}} $\hat{\Theta}(\underline{X})$ is consistent if $\hat{\Theta}(\underline{X}) \rightarrow \theta$ as $n \to \infty$ or $\forall \epsilon >0, \lim_{n \to \infty} P[|\hat{\Theta}(\underline{X}) - \theta| < \epsilon] \rightarrow 1$.

    \textcolor{blue}{\textbf{Sufficient:}} A statistic is sufficient if the expression depends only on the statistic, it should be made up of $x_1, x_2, \ldots, x_n$.

    \textcolor{darkgreen}{\textbf{Sample Mean:}} $M_n = \frac{1}{n} S_n = \frac{1}{n} \sum_{i=1}^n X_i$. \\
    *Given a sequence of i.i.d. RVs, $X_1, X_2, \ldots, X_n$, $M_n$ is unbiased and consistent.

    \textcolor{darkgreen}{\textbf{Sample Variance:}} $S_n^2 = \frac{1}{n} \sum_{i=1}^n (X_i - M_n)^2$. \\
    *Given a sequence of i.i.d. RVs, $X_1, X_2, \ldots, X_n$, $S_n^2$ is biased and consistent. \\
    *Use $S_n^2 = \frac{1}{n-1} \sum_{i=1}^n (X_i - M_n)^2$ for unbiased.

    \textcolor{blue}{\textbf{Chebychev's Inequality:}} $P[|X - E[X]| \geq \epsilon] \leq \frac{\text{Var}[X]}{\epsilon^2}$ \\
    *$P[|X - E[X] | < \epsilon] \geq 1 - \frac{\text{Var}[X]}{\epsilon^2}$

    \textcolor{blue}{\textbf{Weak Law of Large \#s:}} $\lim_{n \to \infty} P[|M_n - \mu| < \epsilon] = 1 \; \forall \epsilon > 0$.

    \textcolor{blue}{\textbf{ML Estimation:}} Choose $\theta$ that is most likely to generate the obs. $x_1, x_2, \ldots, x_n$. \\
    *Disc: $\hat{\Theta} = \arg \max_\theta P_{\underline{X}} (\underline{x} | \theta) \overset{\text{log}}{\rightarrow} \hat{\theta} = \arg \max_\theta \sum_{i=1}^n \log P_X(x_i | \theta)$ \\
    *Cont: $\hat{\Theta} = \arg \max_\theta f_{\underline{X}} (\underline{x} | \theta) \overset{\text{log}}{\rightarrow} \hat{\theta} = \arg \max_\theta \sum_{i=1}^n \log f_X(x_i | \theta)$

    \textcolor{blue}{\textbf{Maximum A Posteriori (MAP) Estimation:}} \\
    *Disc: $\hat{\theta} = \arg \max_\theta P_{\Theta | \underline{X}} (\theta | \underline{x}) = \arg \max_\theta P_{\underline{X} | \Theta} (\underline{x} | \theta) P_\Theta (\theta)$\\
    *Cont: $\hat{\theta} = \arg \max_\theta f_{\Theta | \underline{X}} (\theta | \underline{x}) = \arg \max_\theta f_{\underline{X} | \Theta} (\underline{x} | \theta) f_\Theta (\theta)$ \\
    *$f_{\Theta | \underline{X}} (\theta | \underline{x})$: Posteriori, $f_{\underline{X} | \Theta} (\underline{x} | \theta)$: Likelihood, $f_\Theta (\theta)$: Prior

    \textcolor{darkgreen}{\textbf{Bayes' Rule:}} $P_{\Theta | \underline{X}} (\theta | \underline{x}) = \begin{cases}
        \frac{P_{\underline{X} | \Theta} (\underline{x} | \theta) P_\Theta (\theta)}{P_{\underline{X}} (\underline{x})} & \text{if } \underline{X} \text{ disc.} \\
        \frac{f_{\underline{X} | \Theta} (\underline{x} | \theta) P_\Theta (\theta)}{f_{\underline{X}} (\underline{x})} & \text{if } \underline{X} \text{ cont.}
    \end{cases}$ \\
    $f_{\Theta | \underline{X}} (\theta | \underline{x}) = \begin{cases}
        \frac{P_{\underline{X} | \Theta} (\underline{x} | \theta) f_\Theta (\theta)}{P_{\underline{X}} (\underline{x})} & \text{if } \underline{X} \text{ disc.} \\
        \frac{f_{\underline{X} | \Theta} (\underline{x} | \theta) f_\Theta (\theta)}{f_{\underline{X}} (\underline{x})} & \text{if } \underline{X} \text{ cont.} 
    \end{cases}$ \\
    *Independent of $\theta$: $f_{\underline{X}} (\underline{x}) = \int_{-\infty}^{\infty} f_{\underline{X} | \Theta} (\underline{x} | \theta) f_\Theta (\theta) \, d\theta$ \\

    \textcolor{blue}{\textbf{Least Mean Squares (LMS) Estimation:}} Assume prior $P_\Theta (\theta)$ or $f_\Theta (\theta)$ w/ obs. $\underline{X} = \underline{x}$. \\
    *$\hat{\theta} = g(\underline{x}) = \mathbb{E} [\Theta | \underline{X} = \underline{x}] \; \mid \; \hat{\Theta} = g(\underline{X}) = \mathbb{E} [\Theta | \underline{X}]$ \\

    \textcolor{blue}{\textbf{Conditional Exp.}} $E[X|Y] = \int_{-\infty}^{\infty} x f_{X|Y} (x|y) \, dx$

    \textcolor{red}{\textbf{Binary Hyp. Testing:}} $H_0$: Null Hyp., $H_1$: Alt. Hyp. 

    \textcolor{blue}{$\Omega_{\underline{X}}:$} Set of all possible obs. $\underline{x}$.

    \customFigure[0.11]{../Images/TB_1.png}
    
    \textcolor{darkgreen}{\textbf{TI Err. (False Rejection):}} Reject $H_0$ when $H_0$ is true. \\
    *$\alpha(R) = P[\underline{X} \in R \mid H_0]$ (false alarm)

    \textcolor{darkgreen}{\textbf{TII Err. (False Accept.):}} Accept $H_0$ when $H_1$ is true. \\
    *$\beta(R) = P[\underline{X} \in R^c \mid H_1]$ (missed detection)

    \textcolor{blue}{\textbf{Likelihood Ratio Test:}} $\forall \underline{x} \; L(\underline{x}) = \frac{P_{\underline{X}} (\underline{x}|H_1)}{P_{\underline{X}} (\underline{x}|H_0)} \overset{H_1}{\underset{H_0}{\gtrless}} 1 \text{ or } \xi$ \\
    *\textbf{Max. Likelihood Test:} $1$, \textbf{Likelihood Ratio Test:} $\xi$

    \textcolor{blue}{\textbf{Neyman-Pearson Lemma:}} Given a false rejection prob. ($\alpha$), the LRT offers the smallest possible false accept. prob. ($\beta$), and vice versa. \\
    *LRT produces ($\alpha,\beta$) pairs that lie on the efficient frontier.

    \customFigure[0.1]{../Images/L10_0.png}{}
    
    % Given $L(X), \xi$ so that \\ 
    % $P[L(X) > \xi \mid H_0] = \alpha$ and $P[L(X) \leq \xi \mid H_1] = \beta$,
    % then for any other test (rejection region) w/ $P[X \in R \mid H_0] \leq \alpha$, then $P[X \notin R \mid H_1] \geq \beta$.

    % \textcolor{darkgreen}{\textbf{Sig. Testing:}} Given $X_1, \ldots, X_n$, find a rejection reg. so a level of T1 err. is achieved: $P[\text{Reject } H_0 \mid H_0] = \alpha$. \\
    % *$\alpha$: Significance level, $1 - \alpha$: Confidence level.

    \textcolor{blue}{\textbf{Bayesian Hyp. Testing:}} \\
    \textcolor{darkgreen}{\textbf{MAP Rule:}} $L(\underline{x}) = \frac{p_{\underline{X}} (\underline{x} | H_1)}{p_{\underline{X}} (\underline{x} | H_0)} \overset{H_1}{\underset{H_0}{\gtrless}} \frac{P[H_0]}{P[H_1]}$ 
    
    % Selects hyp. w/ higher a posteriori prob, reject $H_0$ if: \\
    % $p(H_1 \mid \underline{x}) \overset{H_1}{\underset{H_0}{\gtrless}} p(H_0 \mid \underline{x}) \; \mid \; f(H_1 \mid \underline{x}) \overset{H_1}{\underset{H_0}{\gtrless}} f(H_0 \mid \underline{x})$ \\
    % $p_{\underline{X}} (\underline{x} | H_1) P[H_1] \overset{H_1}{\underset{H_0}{\gtrless}} p_{\underline{X}} (\underline{x} | H_0) P[H_0]$ \\
    
    % $p(\underline{x} \mid H_1)\pi_j \overset{H_1}{\underset{H_0}{\gtrless}} p(\underline{x} \mid H_0)\pi_0 \; \mid \; f(\underline{x} \mid H_1)\pi_j \overset{H_1}{\underset{H_0}{\gtrless}} f(\underline{x} \mid H_0)\,\pi_0$ \\
    % *$p(H_j \mid \underline{x}) \text{=} \frac{p_{\underline{X}} (\underline{x} \mid H_j) P [H_j]}{p_{\underline{X}} (\underline{x} \mid H_0) P[H_0] + p_{\underline{X}} (\underline{x} \mid H_1) P[H_1]}$: A posteriori 

    \textcolor{orange}{\textbf{Gaussian to Q Fcn:}} 1. Find $Q(x) = \frac{1}{\sqrt{2\pi}} \int_x^\infty e^{-t^2/2} \, dt$. \\
    2. Use table to find $Q(x)$ for $x \geq 0$. 

    \textcolor{darkgreen}{\textbf{Min. Cost Bayes' Dec. Rule:}} $C_{ij}$ is cost of choosing $H_j$ when $H_i$ is true. Given obs. $\underline{X} = \underline{x}$, the expected cost of choosing $H_j$ is $A_j (\underline{x}) = \sum_{i=0}^1 C_{ij} \, P[H_i | \underline{X} = \underline{x}]$. 
    
    % $\text{Min. Cost Detection} = \sum_{i=0}^1 \sum_{j=0}^1 C_{ij} \, P[\text{decide } j \mid H_i] \, \pi_i$ \\
    % *$j=0$: Accept $H_0$, $j=1$: Reject $H_0$

    \textcolor{darkgreen}{\textbf{Min. Cost Dec. Rule:}} 
    $L(\underline{x}) = \frac{P_{\underline{X}} (\underline{x} \mid H_1)}{P_{\underline{X}} (\underline{x} \mid H_0)}\overset{H_1}{\underset{H_0}{\gtrless}} \frac{(C_{01} - C_{00}) P[H_0]}{(C_{10} - C_{11})P[H_1]}$. \\
    *$C_{01}$: False accept. cost, $C_{10}$: False reject. cost.

    \textcolor{blue}{\textbf{Naive Bayes Assumption:}} Assume $X_1 \ldots, X_n$ (features) are ind., then $p_{\underline{X} \mid \Theta} (\underline{x} \mid \theta) = \Pi_{i=1}^n p_{X_i \mid \Theta} (x_i \mid \theta)$.

    \textbf{Notation:} $P_{\underline{X} | \Theta} (\underline{x} | \theta)$, only put RVs in subscript, not values. $P_{\underline{X}} (\underline{x} | H_i)$, didn't put $H$ in subscript b/c it's not a RV.

    \textbf{Binomial} \# of successes in $n$ trials, each w/ prob. $p$ \\
    $ b(x \mid n, p) = \binom{n}{x} p^x (1 - p)^{n - x}, x = 0, 1, 2, \dots $ \\
    *$ E[X] = \mu = np \; \mid \; Var(X) = \sigma^2 = np(1 - p) $

    \textbf{Multinomial} \# of $x_i$ successes in $n$ trials, each w/ prob. $p_i$ \\
    $ f(x_i \mid p_i \forall i, n) = \frac{n!}{x_1! \dots x_m!} p_1^{x_1} \dots p_m^{x_m} $  \\
    *$ \sum_{i} x_i = n $, and $ \sum_{i=1}^{m} p_i = 1 $ \\
    *$ E[X_i] = \mu_i = np_i \; \mid \; Var(X_i) = \sigma^2_i = np_i(1 - p_i) $

    \textbf{Hypergeometric} \# of successes in $n$ draws from $N$ items, $k$ of which are successes \\
    $ h(x \mid N, n, k) = \frac{\binom{k}{x} \binom{N-k}{n-x}}{\binom{N}{n}}$ \\
    *$\max\{0, n - (N - k)\} \leq x \leq \min\{n, k\} $ \\
    *$E[X] = \mu = \frac{nk}{N} \; \mid \; Var(X) = \sigma^2 = \frac{N-n}{N-1} \cdot n \cdot \frac{k}{N} \cdot \left(1 - \frac{k}{N} \right) $

    \textbf{Negative Binomial} \# of trials until $k$ successes, each w/ prob. $p$ \\
    $ b^*(x \mid k, p) = \binom{x-1}{k-1} p^k (1 - p)^{x - k}$ \\
    *$x \geq k, x = k, k+1, \dots $ \\
    *$ E[X] = \mu = \frac{k}{p} \; \mid \; Var(X) = \sigma^2 = \frac{k(1 - p)}{p^2} $

    \textbf{Geometric} \# of trials until 1st success, each w/ prob. $p$ \\
    $ g(x \mid p) = p(1 - p)^{x - 1}$ \\
    *$x \geq 1, x = 1, 2, 3, \dots $ \\
    *$ E[X] = \mu = \frac{1}{p} \; \mid \; Var(X) = \sigma^2 = \frac{1 - p}{p^2} $

    \textbf{Poisson} \# of events in a fixed interval w/ rate $\lambda$ \\
    $ p(x \mid \lambda t) = \frac{e^{-\lambda t} (\lambda t)^x}{x!}$ \\
    *$x \geq 0, x = 0, 1, 2, \dots $ \\
    *$ E[X] = \mu = \lambda t \; \mid \; Var(X) = \sigma^2 = \lambda t $

    \textcolor{blue}{\textbf{Beta Prior}} $\Theta$ is a Beta R.V. w/ $\alpha,\beta>0$\\
    $f_\Theta (\theta) = \begin{cases}
        \frac{\Gamma(\alpha + \beta)}{\Gamma(\alpha) \Gamma(\beta)} \theta^{\alpha - 1} (1 - \theta)^{\beta - 1} & \text{if } 0 < \theta < 1 \\
        0 & \text{otherwise}
    \end{cases}$ \\
    *$\Gamma(x) = \int_{0}^{\infty} t^{x-1} e^{-t} \, dt$

    \textcolor{darkgreen}{\textbf{Prop.:}} 1. $\Gamma(x+1) = x \Gamma(x)$. For $m \in \mathbb{Z}^+$, $\Gamma(m+1) = m!$. \\
    2. $\beta(\alpha,\beta) = \frac{(\alpha + \beta -1)!}{(\alpha - 1)! (\beta - 1)!} = \beta \binom{\alpha + \beta - 1}{\alpha - 1}$ \\
    3. Expected Value: $E[\Theta] = \frac{\alpha}{\alpha + \beta}$ for $\alpha, \beta > 0$ \\
    4. Mode (max of PDF): $\frac{\alpha - 1}{\alpha + \beta - 2}$ for $\alpha, \beta > 1$ \\

    \textcolor{orange}{\textbf{Drawing Beta Dist.}} 1. Label $x$-axis from 0 to 1. 2. Identify mode. \\ 
    3. Determine shape based on $\alpha$ and $\beta$: $\alpha = \beta = 1$ (uniform), $\alpha = \beta > 1$ (bell-shaped, peak at 0.5), $\alpha = \beta < 1$ (U-shaped w/ high density near 0 and 1), $\alpha > \beta$ (left-skewed), $\alpha < \beta$ (right-skewed).

    \textcolor{blue}{\textbf{Uniform PDF}} $f_X(x) = \begin{cases}
        \frac{1}{b - a} & \text{if } a \leq x \leq b \\
        0 & \text{otherwise}
    \end{cases}$ \\
    *$E[X] = \frac{a + b}{2}$, $\text{Var}[X] = \frac{(b - a)^2}{12}$

    \textcolor{red}{\textbf{Random Vector:}} $\underline{X} = (X_1,\ldots,X_n) \text{=} \begin{bmatrix} X_1 \\ \vdots \\ X_n \end{bmatrix} \text{=} \begin{bmatrix} X_1 & \cdots & X_n \end{bmatrix}^T$

    \textcolor{blue}{\textbf{Mean Vector:}} $\underline{m}_{\underline{X}} = E[\underline{X}] = [\mu_1, \ldots, \mu_n]^T$ \\

    \textcolor{blue}{\textbf{Corr. Mat.:}} $R_{\underline{X}} =
    \begin{bmatrix}
    E[X_1^2]  & \cdots & E[X_1 X_n] \\
    E[X_2 X_1] & \cdots & E[X_2 X_n] \\
    \vdots & \ddots & \vdots \\
    E[X_n X_1] & \cdots & E[X_n^2]
    \end{bmatrix}$ \\
    *Real, symmetric ($R=R^T$), and PSD ($\forall \underline{a}, \underline{a}^T R \underline{a} \geq 0$).

    \textcolor{blue}{\textbf{Covar. Mat.:}} $K_{\underline{X}} =
    \begin{bmatrix}
    \operatorname{Var}[X_1] & \cdots & \operatorname{Cov}[X_1, X_n] \\
    \operatorname{Cov}[X_2, X_1] & \cdots & \operatorname{Cov}[X_2, X_n] \\
    \vdots & \ddots & \vdots \\
    \operatorname{Cov}[X_n, X_1] & \cdots & \operatorname{Var}[X_n]
    \end{bmatrix}$ \\
    *$K_{\underline{X}} = R_{\underline{X} - \underline{m}_{\underline{X}}} = R_{\underline{X}} - \underline{m} \underline{m}^T$ \\
    *Diagonal $K_{\underline{X}} \iff X_1, \ldots, X_n$ are (mutually) uncorrelated.

    \textcolor{red}{\textbf{Lin. Trans.}} $\underline{Y} = A \underline{X}$ (A rotates and stretches $\underline{X}$)

    \textcolor{darkgreen}{\textbf{Mean:}} $E[\underline{Y}] = A \underline{m}_{\underline{X}}$

    \textcolor{darkgreen}{\textbf{Covar. Mat.:}} $K_{\underline{Y}} = A K_{\underline{X}} A^T$

    \textcolor{blue}{\textbf{Diagonalization of Covar. Mat. (Uncorrelated):}} \\
    $\forall \underline{X}$, set $P = [\underline{e}_1, \ldots, \underline{e}_n]$ of $K_{\underline{X}}$, if $\underline{Y} = P^T \underline{X}$, then \\
    $K_{\underline{Y}} = P^T K_{\underline{X}} P = \Lambda$ \\
    *$\underline{Y}$: Uncorrelated RVs, $K_{\underline{X}} = P \Lambda P^T$

    \textcolor{orange}{\textbf{Find an Uncorrelated $K_{\underline{Y}}$}} \\
    1. Find eigenvalues, normalized eigenvectors of $K_{\underline{X}}$. \\
    2. Set $K_{\underline{Y}} = \Lambda$, where $\underline{Y} = P^T \underline{X}$

    \textcolor{blue}{\textbf{PDF of L.T.}} If $\underline{Y} = A \underline{X}$ w/ $A$ not singular, then \\
    $f_{\underline{Y}} (\underline{y}) = \frac{f_{\underline{X}} (\underline{x})}{|\det A|} \Big|_{\underline{x} = A^{-1} \underline{y}}$

    \textcolor{orange}{\textbf{Find $f_{\underline{Y}} (\underline{y})$}} 1. Given $f_{\underline{X}} (\underline{x})$ and RV relations, define $A$ \\
    2. Determine $|\det A|$, $A^{-1}$, then $f_{\underline{Y}} (\underline{y})$.

    \textcolor{red}{\textbf{Gaussian RVs:}} $\underline{X} \sim \mathcal{N} (\underline{\mu}, \Sigma)$ \\ 
    PDF of jointly Gaus. $X_1, \ldots, X_n$ $\equiv$ Guas. vector: \\
    $f_{\underline{X}} (\underline{x}) = \frac{1}{(2\pi)^{n/2} |\det \Sigma|^{1/2}} e^{-\frac{1}{2} (\underline{x} - \underline{\mu})^T \Sigma^{-1} (\underline{x} - \underline{\mu})}$ \\ 
    *1D: $f_X(x) = \frac{1}{\sqrt{2\pi} \sigma} e^{-\frac{1}{2} \left( \frac{x - \mu}{\sigma} \right)^2}$ \\
    *$\underline{\mu} = \underline{m}_{\underline{X}}$, $\Sigma = K_{\underline{X}}$ ($\Sigma$ not singular) \\
    *Indep.: $f_{\underline{X}} (\underline{x}) = \frac{1}{(2\pi)^{n/2} \prod_{i=1}^n \sigma_i} e^{-\frac{1}{2} \sum_{i=1}^n \left( \frac{x_i - \mu_i}{\sigma_i} \right)^2}$ \\
    *IID: $f_{\underline{X}} (\underline{x}) = \frac{1}{(2\pi)^{n/2} \sigma^n} e^{-\frac{1}{2\sigma^2} \sum_{i=1}^n (x_i - \mu)^2}$ \\
    *Cond. PDF: $f_{\underline{X} | \underline{Y}} (\underline{x} | \underline{y}) = \mathcal{N} (\underline{\mu}_{\underline{X} | \underline{Y}}, \Sigma_{\underline{X} | \underline{Y}})$

    \textcolor{blue}{\textbf{Properties of Guassian Vector:}} \\
    1. PDF is completely determined by $\underline{\mu}$, $\Sigma$. \\
    2. $\underline{X}$ uncorrelated $\iff$ $\underline{X}$ independent. \\
    3. Any L.T. $\underline{Y} = A \underline{X}$ is Gaus. vector w/ $\underline{\mu}_{\underline{Y}} \text{=} A \underline{\mu}_{\underline{X}}$, $\Sigma_{\underline{Y}} \text{=} A \Sigma_{\underline{X}} A^T$. \\
    4. Any subset of $\{X_i\}$ are jointly Gaus. \\
    5. Any cond. PDF of a subset of $\{X_i\}$ given the other elements is Gaus. 

    \textcolor{blue}{\textbf{Diagonalization of Guassian Covar. (Indep.)}} \\ 
    $\forall \underline{X}$, set $P = [\underline{e}_1, \ldots, \underline{e}_n]$ of $\Sigma_{\underline{X}}$, if $\underline{Y} = P^T \underline{X}$, then \\
    $\Sigma_{\underline{Y}} = P^T \Sigma_{\underline{X}} P = \Lambda$ \\
    *$\underline{Y}$: Indep. Gaussian RVs, $\Sigma_{\underline{X}} = P \Lambda P^T$

    \textcolor{orange}{\textbf{How to go from $Y$ to $X$?}} 1. Given, $\underline{X} \sim \mathcal{N}(\underline{\mu}, \Sigma)$ \\
    2. $\underline{V} \sim \mathcal{N} (\underline{0}, I)$ 3. $\underline{W} = \sqrt{\Lambda} \underline{V}$ 4. $\underline{Y} = P \underline{W}$ 4. $\underline{X} = \underline{Y} + \underline{\mu}$

    \textcolor{red}{\textbf{Guassian Discriminant Analysis:}} \\
    Obs: $\underline{X} = \underline{x} = (x_1,\ldots,x_D)$ \\
    Hyp: $\underline{x}$ is generated by $\mathcal{N} (\underline{\mu}_c, \Sigma_c), c \in C$ \\
    Dec: Which "Guassian bump" generated $\underline{x}$? \\
    Prior: $P[C = c] = \pi_c$  (Gaussian Mixture Model) 

    \textcolor{blue}{\textbf{MAP:}} $\hat{c} = \arg \max_c P_C[c | \underline{X} = \underline{x}] = \arg \max_c f_{\underline{X} \mid C} (\underline{x} \mid c) \pi_c$ 

    \textcolor{blue}{\textbf{LGD:}} Given $\Sigma_c = \Sigma \; \forall c$, find $c$ w/ best $\underline{\mu}_c$ \\
    $\hat{c} = \arg \max_c \underline{\beta}_c^T \underline{x} + \gamma_c$ \\
    *$\underline{\beta}_c^T = \underline{\mu}_c^T \Sigma^{-1} \; \mid \; \gamma_c = \log \pi_c - \frac{1}{2} \underline{\mu}_c^T \Sigma^{-1} \underline{\mu}_c$ \\

    \textcolor{darkgreen}{\textbf{Bin. Hyp. Decision Boundary}} $\underline{\beta}_0^T \underline{x} + \gamma_0 = \underline{\beta}_1^T \underline{x} + \gamma_1$ \\
    *Linear in space of $\underline{x}$

    \textcolor{blue}{\textbf{QGD:}} Given $\Sigma_c$ are diff., find $c$ w/ best $\underline{\mu}_c$, $\Sigma_c$ \\
    $\hat{c} = \arg \max_c -\frac{1}{2} \log |\Sigma_c| - \frac{1}{2} (\underline{x} - \underline{\mu}_c)^T \Sigma_c^{-1} (\underline{x} - \underline{\mu}_c) + \log \pi_c$ \

    \textcolor{darkgreen}{\textbf{Bin. Hyp. Decision Boundary}} Quadratic in space of $\underline{x}$

    \textcolor{orange}{\textbf{How to find $\underline{\pi}_c, \underline{\mu}_c, \Sigma_c$:}} Given $n$ points gen. by GMM, then $n_c$ points $\{\underline{x}_1^c, \ldots, \underline{x}_{n_c}^c\}$ come from $\mathcal{N} (\underline{\mu}_c, \Sigma_c)$ \\
    $\hat{\pi}_c = \frac{n_c}{n}$ (categorical RV) \\
    $\hat{\mu}_c = \frac{1}{n_c} \sum_{i=1}^{n} \underline{x}_i^c$, (sample mean)\\
    $\Sigma_c = \frac{1}{n_c} \sum_{i=1}^{n_c} (x_i^c - \hat{\mu}_c)(x_i^c - \hat{\mu}_c)^T$ (biased sampled var.)

    \textcolor{red}{\textbf{Guassian Estimation:}} \\
    \textcolor{blue}{\textbf{MAP Estimator for $\underline{X}$ Given $\underline{Y}$ When $\underline{W} \text{=} (\underline{X},\underline{Y}) \sim \mathcal{N} (\underline{\mu}, \Sigma)$}} \\
    Given $\underline{X} = \{X_1,\ldots,X_n\}$, $\underline{Y} = \{Y_1,\ldots,Y_m\}$ \\ 
    $\hat{\underline{x}}_{\text{MAP}}(\underline{y}) = \hat{\underline{x}}_{\text{LMS}}(\underline{y}) = \underline{\mu}_{\underline{X} \mid \underline{Y}}= \underline{\mu}_{\underline{X}} + \Sigma_{\underline{X} \underline{Y}} \Sigma_{\underline{Y} \underline{Y}}^{-1} (\underline{y} - \underline{\mu}_Y)$ \\
    *$\hat{\underline{x}}_{\text{MAP/LMS}}$: Linear fcn of $\underline{y}$ 

    \textcolor{darkgreen}{\textbf{Covar. Matrices:}} $\Sigma = \begin{bmatrix} \Sigma_{\underline{XX}} & \Sigma_{\underline{X} \underline{Y}} \\ \Sigma_{\underline{Y} \underline{X}} & \Sigma_{\underline{YY}} \end{bmatrix}$ \\
    *$\Sigma_{\underline{X} \underline{X}} = \Sigma_{\underline{X}} = E\left[(\underline{X} - \underline{\mu}_{\underline{X}})(\underline{X} - \underline{\mu}_{\underline{X}})^T\right] \mid \Sigma_{\underline{Y} \underline{Y}} = \Sigma_{\underline{Y}}$ \\ 
    *$\Sigma_{\underline{X} \underline{Y}} = E\left[(\underline{X} - \underline{\mu}_{\underline{X}})(\underline{Y} - \underline{\mu}_{\underline{Y}})^T\right] \mid \Sigma_{\underline{Y} \underline{X}} = \Sigma_{\underline{X} \underline{Y}}^T$ 

    \textcolor{darkgreen}{\textbf{Prec. Matrices:}} $\Lambda = \Sigma^{-1}$ 

    \textcolor{darkgreen}{\textbf{Mean and Covar. Mat. of $\underline{X}$ Given $\underline{Y}$:}} \\
    *$\underline{\mu}_{\underline{X} \mid \underline{Y}} = \underline{\mu}_{\underline{X}} + \Sigma_{\underline{X} \underline{Y}} \Sigma_{\underline{Y} \underline{Y}}^{-1} (\underline{y} - \underline{\mu}_Y)$ \\
    *$\Sigma_{\underline{X} \mid \underline{Y}} = \Sigma_{\underline{X}} - \Sigma_{\underline{X} \underline{Y}} \Sigma_{\underline{Y} \underline{Y}}^{-1} \Sigma_{\underline{Y} \underline{X}}$ \\ 
    *\textbf{Reducing Uncertainty:} 2nd term is PSD, so given $\underline{Y} = \underline{y}$, always reducing uncertainty in $\underline{X}$.
    
    \textcolor{darkgreen}{\textbf{ML Estimator for $\theta$ w/ Indep. Guas:}}  \\
    Given $\underline{X} \text{=} \{X_1,\ldots,X_n\}$: $\hat{\theta}_{\text{ML}} \text{=} \frac{\sum_{i=1}^n \frac{x_i}{\sigma_i^2}}{\sum_{i=1}^n \frac{1}{\sigma_i^2}}$ (weighted avg. $\underline{x}$) \\
    *$X_i \text{=} \theta + Z_i$: Measurement $\mid$ $Z_i \sim \mathcal{N}(0, \sigma_i^2)$: Noise (indep.) \\
    *$\frac{1}{\sigma_i^2}$: Precision of $X_i$ (i.e. weight) \\
    *Larger $\sigma_i^2 \implies$ less weight on $X_i$ (less reliable measurement) \\
    *\textbf{SC:} If $\sigma_i^2 = \sigma^2 \; \forall i$ (iid), then $\hat{\theta}_{\text{ML}} = \frac{1}{n} \sum_{i=1}^n x_i$.

    \textcolor{darkgreen}{\textbf{MAP Estimator for $\theta$ w/ Indep. Gaus., Gaus. Prior:}} \\
    Given $\underline{X} \text{=} \{X_1,\ldots,X_n\}$, prior $\Theta \sim \mathcal{N} (x_0, \sigma_0^2)$ \\
    $\hat{\theta}_{\text{MAP}} = \frac{\sum_{i=0}^n \frac{x_i}{\sigma_i^2}}{\sum_{i=1}^n \frac{1}{\sigma_i^2}} = \frac{\frac{1}{\sigma_0^2}}{\frac{1}{\sigma_0^2} + \sum_{i=1}^n \frac{1}{\sigma_i^2}} x_0 + \frac{\sum_{i=1}^n \frac{1}{\sigma_i^2}}{\frac{1}{\sigma_0^2} + \sum_{i=1}^n \frac{1}{\sigma_i^2}} \hat{\theta}_{\text{ML}}$ \\
    *$X_i \text{=} \theta + Z_i$: Measurement $\mid$ $Z_i \sim \mathcal{N}(0, \sigma_i^2)$: Noise (indep.) \\
    *$f_\Theta$: Gaussian prior $\equiv$ prior meas. $x_0$ w/ $\sigma_0^2$. \\
    *\textbf{SC:} As $n \rightarrow \infty$, $\hat{\theta}_{\text{MAP}} \rightarrow \hat{\theta}_{\text{ML}}$. As $\sigma_0^2 \rightarrow \infty$, $\hat{\theta}_{\text{MAP}} \rightarrow \hat{\theta}_{\text{ML}}$ 

    \textcolor{darkgreen}{\textbf{LMMSE Estimator for $\underline{X}$ Given $\underline{Y}$ w/ non-Guas. $\underline{X}$, $\underline{Y}:$}} \\
    $\hat{\underline{x}}_{\text{LMMSE}}(\underline{y}) = \underline{\mu}_{\underline{X}} + \Sigma_{\underline{X} \underline{Y}} \Sigma_{\underline{Y} \underline{Y}}^{-1} (\underline{y} - \underline{\mu}_Y)$  

    \textcolor{red}{\textbf{Linear Guassian System:}} Given $\underline{Y} = A \underline{X} + \underline{b} + \underline{Z}$ \\
    *$\underline{X} \sim \mathcal{N} (\underline{\mu}_{\underline{X}}, \Sigma_{\underline{X}})$, $\underline{Z} \sim \mathcal{N} (\underline{0}, \Sigma_{\underline{Z}})$: Noise (indep. of $\underline{x}$) \\
    *$A\underline{X} + \underline{b}$: channel distortion, $\underline{Y}$: Observed sig.
    
    \textcolor{blue}{\textbf{MAP/LMS Estimator for $\underline{X}$ Given $\underline{Y}$ w/ $\underline{W}=(\underline{X},\underline{Y})$}} \\
    Given $\underline{W} = \begin{bmatrix}
    \underline{X} \\
    A \underline{X} + \underline{b} + \underline{Z}
    \end{bmatrix} = \begin{bmatrix}
    I & 0 \\
    A & I
    \end{bmatrix} \begin{bmatrix}
    \underline{X} \\
    \underline{Z}
    \end{bmatrix} + \begin{bmatrix}
    \underline{0} \\
    \underline{b}
    \end{bmatrix}$ \\

    $\hat{x}_{\text{MAP/LMS}} \text{=} \underline{\mu}_{\underline{X}} + \Sigma_{\underline{X}} A^T (A \Sigma_{\underline{X}} A^T \text{+} \Sigma_{\underline{Z}})^{-1} (\underline{y} - A \underline{\mu}_{\underline{X}} - \underline{b})$ \\
    *$\Sigma_{\underline{X} \underline{Y}} = \Sigma_{\underline{X}} A^T$, $\Sigma_{\underline{Y} \underline{Y}} = A \Sigma_{\underline{X}} A^T + \Sigma_{\underline{Z}}$ \\
    $\hat{x}_{\text{MAP/LMS}} \text{=} \left(\Sigma_{\underline{X}}^{-1} + A^T \Sigma_{\underline{Z}}^{-1} A \right)^{-1} \left(A^T \Sigma_{\underline{Z}}^{-1} (\underline{y} - \underline{b}) + \Sigma_{\underline{X}}^{-1} \underline{\mu}_{\underline{X}} \right)$ 
    *\textbf{Use:} Good to use when $\underline{Z}$ is indep. 

    \textcolor{darkgreen}{\textbf{Covar. Mat of $\underline{X}$ Given $\underline{Y}=\underline{y}$:}} $\Sigma_{\underline{X} \mid \underline{y}} = \left(\Sigma_{\underline{X}}^{-1} + A^T \Sigma_{\underline{Z}}^{-1} A \right)^{-1}$

    \textcolor{red}{\textbf{Linear Regression:}} Estimate unknown target fn $Y=g(\underline{X})$ w/ iid obs. $\{(\underline{x}_1, y_1), \ldots, (\underline{x}_n, y_n)\}$ (MLE/MAP) \\ 
    *$\underline{y} = \begin{bmatrix} y_1 & \cdots & y_n \end{bmatrix}^T$ \\
    *$X = \begin{bmatrix}  \underline{x}_1^T \\ \vdots \\ \underline{x}_n^T \end{bmatrix} \in \mathbb{R}^{n \times D}$ 

    \textcolor{blue}{\textbf{ML Estimator:}} $Y = h(\underline{x}) = \underline{w}^T \underline{x} + Z \approx g(\underline{X})$, then $\hat{\underline{w}}_{\text{ML}} = (X X^T)^{-1} X^T \underline{y}$ \\
    *Assume $X^T X$ has full rank (i.e. invertible) since $n \gg D$ \\
    *$n$: \# of obs., $D$: \# of features. \\ 
    *$\underline{x}  = \{x_1, \ldots, x_D\}$: Input features \\
    *$\underline{w} = \{w_1, \ldots, w_D\}$: Weights (parameter) \\
    *$Z\sim \mathcal{N}(0, \sigma^2)$: Noise (i.i.d.) \\
    *$Y$: Target/observed output \\
    *$X^\dagger = (X^T X)^{-1} X^T$: Pseudo-inverse of $X$ (minimizes $||X \underline{w} - \underline{y}||_2^2 \iff$ maximizes the likelihood of training data) 

    \textcolor{green}{\textbf{Non-Linear Trans:}} $\hat{y} = \underline{w}^T \underline{\phi}(\underline{x}) + Z$ w/ same assumptions, then $\hat{\underline{w}}_{\text{ML}} = (X X^T)^{-1} X^T \underline{y}$ \\
    *$\underline{\phi}(\underline{x})$: Non-linear transformation of $\underline{x}$ \\
    -E.g. of $1$ dim $x$: $\underline{\phi}(x) = \begin{bmatrix} 1 \\ x \\ x^2 \\ \vdots \\ x^M \end{bmatrix}$: Polynomial regression \\ 
    *$M$: Degree of polynomial, $D=1+M$: \# of features. \\
    *$X = \begin{bmatrix}  \underline{\phi}(\underline{x}_1)^T \\ \vdots \\ \underline{\phi}(\underline{x}_n)^T \end{bmatrix} \in \mathbb{R}^{n \times D}$
    
    \textcolor{green}{\textbf{Underfitting vs. Overfitting:}} \\
    *Underfitting: Model too simple, high bias, low variance. \\
    -Results in high train/test error. \\
    *Overfitting: Model too complex, low bias, high variance. \\
    -Results in low train error, high test error. 

    \textcolor{blue}{\textbf{MAP Estimator (Bayesian Linear Regression):}} Assume prior $w_i\sim \mathcal{N}(0, \tau^2)$ (i.i.d.) and 
     $\hat{y} = \underline{w}^T \underline{x} + Z$, then \\
    $\hat{\underline{w}}_{\text{MAP}} = (X^T X + \lambda I)^{-1} X^T \underline{y}$ \\
    *$\lambda = \frac{\sigma^2}{\tau^2}$: Regularization parameter \\
    *$X$: Can be linear or non-linear transformation of $\underline{x}$ \\
    *$\underline{x}  = \{x_1, \ldots, x_D\}$: Input features \\
    *$\underline{w} = \{w_1, \ldots, w_D\}$: Weights (parameter) \\
    *$Z\sim \mathcal{N}(0, \sigma^2)$: Noise (i.i.d.) \\
    *$Y$: Target/observed output 

    \textcolor{green}{\textbf{Notes:}} \\
    1. Useful when training data set size is small i.e. $n \ll D$. \\
    2. Regularization: Prevents overfitting by penalizing large weights. \\
    *$\tau = \infty \implies \lambda = 0$: No regularization so $\hat{\underline{w}}_{\text{MAP}} = \hat{\underline{w}}_{\text{ML}}$ \\
    *$\tau = 0 \implies \lambda = \infty$: Infinite regularization so $\hat{\underline{w}}_{\text{MAP}} = \underline{0}$
    *$\tau \downarrow \implies \lambda \uparrow$: More regularization, simpler model. \\
    *$\tau \uparrow \implies \lambda \downarrow$: Less regularization, more complex model.

    \textcolor{blue}{\textbf{Guassian Linear System}} Given training data $\underline{Y} = \underline{X} \underline{w} + \underline{Z}$ \\ 
    $\hat{\underline{w}}_{\text{MAP}} = \mu_{\underline{w} \mid \underline{Y}} = (X^T X + \lambda I)^{-1} X^T \underline{y}$ \\ 
    *$\underline{w} \sim \mathcal{N} (\underline{0}, \tau^2 I)$, $\underline{Z} \sim \mathcal{N} (\underline{0}, \sigma^2 I)$ \\
    *$E[\hat{\underline{w}} (\underline{Y})] \rightarrow \underline{w}$ as $n \rightarrow \infty$ \\
    *Note: Matching it to canonical form. 

    \textcolor{green}{\textbf{Covar. Mat:}} $\Sigma_{\underline{w} \mid \underline{y}} = \left( \frac{1}{\sigma^2} X^T X + \frac{1}{\tau^2} I \right)^{-1} \preceq \tau^2 I$ \\ 
    -Less uncertainty in $\underline{w}$ w/ more data. As $n \uparrow$, $\Sigma_{\underline{w} \mid \underline{y}} \downarrow$ 

    \textcolor{blue}{\textbf{Bayesian Prediction}} Given some new $\underline{x}'$ (test data sample), find its label $y'$ 

    \textcolor{green}{\textbf{Plug-In Approx:}} $\hat{Y}' = \underline{x}'^T \hat{\underline{w}}_{\text{MAP}}(\mathcal{D}) + Z'$ \\
    *$\mathcal{D}$: Training data set, $Z' \sim \mathcal{N}(0, \sigma^2)$: Noise \\

    \textcolor{green}{\textbf{Bayesian Prediction:}} Use $Y' = \underline{x}'^T \underline{w} + Z'$ and \\
    $f_{\underline{w} \mid \underline{Y}} (\underline{w} \mid \underline{y}) = \mathcal{N}(\mu_{\underline{w} \mid \underline{Y}}, \Sigma_{\underline{w} \mid \underline{Y}})$ to return $f_{Y'} (y' \mid \mathcal{D})$ where
    $Y'$ is Gaussian given $\mathcal{D}$ w/ \\
    *$\mu_{Y' \mid \mathcal{D}} = \underline{x}'^T \mu_{\underline{w} \mid \underline{Y}}$ \\
    *$\sigma^2_{Y' \mid \mathcal{D}} = \underline{x}'^T \Sigma_{\underline{w} \mid \underline{Y}} \underline{x}' + \sigma^2$

    \textcolor{red}{\textbf{Linear Classification (Hyp. Test):}} 

    \textcolor{blue}{\textbf{Binary Logistic Regression:}} Estimate $\underline{w}$ s.t. it is a soft decision \\
    $P_{Y \mid \underline{X}}(1 \mid \underline{x}) = \frac{P_{\underline{X} \mid Y}(\underline{x} \mid 1) P_Y(1)}{P_{\underline{X} \mid Y} (\underline{x} \mid 0) P_Y(0) + P_{\underline{X} \mid Y} (\underline{x} \mid 1) P_Y(1)}$ \\
    $P_{Y \mid \underline{X}}(1 \mid \underline{x}) = \frac{1}{1 + e^{-\alpha}} = \sigma(\alpha)$ \\
    *$P_{Y \mid \underline{X}}(0 \mid \underline{X}) = 1 - \sigma(\alpha) = \frac{1}{1 + e^{\alpha}} = \sigma(-\alpha)$ \\
    *$\alpha = \log \frac{P_{\underline{X} \mid Y} (\underline{x} \mid 1) P_Y(1)}{P_{\underline{X} \mid Y} (\underline{x} \mid 0) P_Y(0)} = \underline{w}^T \underline{x}$ \\
    -$\alpha \rightarrow \infty \implies$ more likely to be in class 1 \\
    -$\alpha \rightarrow -\infty \implies$ more likely to be in class 0. \\
    -$\alpha = 0 \implies$ equally likely to be in class 0 or 1.

    \textcolor{green}{\textbf{Non-Linear Trans.}} Use $\sigma(\underline{w}^T \underline{\phi}(\underline{x}))$ 

    \textcolor{green}{\textbf{ML Estimator:}} Given $\mathcal{D} = \{(\underline{x}_i, y_i)\}, i=1,\ldots,n$, then $\hat{\underline{w}}_{\text{ML}} = \arg \min_{\underline{w}} - \sum_{i=1}^n \log P_{Y \mid \underline{X}}(y_i \mid \underline{x}_i, \underline{w})$

    \textcolor{black}{\textbf{Cross Entropy}} b/w actual $y_i$ and $P_{Y \mid \underline{X}} (\cdot \mid \underline{x}_i, \underline{w})$ is \\
    $P_{Y \mid \underline{X}}(y_i \mid \underline{x}_i, \underline{w}) = \sum_{i=1}^n - \left(y_i \log P(1 \mid \underline{x}_i, \underline{w}) + (1 - y_i) \log P(0 \mid \underline{x}_i, \underline{w})\right)$ \\ 
    *Note: Measures the distance between 2 distributions. \\
    *Dropped the subscripts.

    \textcolor{black}{\textbf{Gradient Descent:}} No closed-form soln. so use GD.

    \textcolor{green}{\textbf{MAP Estimator:}} Given $\mathcal{D} = \{(\underline{x}_i, y_i)\}, i=1,\ldots,n$, then \\
    $\hat{\underline{w}}_{\text{MAP}} = \arg \min_{\underline{w}} - \sum_{i=1}^n \log P_{Y \mid \underline{X}}(y_i \mid \underline{x}_i, \underline{w}) + \lambda ||\underline{w}||^2$ \\
    *$\underline{w} \sim \mathcal{N}(\underline{\mu}, \Sigma)$: Prior on $\underline{w}$ \\
    *Necessary: B/c same boundary $\underline{w}^T \underline{x} = 0$ for any scaling of $\underline{w}$.

    \textcolor{blue}{\textbf{Multiclass Logistic Regression:}} $Y \in \{1,2,\ldots,C\}$, then use softmax fn $P_Y(k \mid \underline{x}, \underline{w}_1, \ldots, \underline{w}_C) = \frac{e^{\underline{w}_k^T \underline{x}}}{\sum_{c=1}^C e^{\underline{w}_c^T \underline{x}}}$ \\
    *$W=[\underline{w}_1, \ldots, \underline{w}_C] \in \mathbb{R}^{D \times C}$: Weights matrix 

    \textcolor{green}{\textbf{ML Estimator:}} Given $\mathcal{D} = \{(\underline{x}_i, y_i)\}, i=1,\ldots,n$, then $\hat{W}_{\text{ML}} = \arg \min_W - \sum_{i=1}^n \log P(y_i \mid \underline{x}_i, W)$ 

    \textcolor{green}{\textbf{MAP Estimator:}} Given $\mathcal{D} = \{(\underline{x}_i, y_i)\}, i=1,\ldots,n$, then \\
    $\hat{W}_{\text{MAP}} = \arg \min_W - \sum_{i=1}^n \log P(y_i \mid \underline{x}_i, W) + \sum_{c=1}^C \lambda_c ||\underline{w}_c||^2$ 

    \textcolor{red}{\textbf{Markov:}} 

    \textcolor{blue}{\textbf{Notation:}} \\
    *$P[X_n \text{=} x_n,\ldots,X_0\text{=} x_0] = P(x_n,\ldots,x_0)$ \\
    *Index the possible values of $X_n$ w/ integers $0,1,2,\ldots$ 
    
    \textcolor{blue}{\textbf{Markov Chain (Memoryless/Markovian Property):}}
    A sequence of discrete-valued RVs $X_0,X_1,\ldots$ is a (discrete-time) Markov chain if \\ 
    $P[\underbrace{X_{k+1} = x_{k+1}}_{\text{Future}} \mid \underbrace{X_k = x_k}_{\text{Present}}, \underbrace{X_{k-1} = x_{k-1},\ldots,X_0=x_0}_{\text{Past}}] = $ \\
    $P[X_{k+1} = x_{k+1} \mid X_k = x_k] \; \forall k, x_1,\ldots,x_{k+1}$ \\
    *\textbf{Markovian:} $P(x_n,\ldots,x_0) \text{=} P(x_n \mid x_{n-1}) \cdots P(x_1 \mid x_0) P(x_0)$ \\
    *\textbf{Equiv. Form:} $k+1 \rightarrow n_{k+1}$, $k \rightarrow n_k$ and so on \\
    for any $n_{k+1} > n_k > \cdots > n_0$ (i.e. farther in future/past) 

    \textcolor{blue}{\textbf{State Distribution:}} State distribution of the MC at time $n$ is $P_j(n) \equiv P[X_n = j], j=0,1,\ldots \; \mid \; \underline{P}(n) \equiv [P_0(n), P_1(n), \ldots]$ \\
    *Subscript: Value of $X_n$, Argument: Time step \\
    *Row vector NOT col vector.

    \textcolor{blue}{\textbf{Transition Probabilities:}} \\ 
    $P_{ij} (n,n+1) \equiv P[X_{n+1} = j \mid X_n = i] \; \forall i,j,n$ 

    \textcolor{green}{\textbf{Homogeneous MC:}} $P_{ij}(n,n+1) = P_{ij} \; \forall i,j,n$ \\
    *Time invariant, $P_{ij}$ does not depend on $n$

    \textcolor{blue}{\textbf{Transition Probability Matrix:}} $P = \begin{bmatrix}
        P_{00} & P_{01} & \cdots \\
        P_{10} & P_{11} & \cdots \\
        \vdots & \vdots & \ddots
    \end{bmatrix}$ 

    \textcolor{green}{\textbf{Notes:}} (1) \textbf{Stochastic Matrix:} (1) All entries of $P$ are non-negative and (2) each row sums to $1$: $\sum_{j} P_{ij} = 1 \; \forall i$ \\
    (2) State Dist. at time $n+1$: $\underline{P}(n) = \underline{P}(n-1) P$ \\
    *$\underline{P}(n) = \underline{P}(0) P^n$ in terms of initial distribution $\underline{P}(0)$\\
    (3) State Dist. at time $n+m$: $\underline{P}(n+m) = \underline{P}(m) P^n \; \forall n,m$

    \textcolor{blue}{\textbf{n-step Transition Probabilities:}} Stochastic matrix $P^n$ s.t. 
    $P_{ij}^{(n)} \equiv P[X_{k+n} = j \mid X_k = i]$ for $n \geq 0$ are the entries of $P^n$ 

    \textcolor{blue}{\textbf{Limiting Distribution}} A MC has a limiting distribution $\underline{q}$ if for any initial distribution $\underline{P}(0)$ \\
    $\underline{P}(\infty) \equiv \lim_{n \rightarrow \infty} \underline{P}(n) = \underline{q}$ or \\ 
    $\underline{P}(0)P^{\infty} \equiv \underline{P}(0) \lim_{n \rightarrow \infty} P^n = \underline{q}$ \\

    \textcolor{green}{\textbf{Theorem:}} A MC has a limiting distribution $\underline{q}$ iff \\ 
    $q_i = \lim_{n \rightarrow \infty} P_{ij}^{(n)} \; \forall i,j$\\
    *i.e. every row of $P^\infty$ equals $\underline{q}$ (row vector)

    \textcolor{blue}{\textbf{Steady State (Stationary) Distribution}} $\underline{\pi}$ is a steady state distribution of a MC if $\underline{\pi} = \underline{\pi} P$ \\
    *$1 = \sum_{j} \pi_j$

    \textcolor{green}{\textbf{Theorem:}} If a limiting dist. exists $\underline{q} = \underline{P}(\infty)$, then it is also a steady state dist. 

    \textcolor{blue}{\textbf{Ergodic:}} For a finite-state, irreducible, and aperiodic MC, then \\
    (1) Limiting dist. $\underline{q} = \lim_{n \rightarrow \infty} \underline{P}(n)$ exists and \\ 
    $q_j = \lim_{n \rightarrow \infty} P_{ij}^{(n)} \; \forall i,j$ \\
    (2) Steady state dist. $\underline{\pi}$ is unique. \\ 
    (3) $\underline{\pi} = \underline{q}$

    \textcolor{blue}{\textbf{How Fast Does $\underline{P}(n)$ Converge to $\underline{\pi}$?}} (1) $\underline{\pi}^T = \underline{\pi}^T P^T$ \\
    *$\underline{\pi}^T$ is an eigenvector of $P^T$ w/ eigenvalue $1$ \\
    (2) Suppose $P^T$ has eigenvectors $U \equiv [\underline{\pi}^T, \underline{u}_2, \ldots, \underline{u}_D]$ \\ 
    and eigenvalues $\Lambda \equiv \text{diag}[1, \lambda_2, \ldots, \lambda_D]$, then \\
    $P^T U = U \Lambda \implies P^T = U \Lambda U^{-1}$ so $n$ times \\
    $P^n = (P^T)^n = (U \Lambda U^{-1})^n = U \Lambda^n U^{-1}$ \\
    Therefore, $\Lambda^n = \text{diag}[1, \lambda_2^n, \ldots, \lambda_D^n]$ \\
    (3) For ergodic MC, $P^n \rightarrow [\underline{\pi}, \ldots, \underline{\pi}]^T$ (i.e. rank 1) \\
    Therefore, \# of non-zero eigenvalues is $1$, so the rest of the eigenvalues must be $|\lambda_i| < 1 \; \forall i \geq 2$ s.t. $\Lambda^n = \text{diag}[1,0,\ldots,0]$

    \textcolor{green}{\textbf{Rate of Convergence:}} Depends on the 2nd largest eigenvalue of $P^T$ i.e. $(\lambda_2)^n$ is the rate of convergence.

    \textcolor{red}{\textbf{Bayesian Network:}} Network of RVs $X_1,\ldots,X_n$ w/ directed edges \\ 
    *\textbf{Not State-Transition Diagram:} 1 RV w/ different values w/ different probabilities to each value. \\
    *\textbf{Fully Connected Graph:} No special dependency structure, so doesn't give additional info as we can write joint dist. from defn. (true for any graph). \\
    *\textbf{Non-Fully Connected Graph (Absence of Links):} Conveys useful info about the dependency structure. \\
    *\textbf{Purpose:} Clarify the dependencies among a set of RVs to simplify the calculation of joint probabilities. 

    \textcolor{blue}{\textbf{Factorization of Joint Dist.}} Suppose the dependencies among RVs can be represented by a DAG, then \\
    $P(x_1,\ldots,x_n) = \prod_{i=1}^N P(x_i \mid \text{pa}\{X_i\})$ 

    \textcolor{blue}{\textbf{Topological Ordering:}} Often index the RVs s.t. each child has an index greater than those of the parents. 

    \textcolor{green}{\textbf{Fact:}} Every DAG has at least one topological ordering. 

    \textcolor{blue}{\textbf{Conditional Independence:}} $A \perp B \mid C$ if \\
    (1) $P(a,b \mid c) = P(a \mid c) P(b \mid c) \; \forall a,b,c$ (i.e. A and B are indep. given C) \\
    (2) $P(a \mid b,c) = P(a \mid c) \; \forall a,b,c$ (i.e. B gives no add. info about A given C)

    \textcolor{green}{\textbf{Common Cause (Tail to Tail):}} $A \perp B \mid C$, o.w. $A \not\perp B$ 

    \textcolor{green}{\textbf{Causal Chain (Head to Tail or Tail to Head):}} $A \perp B \mid C$, o.w. $A \not\perp B$

    \textcolor{green}{\textbf{Common Effect (Head to Head):}} $A \perp B$, o.w. $A \not\perp B \mid C \text{ or its descendants}$ \\
    *\textbf{Explaining Away:} If $A \rightarrow B \leftarrow C$, then if you observe $B$, then the other cause $A$ is less likely to be the cause for the effect $B$.

    \textcolor{blue}{\textbf{Directed Separation (D-seperation):}} For non-overlapping subsets of RVs $\mathcal{A}, \mathcal{B}, \mathcal{C}$, if all undirected paths blocked, then $\mathcal{A}$ and $\mathcal{B}$ are \textbf{d-separated} by $\mathcal{C}$, i.e. $\mathcal{A} \perp \mathcal{B} \mid \mathcal{C}$

    \textcolor{green}{\textbf{Blocked Path:}} An undirected path is blocked if it includes a node s.t. \\
    1. The node is head-to-tail or tail-to-tail (Cases 1 and 2) and it is in set $\mathcal{C}$ \\
    2. The node is head-to-head, but neither itself nor any of its descendants are in set $\mathcal{C}$ (Case 3)

    \textcolor{blue}{\textbf{Markov Boundary (Blanket):}} Minimal set of RVs $\mathcal{M}$ that isolate $X_i$ from all the remaining RVs, i.e. \\
    $X_i \perp \mathcal{N} \setminus (\{X_i\} \cup \mathcal{M}) \mid \mathcal{M}$ \\
    *$\mathcal{N}$: Set of all RVs \\
    *$\mathcal{M} = \text{parents} \cup \text{children} \cup \text{co-parents}$: Blocks all paths b/w $X_i$ and the remaining nodes.

    \textcolor{red}{\textbf{Markov Random Field:}} Represent RVs as an undirected graph s.t. conditional independence $\mathcal{A} \perp \mathcal{B} \mid \mathcal{C}$ hold iff all paths b/w $\mathcal{A}$ and $\mathcal{B}$ so through $\mathcal{C}$. \\
    *\textbf{Markov blanket of $X_i:$} $= \text{set of neighbours of } X_i$ \\
    *\textbf{No Order:} No longer a way to order the RVs.

    \textcolor{blue}{\textbf{Factorization of Joint Dist.}} 2 \textbf{non-neighbouring} nodes (RVs) are conditionally indep. given the set of nodes that separate them: \\
    $P(x_i, x_j \mid C) = P(x_i \mid C) P(x_j \mid C) \; \forall i,j$ \\
    *\textbf{i.e.} $x_i$ and $x_j$ should not appear in the same factor. 

    \textcolor{blue}{\textbf{Clique:}} A set of nodes s.t. there is link b/w any pair of them

    \textcolor{green}{\textbf{Maximal Clique:}} A clique s.t. we cannot add another node and maintain a clique. 

    \textcolor{blue}{\textbf{Hammersley-Clifford Theorem:}} Let $\underline{x}_C$ denote the values of RVs in set $C$. Any strictly postiive dist. $P(\underline{x})$ that satisfies a Markov random field can be factorized as \\
    $P(\underline{x}) = \frac{1}{Z} \prod_{C} \psi_C(\underline{x}_C) = \frac{1}{Z} e^{-\sum_C E(\underline{x}_C)}$ \\
    *$Z= \sum_{\underline{x}} \prod_C \psi_C (\underline{x}_C)$: Normalization constant \\
    *$\Pi_C$: Product of all maximal cliques \\
    *$\psi_C(\underline{x}_C) = e^{-E(\underline{x}_C)}$: Potential function over the clique $C$ (not necessarily a prob.) \\
    *$E(\underline{x}_C)$: Energy function over the clique $C$ 

    \textcolor{blue}{\textbf{Conversion from Bayesian Net to Markov Random Field}} Always possible, but some dependency structure will be lost \\
    (1) For each clique $C$, define a potential function $\psi_C$ \\
    (2) For each pair of nodes $i,j$ that are not connected by an edge, add a clique $C$ that contains $i,j$ and define $\psi_C$ \\
    (3) For each node $i$, add a clique $C$ that contains $i$ and its parents and define $\psi_C$

    \textcolor{red}{\textbf{Hidden Markov Model (HMM):}} \\
    *\textbf{State-transition Prob.} \\ 
    $P(z_n \mid z_{n-1}) \equiv P[Z_n = z_n \mid Z_{n-1} = z_{n-1}], \; Z \leq n \leq N$ \\
    *\textbf{Initial State Dist.} $P(z_1) \equiv P[Z_1 = z_1]$ \\
    *\textbf{Emission Prob.} \\
    $P(x_n \mid z_n) \equiv P[X_n = x_n \mid Z_n = z_n], \; 1 \leq n \leq N$ \\
    *\textbf{Note:} Can be continuous (density).

    \textcolor{blue}{\textbf{Notes:}} \\
    (1) $Z_n$ nodes are head-to-tail or tail-to-tail and $Z_1,\ldots,Z_N$ are unobserved\\
    $\implies$ No indep. among $X_n$'s, also $\{X_n\}$ are not a MC.

    (2) Latent var. $Z_1,\ldots,Z_N$ are MC $\implies$ $Z_{n+1} \perp Z_{n-1} \mid Z_n$ \\
    $\implies$ $\{X_1,X_2\} \perp \{X_3,\ldots,X_N\} \mid \{Z_3\}$

    \textcolor{blue}{\textbf{Common Problems}} \\ 
    1. Given HMM, find $P(\underline{x})$ for any $\underline{x} = \{x_1,\ldots,x_N\}$ \\
    2. Given HMM and $\underline{x}$, find most likely $z_n$ or sequence of states $\underline{z} = \{z_1,\ldots,z_N\}$ 

    \textcolor{blue}{\textbf{Message Passing Algos:}} Given HMM, find $P(\underline{x}) \; \forall \underline{x}$ \\
    \textcolor{green}{\textbf{Forward:}} \\
    $\alpha(z_n) \equiv P[\underbrace{X_1 = x_1,\ldots,X_n=x_n}_{\text{obs so far}}, \underbrace{Z_n = z_n}_{\text{cur. state}}]$, $1 \leq n \leq N$\\
    $\alpha(z_n) \equiv P[x_1,\ldots,x_n,z_n]$ \\
    *$\alpha(z_1) = P(x_1,z_1) = P(z_1)P(x_1 \mid z_1)$ \\
    $\boxed{\alpha(z_n) = P(x_n \mid z_n) \sum_{z_{n-1}} P(z_n \mid z_{n-1}) \alpha(z_{n-1})}$ \\
    $\boxed{\alpha(z_N) = P(\underline{x}, z_N) \implies P(\underline{x}) = \sum_{z_N} \alpha(z_N)}$ \\
    * $\{\alpha(z_n)\}_{z_n = 1,\ldots,K}$: Message from $Z_n$ to $Z_{n+1}$ \\
    *\textbf{Complexity:} $O(K^2 N)$ \\
    *$O(K)$ for each $\alpha(z_n)$, $O(K^2)$ for message at time $n$

    \textcolor{green}{\textbf{Backward:}} \\
    $\beta(z_n) \equiv P[\underbrace{X_{n+1} = x_{n+1},\ldots,X_N=x_N}_{\text{future obs}} \mid \underbrace{Z_n = z_n}_{\text{cur. state}}]$ \\
    $\beta(z_n) \equiv P[x_{n+1},\ldots,x_N \mid z_n]$ \\
    *$\beta(z_N) = 1 \; \forall z_n$ \\
    $\boxed{\beta(z_n) = \sum_{z_{n+1}} P(x_{n+1} \mid z_{n+1}) P(z_{n+1} \mid z_n) \beta(z_{n+1})}$ \\
    $\boxed{\beta(z_1) \text{=} P(x_2,\ldots,x_n \mid z_1) \implies P(\underline{x}) = \sum_{z_1} P(z_1) P(x_1 \mid z_1) \beta(z_1)}$ \\
    *$\{\beta(z_n)\}_{z_n = 1,\ldots,K}$: Message from $Z_n$ to $Z_{n-1}$ \\
    *\textbf{Complexity:} $O(K^2 N)$ \\
    *$O(K)$ for each $\beta(z_n)$, $O(K^2)$ for message at time $n$

    \textcolor{green}{\textbf{Forward Backward (Same Time):}} $\alpha (z_n) \beta(z_n) = P(\underline{x}, z_n)$ \\
    $P(\underline{x}) = \sum_{z_n} \alpha(z_n) \beta(z_n) \; \forall n$ 

    \textcolor{blue}{\textbf{Approx. Algo:}} Given HMM and $\underline{x}$, find most likely $z_n$\\
    $\gamma (z_n) \equiv P(z_n \mid \underline{x}), \; 1 \leq n \leq N$ \\
    $\boxed{\gamma(z_n) = \frac{\alpha(z_n) \beta(z_n)}{\sum_{z_n'} \alpha(z_n') \beta(z_n')}}$ \\
    $\boxed{z_n^* = \arg \max_{z_n} \gamma(z_n)}$ \\
    *\text{Complexity:} $O(K^2 N)$ 

    \textcolor{green}{\textbf{Scaling:}} $\alpha(z_n),\beta(z_n)$ can be small for large/small $n$ \\ 
    \textbf{1. Forward:} \\
    $\hat{\alpha}(z_n) \equiv \frac{\alpha(z_n)}{P(x_1,\ldots,x_n)} = P(z_n \mid x_1,\ldots,x_n)$ \\
    *Does not shrink as $n \uparrow$ \\
    $c_n \equiv P(x_n \mid x_1,\ldots,x_{n-1})$ \\
    Then $P(x_1,\ldots,x_n) = \prod_{m=1}^n c_m$ \\
    $\boxed{\hat{\alpha}(z_n) = \frac{1}{c_n} P(x_n \mid z_n) \sum_{z_{n-1}} P(z_n \mid z_{n-1}) \hat{\alpha}(z_{n-1})}$ \\
    $\boxed{c_n = \sum_{z_n} P(x_n \mid z_n) \sum_{z_{n-1}} P(z_n \mid z_{n-1}) \hat{\alpha}(z_{n-1})}$ 

    \textcolor{green}{\textbf{2. Backward:}} \\
    $\hat{\beta}(z_n) = \frac{\beta(z_n)}{\prod_{m=n+1}^N c_m}$ \\
    $\boxed{\hat{\beta}(z_n) = \frac{1}{c_{n+1}} \sum_{z_{n+1}} P(x_{n+1} \mid z_{n+1}) P(z_{n+1} \mid z_n) \hat{\beta}(z_{n+1})}$ \\
    $\boxed{c_{n+1} = \sum_{z_{n+1}} P(x_{n+1} \mid z_{n+1}) P(z_{n+1} \mid z_n) \hat{\beta}(z_{n+1})}$

    \textcolor{green}{\textbf{3. Forward-Backward}} $\boxed{\gamma (z_n) = \hat{\alpha}(z_n) \hat{\beta}(z_n)}$ \\

    \textcolor{blue}{\textbf{Forward-Backward Algo}} 
    *Have to fwd, then bwd pass. \\
    0. $c_1 = P(x_1) = \sum_{z_1} P(z_1) P(x_1 \mid z_1)$ \\
    $\hat{\alpha}(z_1) = \frac{1}{c_1} P(z_1) P(x_1 \mid z_1) $ \\
    1. Fwd message passing to compute $\hat{\alpha}(z_n)$ and $c_n$, $2 \leq n \leq N$ \\
    2. $\hat{\beta}(z_N) = \beta(z_N) =1$ \\
    Bwd message passing to compute $\hat{\beta}(z_n)$, $1 \leq n \leq N-1$ \\
    3. $\gamma(z_n) = \hat{\alpha}(z_n) \hat{\beta}(z_n)$ \\
    4. $z_n^* = \arg \max_{z_n} \gamma(z_n) \forall n$

    \textcolor{blue}{\textbf{Viterbi Algo:}} Given HMM and $\underline{x}$, find most likely $\underline{z}$ \\
    $\hat{\underline{z}} = \arg \max_{\underline{z}} P(\underline{z} \mid \underline{x}) = \arg \max_{\underline{z}} P(\underline{x}, \underline{z})$? \\
    


    
    }
\end{paracol}

\end{document}