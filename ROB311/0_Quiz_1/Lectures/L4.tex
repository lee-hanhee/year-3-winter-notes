\subsection{Probably Approximately Correct (PAC) Estimations}

\subsubsection{Hoeffding's Inequality}
\begin{definition}
    For any $\epsilon > 0$,
    \begin{equation}
        \mathbb{P}(|\nu - \mu| \geq \epsilon) \leq 2e^{-2\epsilon^2N}
    \end{equation}
    \begin{itemize}
        \item $\mu$: Probabillity of an event.
        \item $\nu$: Relative frequency in a sample size $N$.
        \item $\mu \overset{?}{\approx} \nu $: $\mu$ is probably approximately equal to $\nu$. As $N \rightarrow \infty$: $\nu \rightarrow \mu$
        \item $\epsilon$: Tolerance (i.e. how close we want $\nu$ to be to $\mu$).
        \begin{itemize}
            \item $\epsilon \rightarrow 0$: $\nu = \mu$
        \end{itemize}
    \end{itemize}
\end{definition}

\begin{warning}
    Approximate the true distribution with high probability by taking a large enough sample size (i.e. empirical distribution converges to true distribution).
\end{warning}

\subsection{PAC Learning}
\subsubsection{Error}
\begin{definition}
    \begin{itemize}
        \item \textbf{Out-Sample Error:}
        \begin{equation*}
            E_{\text{out}} = \mathbb{P}[f \neq h]
        \end{equation*}
        \item \textbf{In-Sample Error:}
        \begin{equation*}
            E_{\text{in}} = \frac{1}{N} \sum_{i=1}^{N} \mathbb{I}[f(x^{(i)}) \neq h(x^{(i)})]
        \end{equation*}
    \end{itemize}
\end{definition}

\subsubsection{Union Bound Theorem}
\begin{theorem}
    \begin{equation*}
        \mathbb{P} \left[E_1 \lor \cdots \lor E_M \right] \leq \sum_{i=1}^{M} \mathbb{P}[E_i]
    \end{equation*}
\end{theorem}

\begin{warning}
    If the events are highly correlated, then the union bound is not tight.
\end{warning}

\subsubsection{Generalization of Hoeffding's Inequality}
\begin{definition}
    Assuming that $h$ is chosen from a set of hypotheses $\mathcal{H}$, derive a (loose) upper-bound on $|E_{\text{out}} - E_{\text{in}}|$:
    \begin{align*}
        \mathbb{P} \left[ \bigvee_{h \in \mathcal{H}} \left( |E_{\text{out}} - E_{\text{in}}(h)| > \varepsilon \right) \right]
        &\leq \sum_{h \in \mathcal{H}} \mathbb{P} \left[ |E_{\text{out}} - E_{\text{in}}(h)| > \varepsilon \right] \\
        &\leq \sum_{h \in \mathcal{H}} 2e^{-2\varepsilon^2 N} \\
        &= 2 |\mathcal{H}| e^{-2\varepsilon^2 N} 
    \end{align*}
    \begin{itemize}
        \item Endow $\mathcal{F}$ w/ prob. distribution, $P : \mathcal{X} \to [0,1]$, then 
        \begin{itemize}
            \item $E_{\text{out}}$ is analogous to $\mu$ 
            \item $E_{\text{in}}(h)$ is analogous to $\nu$. 
        \end{itemize}
    \end{itemize}
\end{definition}

\begin{notes}
    \begin{itemize}
        \item $E_{\text{in}}(h) \stackrel{?}{\approx} E_{\text{out}}$ requires small $|\mathcal{H}|$ (generalization)
        \item $E_{\text{in}}(h) \approx 0$ requires large $|\mathcal{H}|$ (discrimination)
    \end{itemize}
\end{notes}

\begin{example}
    \begin{enumerate}
        \item \textbf{Given:} An opaque box containing \textcolor{red}{red} and \textcolor{blue}{blue} balls. Take $N$ IID samples.
        \begin{itemize}
            \item $\mu$: Probability of drawing a \textcolor{blue} ball (unknown).
            \item $\nu$: Relative frequency of \textcolor{blue} balls in the sample (known).
        \end{itemize}
        \item \textbf{Problem 1:} What is $\nu$ in this case? 8 balls total, 5 are blue. 
        \item \textbf{Solution 1:} $\nu = \frac{5}{8}$
        \item \textbf{Problem 2:} How to partition $\mathcal{F}$ into regions where \textcolor{blue}{$f=h$} and \textcolor{red}{$f \neq h$}?
        \item \textbf{Solution 2:} 
        \customFigure[0.5]{../Images/L5_8.png}{LS $h$, MS $f$}
        \item \textbf{Problem 3:} What is the out-sample error?
        \item \textbf{Solution 3:} In words, the probability of the hypothesis being wrong.
        \customFigure[0.3]{../Images/L5_9.png}{}
        \item \textbf{Problem 4:} What is the in-sample error given this sample of 11 balls s.t. $f=h$, $1$ ball s.t. $f \neq h$?
        \item \textbf{Solution 4:} $E_{\text{in}} = \frac{1}{12}$
        
    \end{enumerate}
\end{example}



