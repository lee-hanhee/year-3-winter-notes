\documentclass{article}
\usepackage{style}
\title{ECE353 Lectures}
\author{Hanhee Lee}
\lhead{ECE353}
\rhead{Hanhee Lee}

\begin{document}
\maketitle

\tableofcontents
\newpage

\section{Prologue}
\begin{summary}
    \begin{itemize}
        \item 
    \end{itemize}    
\end{summary}

\subsection{Three OS Concepts}
\begin{definition}
    \begin{enumerate}
        \item \textbf{Virtualization:} Share one resource by mimicking multiple independent copies.
        \item \textbf{Concurrency:} Handle multiple things happening at the same time.
        \item \textbf{Persistence:} Retain data consistency even without power. 
    \end{enumerate}
\end{definition}

\subsection{OS Manages Resources}
\begin{definition}
    Insert picture. 
\end{definition}

\subsection{Program}
\begin{definition}
    A file containing all the instructions and data required to run. 
\end{definition}

\subsection{Process (Abstraction)}
\begin{definition}
    An instance of running a program.
\end{definition}

\subsubsection{Basic Requirements for a Process}
\begin{definition}
    Insert picture w/ virtual memory. 
\end{definition}

\subsection{Process (Abstraction)}

\subsubsection{Static}
\begin{definition}
    Only able to use the global variable in the current C file.     
\end{definition}

\subsubsection{Motivation for Virtualization}
\begin{motivation}
    How to run two different programs at the same time?
        Insert code. 
        \begin{itemize}
            \item Was the address of local the same b/w 2 processes? Different address in physical memory b/w different processes.
            \item Was the address of global the same b/w 2 processes? Same address in physical memory b/w different processes, but uses virtual memory.
            \item What else may be needed for a process?  
        \end{itemize}
\end{motivation}

\begin{warning}
    Local variables are stored on the stack. 
\end{warning}

\subsubsection{Does the OS allocate different stacks for each process?}
\begin{definition}
    The stacks for each process need to be in physical memory. One option is the operating system just allocates any unused memory for the stack. 
    \begin{itemize}
        \item 
    \end{itemize}
\end{definition}

\subsubsection{What about global variables?}
\begin{definition}
    The compiler needs to pick an address (random) for each variable when you compile.
    \begin{itemize}
        \item What if we had a global registry of addresses? Impossible (too much space and know memory addresses ahead of time).
    \end{itemize}
\end{definition}

\subsubsection{Potential Memory Layout for Multiple Processes}
\begin{definition}
    Insert picture.
\end{definition}

\begin{warning}
    Process 1 wants to use more memory than its allocated.
\end{warning}
\newpage

\section{Uninformed/Informed Search Algorithms}
\begin{summary}
    \begin{itemize}
        \item Not responsible for proofs, but know when to use each algorithm.
    \end{itemize}
\end{summary}
\subsection{Setup}
\begin{definition} In a search problem, it is assumed that: 
    \begin{itemize}
        \item There is only one agent (us).
        \item For each state, $s \in S$, we have a discrete set of actions, $\mathcal{A}(s)$.
        \item The transition resulting from a move, $(s, a)$, is deterministic; the resulting state is $tr(s, a)$.
        \item $cst(s, a, tr(s, a))$ is our cost for the transition, $(s, a, tr(s, a))$.
        \item We want to realize a path that minimizes our cost.
    \end{itemize}
    
    A search problem may have no solutions, in which case, we define the solution as \texttt{NULL}.
\end{definition}

\subsection{Search Graphs}
\begin{definition}
    In a search graph (a graph representing a search problem):
    \begin{itemize}
        \item $S$ is defined by the vertices.
        \item $\mathcal{G}$ is a subset of the vertices.
        \item $s^{(0)}$ is some vertex.
        \item $tr(\cdot, \cdot)$ and $\mathcal{T}$ are defined by the edges.
        \item $cst(\cdot, \cdot, \cdot)$ is defined by the edge weights.
    \end{itemize}
\end{definition}

\subsection{Path Trees}
\begin{definition}
    A search algorithm explores a tree of possible paths. 
    \begin{itemize}
        \item In such a tree, each node represents the path from the root to itself.
        \begin{itemize}
            \item The node may also include other info (such as the path's origiin, cost, etc).
        \end{itemize}
    \end{itemize}
\end{definition}

\subsection{Search Algorithms}
\begin{definition}
    All search algorithms follow the template below:

\begin{lstlisting}
$\mathcal{O} \gets \{(\langle \rangle, 0)\}$ (*\hfill $\triangleright$ initialize a set of open nodes*) 
SEARCH($\mathcal{O}$)
\end{lstlisting}
\begin{itemize}
    \item $\langle \rangle$ is the empty path, and $0$ is the cost of the empty path.
\end{itemize}

\begin{lstlisting}
procedure SEARCH($\mathcal{O}$)
    if $\mathcal{O} = \emptyset$ then
        return NULL  (*\hfill $\triangleright$ the search algorithm failed to find a path to a goal*)
    $n \gets \textsc{Remove}(\mathcal{O})$ (*\hfill $\triangleright$ "explore" a node $n$*)
    if $\textsc{dst}(n) \in \mathcal{G}$ then
        return $n$ (*\hfill $\triangleright$ the search algorithm found a path to a goal*)
    for $n' \in \textsc{chl}(n)$ do
        $\mathcal{O} \gets \mathcal{O} \cup \{n'\}$ (*\hfill $\triangleright$ "expand" $n$ and "export" its children*)
    SEARCH($\mathcal{O}$)
\end{lstlisting}
\begin{itemize}
    \item Explore: Remove a node from the open set.
    \item Exapnd: Generate the children of the node.
    \item Export: Add the children to the open set.
\end{itemize}

\end{definition}

\begin{warning}
    The key difference is in the order that \textsc{Remove}($\cdot$) removes nodes.
\end{warning}

\subsubsection{Characteristics of a Search Algorithm}
\begin{definition}
    We want to choose \texttt{REMOVE(·)} so that the algorithm exhibits the following characteristics:

    \begin{center}
        \begin{tabular}{|p{3cm}|p{9cm}|}
        \hline
        \textbf{Characteristic} & \textbf{Description} \\ \hline
        Halting & Terminates after finitely many nodes explored \\ \hline
        Sound & Returned (possibly NULL) solution is correct \\ \hline
        Complete & Halting and sound when a non-NULL solution exists \\ \hline
        Optimal & Returns an optimal solution when multiple exist \\ \hline
        Time Efficient & Minimizes the nodes \textbf{explored}/expanded/exported \\ \hline
        Space Efficient & Minimizes the nodes simultaneously open \\ \hline
        \end{tabular}
    \end{center}
    \vspace{1em}   
    \begin{itemize}
        \item Will be using explored for time efficiency.
    \end{itemize} 
    \vspace{1em}

    The characteristics of the algorithm also depend on several properties of the path tree over which it searches. These properties include:
    \begin{itemize}
        \item Branching factor: $b$ ($b < \infty$), the maximum number of children a node can have.
        \item Depth: $d$, the length of the longest path.
        \item Length of the shortest solution: $l^*$
        \item Cost of the cheapest solution: $c^*$
        \item Cost of the cheapest edge: $\epsilon$ 
    \end{itemize}

    We want to choose \texttt{REMOVE($\cdot$)} so that the algorithm exhibits the aforementioned characteristics for as many path trees as possible.

\end{definition}

\subsubsection{Breadth First Search (BFS)}
\begin{definition}
    Explores the least-recently expanded open node first.
    \begin{center}
        \begin{tabular}{|p{3cm}|p{3cm}|}
        \hline
        \textbf{Property} & \textbf{Description} \\ \hline
        Halting & $d < \infty$ \newline non-NULL \\ \hline
        Sound & always \\ \hline
        Complete & always \\ \hline
        Optimal & constant cst \\ \hline
        Time & $b^{l^*}$ \\ \hline
        Space & $b^{l^* + 1}$ \\ \hline
        \end{tabular}
    \end{center}
\end{definition}

\subsubsection{Depth First Search (DFS)}
\begin{definition}
    Explores the most-recently expanded open node first.
    \begin{center}
        \begin{tabular}{|p{3cm}|p{3cm}|}
        \hline
        \textbf{Property} & \textbf{Description} \\ \hline
        Halting & $d < \infty$ \\ \hline
        Sound & always \\ \hline
        Complete & $d < \infty$ \\ \hline
        Optimal & never \\ \hline
        Time & $b^d$ \\ \hline
        Space & $bd$ \\ \hline
        \end{tabular}
    \end{center}    
\end{definition}

\subsubsection{Iterative Deepening DFS (IDDFS)}
\begin{definition}
    Same as DFS but with iterative deepening.
    \begin{center}
        \begin{tabular}{|p{3cm}|p{3cm}|}
        \hline
        \textbf{Property} & \textbf{Description} \\ \hline
        Halting & always \\ \hline
        Sound & always \\ \hline
        Complete & always \\ \hline
        Optimal & constant cst \\ \hline
        Time & $b^{l^*}$ \\ \hline
        Space & $bl^*$ \\ \hline
        \end{tabular}
    \end{center}    
\end{definition}

\subsubsection{Cheapest-First Search (CFS)}
\begin{definition}
    Explores the cheapest open node first.
    \begin{center}
        \begin{tabular}{|p{3cm}|p{3cm}|}
        \hline
        \textbf{Property} & \textbf{Description} \\ \hline
        Halting & $d < \infty$ \newline non-NULL \\ \hline
        Sound & yes \\ \hline
        Complete & $\epsilon > 0$ \\ \hline
        Optimal & $\epsilon > 0$ \\ \hline
        Time & $b^{c^*/\epsilon}$ \\ \hline
        Space & $b^{c^*/\epsilon + 1}$ \\ \hline
        \end{tabular}
    \end{center}    
\end{definition}

\subsection{Modifications to Search Algorithms}
\subsubsection{Depth-Limiting}
\begin{definition}
    Depth limit of $d_{\text{max}}$ to any search algorithm by modifying \texttt{SEARCH($\cdot$)} as follows:
\begin{lstlisting}
procedure SEARCHDL($\mathcal{O}$, $d_{\text{max}}$):
    if $\mathcal{O} = \emptyset$ then
        return NULL (*\hfill $\triangleright$ the search algorithm failed to find a path to a goal*)
    $n \leftarrow \text{REMOVE}(\mathcal{O})$ (*\hfill $\triangleright$ "explore" a node, $n$*)
    if dst($n$) $\in \mathcal{G}$ then
        return $n$ (*\hfill $\triangleright$ the search algorithm found a path to a goal*)
    for $n' \in \text{chl}(n)$ do (*\hfill $\triangleright$ "expand" $n$ and "export" its children*)
        if len($n'$) $\leq d_{\text{max}}$ then (*\hfill $\triangleright$ unless the child is too long*)
            $\mathcal{O} \leftarrow \mathcal{O} \cup \{n'\}$
    SEARCHDL($\mathcal{O}$, $d_{\text{max}}$)
\end{lstlisting}

\end{definition}

\subsubsection{Iterative Deepening}
\begin{definition}
    Iteratively increase the depth-limit, $d_{\max}$, to any search algorithm w/ depth-limiting, by placing \texttt{SEARCHDL($\cdot$)} in a wrapper, \texttt{SEARCHID($\cdot$)}:
\begin{lstlisting}
procedure SEARCHID():
    $n \leftarrow \text{NULL}$
    $d_{\text{max}} \leftarrow 0$
    (*$\triangleright$ while a solution has not been found, reset the open set, run the search algorithm, then increase the depth-limit*)
    while $n = \text{NULL}$ do
        $\mathcal{O} \leftarrow \{(\langle \rangle, 0)\}$
        $n \leftarrow \text{SEARCHDL}(\mathcal{O}, d_{\text{max}})$
        $d_{\text{max}} \leftarrow d_{\text{max}} + 1$
    return $n$
\end{lstlisting}
    
\end{definition}

\begin{warning}
    Increasing $d_{\text{max}}$ can be done in different ways.
\end{warning}

\subsubsection{Cost-Limiting}
\begin{definition}
    Cost limit of $c_{\text{max}}$ to any search algorithm by modifying \texttt{SEARCH($\cdot$)} as follows:

\begin{lstlisting}
procedure SEARCHCL($\mathcal{O}$, $c_{\text{max}}$):
    if $\mathcal{O} = \emptyset$ then
        return NULL (*\hfill $\triangleright$ the search algorithm failed to find a path to a goal*)
    $n \leftarrow \text{REMOVE}(\mathcal{O})$ (*\hfill $\triangleright$ "explore" a node, $n$*)
    if dst($n$) $\in \mathcal{G}$ then
        return $n$ (*\hfill $\triangleright$ the search algorithm found a path to a goal*)
    for $n' \in \text{chl}(n)$ do (*\hfill $\triangleright$ "expand" $n$ and "export" its children*)
        if cst($n'$) $\leq c_{\text{max}}$ then (*\hfill $\triangleright$ unless the child is too expensive*)
            $\mathcal{O} \leftarrow \mathcal{O} \cup \{n'\}$
    SEARCHCL($\mathcal{O}$, $c_{\text{max}}$)
\end{lstlisting}

\end{definition}

\subsubsection{Iterative-Inflating}
\begin{definition}
    Iteratively increase the cost limit, $c_{\text{max}}$, to any search algorithm with cost-limiting, by placing \texttt{SEARCHCL($\cdot$)} in a wrapper, \texttt{SEARCHII($\cdot$)}:

\begin{lstlisting}
procedure SEARCHII():
    $n \leftarrow \text{NULL}$
    $c_{\text{max}} \leftarrow 0$
    (*$\triangleright$ while a solution has not been found, reset the open set, run the search algorithm, then increase the cost-limit*)
    while $n = \text{NULL}$ do
        $\mathcal{O} \leftarrow \{(\langle \rangle, 0)\}$
        $n \leftarrow \text{SEARCHCL}(\mathcal{O}, c_{\text{max}})$
        $c_{\text{max}} \leftarrow c_{\text{max}} + \epsilon$
    return $n$
\end{lstlisting}

\end{definition}

\begin{warning}
    Increasing $c_{\text{max}}$ can be done in different ways.
\end{warning}

\subsubsection{Intra-Path Cycle Checking}
\begin{definition}
    Do not expand a path if it is cyclic. Modify \texttt{SEARCH($\cdot$)} as follows:

\begin{lstlisting}
procedure SEARCH($\mathcal{O}$):
    if $\mathcal{O} = \emptyset$ then
        return NULL
    $n \leftarrow \text{REMOVE}(\mathcal{O})$
    if dst($n$) $\in \mathcal{G}$ then
        return $n$
    for $n' \in \text{chl}(n)$ do (*\hfill $\triangleright$ "expand" $n$ and "export" its children*)
        if not CYCLIC($n'$) then (*\hfill $\triangleright$ unless the child is cyclic*)
            $\mathcal{O} \leftarrow \mathcal{O} \cup \{n'\}$
    SEARCH($\mathcal{O}$)
\end{lstlisting}
\begin{itemize}
    \item Optimately of an algorithm is preserved provided $\epsilon>0$.
\end{itemize}

\end{definition}

\subsubsection{Inter-Path Cycle Checking}
\begin{definition}
    We modify \texttt{SEARCH($\cdot$)} as follows:

\begin{lstlisting}
procedure SEARCH($\mathcal{O}$, $\mathcal{C}$):
    if $\mathcal{O} = \emptyset$ then
        return NULL
    $n \leftarrow \text{REMOVE}(\mathcal{O})$
    $\mathcal{C} \leftarrow \mathcal{C} \cup \{n\}$ (*\hfill $\triangleright$ add $n$ to the closed set*)
    if dst($n$) $\in \mathcal{G}$ then
        return $n$
    for $n' \in \text{chl}(n)$ do (*\hfill $\triangleright$ "expand" $n$ and "export" its children*)
        if $n' \notin \mathcal{C}$ then (*\hfill $\triangleright$ unless the child's destination is closed*)
            $\mathcal{O} \leftarrow \mathcal{O} \cup \{n'\}$
    SEARCH($\mathcal{O}$, $\mathcal{C}$)
\end{lstlisting}

and then call the algorithm as follows:

\begin{lstlisting}[mathescape=true, escapeinside={(*}{*)}, numbers=left, frame=single]
$\mathcal{O} \leftarrow \{(\langle \rangle, 0)\}$
$\mathcal{C} \leftarrow \emptyset$ (*\hfill $\triangleright$ initialize a set of closed vertices*)
SEARCH($\mathcal{O}$, $\mathcal{C}$)
\end{lstlisting}

\end{definition}

\subsection{Informed Search Algorithms}
\subsubsection{Estimated Cost}
\begin{definition}
    $\text{ecst}(\cdot)$, to estimate the total cost to a goal given a path, $p$, based on the following:
    \begin{itemize}
        \item Cost of path $p$: $\text{cst}(p)$
        \item Estimate of the extra cost needed to get to a goal from $\text{dst}(p)$: $\text{hur} : S \to \mathbb{R}_+$
        \begin{itemize}
            \item $\text{hur}(s)$ estimates the cost to get to $\mathcal{G}$ from $s$ and $\text{hur}(p)$ means $\text{hur}(\text{dst}(p))$.
        \end{itemize}
    \end{itemize}
\end{definition}

\begin{example}
    Some common choices for $\text{ecst}(\cdot)$ include:
    \begin{enumerate}
        \item $\text{ecst}(p) = \text{hur}(p)$; called nearest-first search (NFS)
        \item $\text{ecst}(p) = \text{cst}(p) + \text{hur}(p)$; called A$^*$ (A-star)
    \end{enumerate}
\end{example}

\subsection{Characteristics of an Informed Search Algorithm}
\begin{definition}
    \begin{enumerate}
        \item Heuristic: $\text{hur}(\cdot)$
        \item Cost estimation: $\text{ecst}(\cdot)$
    \end{enumerate}
\end{definition}
\subsubsection{Heuristics}


\subsubsection{Heuristic-First Search (HFS)}

\subsubsection{A-Star Search (A*)}

\subsubsection{Iterative Inflating A-Star Search (IIA*)}

\subsubsection{Designing Heuristics via Problem Relaxation}

\subsubsection{Combining Heuristics}

\subsection{Anytime Search Algorithms}

\subsection{Formulating a Search Problem}







\newpage

\section{Constraint Satisfaction Problems}
\subsection{2 RVs}
\begin{notes}
    RVs are neither random nor a variable. 
    \begin{equation*}
        \underline{Z} = (X,Y)
    \end{equation*}
    \customFigure[0.5]{../Images/L3_0.png}{Mapping of RVs}
\end{notes}

\subsection{Joint PMF/PDF}
\begin{definition}
    \begin{equation}
    P_{X,Y}(x, y) = P[X = x, Y = y]
    \end{equation}
    
    \begin{equation}
    f_{X,Y}(x, y) = \frac{\partial^2}{\partial x \partial y} F_{X,Y}(x, y)
    \end{equation}
    
    \begin{equation}
    P[(X, Y) \in A] = \int \int_{(x, y) \in A} f_{X,Y}(x, y) \, dx \, dy
    \end{equation}
\end{definition}

\begin{example} Jointly Gaussian RVs $X$ and $Y$ with ($\mu_1, \mu_2, \sigma_1^2, \sigma_2^2, \rho$)
    \[
    f_{X,Y}(x, y) = \frac{1}{2\pi \sigma_1 \sigma_2 \sqrt{1-\rho^2}} 
    \exp \left\{ 
    -\frac{1}{2(1-\rho^2)} 
    \left[ 
    \left(\frac{x-\mu_1}{\sigma_1}\right)^2 
    - 2\rho \left(\frac{x-\mu_1}{\sigma_1}\right) \left(\frac{y-\mu_2}{\sigma_2}\right) 
    + \left(\frac{y-\mu_2}{\sigma_2}\right)^2 
    \right] 
    \right\}
    \]
\end{example}

\subsection{Expectations}
\begin{definition}
    \[
    E[g(X, Y)] = \int_{-\infty}^{\infty} \int_{-\infty}^{\infty} g(x, y) f_{X,Y}(x, y) \, dx \, dy
    \]
\end{definition}

\begin{notes}
    \begin{itemize}
        \item $g(X,Y)$ is also an RV, but inside the integral or sum, you use $x$ and $y$ as dummy variables to vary through the values of the RVs.
    \end{itemize}
\end{notes}

\subsubsection{Correlation}
\begin{definition}
    \begin{equation}
        E[XY]
    \end{equation}
\end{definition}

\subsubsection{Covariance}
\begin{definition}
    \begin{equation}
        \text{Cov}[X, Y] = E[(X - \mu_X)(Y - \mu_Y)] = E[XY] - \mu_X \mu_Y = E[XY] - E[X]E[Y]
    \end{equation}
\end{definition}

\begin{notes}
    \begin{itemize}
        \item Mean shifted to 0.
    \end{itemize}
\end{notes}

\subsubsection{Correlation Coefficient}
\begin{definition}
    \begin{equation}
        \rho_{X,Y} = E \left[ \left(\frac{X - \mu_X}{\sigma_X} \right) \left( \frac{Y - \mu_Y}{\sigma_Y} \right) \right] = \frac{\text{Cov}[X, Y]}{\sigma_X \sigma_Y}
    \end{equation}
    \begin{itemize}
        \item $|\rho_{X,Y}| \leq 1$
    \end{itemize}
\end{definition}

\begin{notes}
    \begin{itemize}
        \item Mean shifted to 0 and normalized by the standard deviation.
    \end{itemize}
\end{notes}

\subsection{Marginal PMF/PDF}
\begin{definition}
    \begin{equation}
    P_X(x) = \sum_{j=1}^{\infty} P_{X,Y}(x, y_j), \quad P_Y(y) = \sum_{j=1}^{\infty} P_{X,Y}(x_j, y)
    \end{equation}
    
    \begin{equation}
    f_X(x) = \int_{-\infty}^{\infty} f_{X,Y}(x, y) \, dy, \quad f_Y(y) = \int_{-\infty}^{\infty} f_{X,Y}(x, y) \, dx
    \end{equation}
\end{definition}

\begin{notes}
    \begin{itemize}
        \item Total probability theorem is being used here.
    \end{itemize}
\end{notes}

\begin{example} Jointly Gaussian $X$ and $Y$:
    \begin{align*}
        f_X(x) &= \int_{-\infty}^{\infty} f_{X,Y}(x, y) \, dy \\
               &= \dots \quad (\text{completing the square}) \\
               &= \frac{1}{\sqrt{2\pi} \sigma_1} e^{-\frac{(x-\mu_1)^2}{2\sigma_1^2}}, \quad \text{marginally Gaussian}
    \end{align*}
    \begin{itemize}
        \item Gaussian RVs has a property that the PDF of a single variable is equal to the marginal Gaussian of two variables.
    \end{itemize}
\end{example}

\subsection{Conditional PMF/PDF}
\begin{definition}
    \begin{equation}
    P_{X|Y}(x|y) \triangleq P[X = x | Y = y] = \frac{P_{X,Y}(x, y)}{P_Y(y)}
    \end{equation}
    
    \begin{equation}
    f_{X|Y}(x|y) \triangleq \frac{f_{X,Y}(x, y)}{f_Y(y)}
    \end{equation}
\end{definition}

\subsection{Bayes' Rule}
\begin{definition}
    \begin{equation}
    P_{Y|X}(x|y) = \frac{P_{X,Y}(x, y)}{P_X(x)} = \frac{P_{X|Y}(x|y) P_Y(y)}{\sum_{j=1}^\infty P_{X,Y}(x, y_j) P_Y (y_j)}
    \end{equation}
    
    \begin{equation}
    f_{Y|X}(y|x) = \frac{f_{X,Y}(x, y)}{f_X(x)} = \frac{f_{X|Y}(x|y) f_Y(y)}{\int_{-\infty}^\infty f_{X|Y}(x|y') f_Y(y') \, dy'}
    \end{equation}  
\end{definition}

\subsection{Independent vs. Uncorrelated vs. Orthogonal}
\begin{definition} 
    \begin{enumerate}
        \item Independent:
        \begin{equation}
        f_{X|Y}(x|y) = f_X(x) \; \forall y
        \Leftrightarrow 
        f_{X,Y}(x, y) = f_X(x) f_Y(y) 
        \end{equation}
        \item Uncorrelated:
        \begin{equation}
        \text{Cov}[X, Y] = 0 \quad \Leftrightarrow \quad \rho_{X,Y} = 0
        \end{equation}
        \item Orthogonal:
        \begin{equation}
        E[XY] = 0
        \end{equation}
    \end{enumerate}
\end{definition}

\begin{theorem}
    If independent, then uncorrelated.
\end{theorem}

\begin{derivation}
    \begin{align*}
    \text{Independent} & \implies E[XY] = \int_{-\infty}^{\infty} \int_{-\infty}^{\infty} x y f_{X,Y}(x, y) \, dx \, dy \\
    &= \int_{-\infty}^{\infty} \int_{-\infty}^{\infty} x y f_X(x) f_Y(y) \, dx \, dy \\
    &= \left( \int_{-\infty}^{\infty} x f_X(x) \, dx \right) \left( \int_{-\infty}^{\infty} y f_Y(y) \, dy \right) \\
    &\implies E[XY] = E[X] E[Y] \\
    &\implies \text{Cov}[X, Y] = 0, \quad \text{uncorrelated} \\
    &\not\Leftarrow \text{in general.}
    \end{align*}
\end{derivation}

\begin{example} Jointly Gaussian RVs $X$ and $Y$: If uncorrelated, i.e. $\rho_{X,Y} = 0$, then $X$ and $Y$ are independent.
    \begin{align*}
    f_{X,Y}(x, y) &= \frac{1}{2\pi \sigma_1 \sigma_2} 
    \exp \left\{ 
    -\frac{1}{2} 
    \left[ 
    \left(\frac{x-\mu_1}{\sigma_1}\right)^2 
    + 
    \left(\frac{y-\mu_2}{\sigma_2}\right)^2 
    \right] 
    \right\} \\
    &= \frac{1}{\sqrt{2\pi} \sigma_1} e^{-\frac{(x-\mu_1)^2}{2\sigma_1^2}} 
    \cdot 
    \frac{1}{\sqrt{2\pi} \sigma_2} e^{-\frac{(y-\mu_2)^2}{2\sigma_2^2}} \\
    &= f_X(x) f_Y(y) \quad \text{independent}
    \end{align*}
\end{example}

\subsection{Conditional Expectation}
\begin{definition}
    \begin{equation}
        E[Y] = E[E[Y|X]]
    \end{equation}
    \begin{equation}
        E[h(Y)] = E[E[h(Y)|X]]
    \end{equation}
\end{definition}

\begin{notes}
    \begin{itemize}
        \item $E[E[Y|X]]$ is w.r.t. $X$.
        \item $E[Y|X]$ is w.r.t. $Y$.
    \end{itemize}
\end{notes}

\begin{derivation}
    \begin{align*}
    E[Y] &= \int_{-\infty}^\infty \int_{-\infty}^\infty y f_{X,Y}(x, y) \, dx \, dy \\
            &= \int_{-\infty}^\infty \int_{-\infty}^\infty y f_{Y|X}(y|x) f_X(x) \, dx \, dy \\
            &= \int_{-\infty}^\infty \left( \int_{-\infty}^\infty y f_{Y|X}(y|x) \, dy \right) f_X(x) \, dx \\
            &= \int_{-\infty}^\infty E[Y|X=x] f_X(x) \, dx \quad \text{(using the total probability theorem)} \\
            &= \int_{-\infty}^\infty g(x) f_X(x) \, dx \\
            &= E[g(X)] \\ 
            &= E[E[Y|X]].
    \end{align*}
\end{derivation}

\begin{example}
    \begin{enumerate}
        \item \textbf{Given:} An unknown voltage. \( X \sim \text{Uniform}(0,1) \). Measurement from a (bad) voltmeter: \( Y \sim \text{Uniform}(0, X) \).
    
        \begin{align*}
            f_X(x) &= 
            \begin{cases} 
                1, & 0 < x < 1 \\ 
                0, & \text{otherwise}
            \end{cases} \\
            f_{Y|X}(y|x) &= 
            \begin{cases} 
                \frac{1}{x}, & 0 < y < x \\ 
                0, & \text{otherwise}
            \end{cases}
        \end{align*}
        \begin{itemize}
            \item \textbf{Note:} Area under PDF is 1.
        \end{itemize}

        \customFigure[0.25]{../Images/L3_1.png}{Uniform Distribution of $X$}
        \customFigure[0.25]{../Images/L3_2.png}{Uniform Distribution of $Y$}

    
        \item \textbf{Expected Value (Average Reading of Bad Voltmeter):}
        \begin{align*}
            E[Y] &= E[E[Y|X]] \\
                 &= E\left[\frac{X}{2}\right] \quad \text{Since in the middle of 0 and x}\\
                 &= \frac{1}{2} \cdot E[X] \\ 
                 &= \frac{1}{2} \cdot \frac{1}{2} = \frac{1}{4} \quad \text{Since $E[X]$ (i.e. mean) is 0.5}
        \end{align*}
    
        \item \textbf{The Long Way:}
        \begin{align*}
            f_Y(y) &= \int_{-\infty}^{\infty} f_{Y|X}(y|x) f_X(x) \, dx \\
                   &= \int_{y}^1 f_{Y|X}(y|x) f_X(x) \, dx \\
                   &= \int_{y}^1 \frac{1}{x} \cdot 1 \, dx \\
                   &= -\ln y. \\
            E[Y] &= \int_{0}^1 y \cdot (-\ln y) \, dy = \dots = \frac{1}{4}
        \end{align*}
    
        \item \textbf{Question:} Suppose \( Y = \frac{1}{8} \). What is "best" given \( X \)? This will be the quesiton for the rest of the course.
    \end{enumerate}    
\end{example}
\newpage

\begin{center}
    \section*{Learning Problems}
\end{center}

\section{PAC Learning}
\begin{definition}
    Assume that there is some (unknown) relationship, 
    \begin{equation*}
        f: \mathcal{X} \rightarrow \mathcal{Y} \text{ s.t. } x \mapsto_f y
    \end{equation*}
    \begin{itemize}
        \item $\mathcal{X}$: Input Space
        \item $\mathcal{Y}$: Output Space (i.e. Information we desire about input)
    \end{itemize}
    \vspace{1em}

    Find $h: \mathcal{X} \rightarrow \mathcal{Y}$ (hypothesis) s.t. $h \approx f$, given some data about $f$: 
    \begin{equation*}
        \mathcal{D} = \left\{ \left(x^{(i)}, y_i\right), x^{(i)} \in \mathcal{X}, y_i = f\left(x^{(i)}\right) \in \mathcal{Y}, i = 1 \ldots N \right\}
    \end{equation*}

    \begin{itemize}
        \item $\text{in}(\mathcal{D}) = \{x \text{ s.t. } (x,y) \in \mathcal{D}\}$
        \item $\text{out}(\mathcal{D}) = \{y \text{ s.t. } (x,y) \in \mathcal{D}\}$
    \end{itemize}
\end{definition}

\subsection{Classification vs. Regression Problems}
\begin{definition}
    \begin{itemize}
        \item \textbf{Classification Problems:} $\mathcal{X} \subseteq \mathbb{R}^M$ and $\mathcal{Y} \subseteq \mathbb{N}$
        \item \textbf{Regression Problems:} $\mathcal{X} \subseteq \mathbb{R}^M$ and $\mathcal{Y} \subseteq \mathbb{R}$
    \end{itemize}
\end{definition}

\subsection{Feature Spaces}
\begin{definition}
    It is often easier to learn relationships from high-level features (instead of the raw input).
\end{definition}

\section{Learning Problems}
\begin{definition}
    In a learning problem, we assume that there is some (unknown) relationship, 
    \begin{equation*}
        f: \mathcal{X} \rightarrow \mathcal{Y}
    \end{equation*}
    s.t. $x \mapsto_f y$
    \vspace{1em}

    Find $h: \mathcal{X} \rightarrow \mathcal{Y}$ (hypothesis) s.t. $h \approx f$, given some data about $f$: 

    \begin{itemize}
        \item $\text{in}(\mathcal{D}) = \{x \text{ s.t. } (x,y) \in \mathcal{D}\}$
        \item $\text{out}(\mathcal{D}) = \{y \text{ s.t. } (x,y) \in \mathcal{D}\}$
    \end{itemize}
\end{definition}

\subsection{Classification vs. Regression Problems}
\begin{definition}
    \begin{itemize}
        \item \textbf{Classification Problems:} $\mathcal{X} \subseteq \mathbb{R}^n$ and $\mathcal{Y} \subseteq \mathbb{N}$
        \item \textbf{Regression Problems:} $\mathcal{X} \subseteq \mathbb{R}^n$ and $\mathcal{Y} \subseteq \mathbb{R}$
    \end{itemize}
\end{definition}

\subsection{Feature Spaces}
\begin{definition}
    It is often easier to learn relationships from high-level features (instead of the raw input).
\end{definition}

\subsection{Feasibility of Learning}
\begin{motivation}
    More than one function (hypothesis) may be consistent with the data.
\end{motivation}

\begin{notes}
    So it may appear that finding the correct one should be impossible. 
\end{notes}

\subsubsection{Probably Approximately Correct (PAC) Estimations}
\begin{example}
    Take $N$ i.i.d. samples (i.e. take out a ball from an urn, record its color, and put it back in).
    \begin{itemize}
        \item $\nu \rightarrow \mu$ (empirical distribution $\rightarrow$ true distribution) as $N \rightarrow \infty$
    \end{itemize}
\end{example}

\subsubsection{Hoeffding's Inequality}
\begin{definition}
    Let $\mu$ denote the probability of an event, and $\nu$ denote its relative frequency in a sample size $N$. Then, for any $\epsilon > 0$,
    \begin{equation}
        P(|\nu - \mu| > \epsilon) \leq 2e^{-2\epsilon^2N}
    \end{equation}
    \begin{itemize}
        \item $\nu$: Relative frequency in the sample (known)
        \item $\mu$: Probabillity of drawing a blue ball (unknown)
        \item $N \rightarrow \infty$: $\nu \rightarrow \mu$
        \item $\epsilon$: How close we want $\nu$ to be to $\mu$
        \item $\epsilon \rightarrow 0$: Probability will be 1
        \item $\epsilon \rightarrow \infty$: $\nu \rightarrow \mu$
        \item $\mu \overset{?}{\approx} \nu $: $\mu$ is probably approximately equal to $nu$.
    \end{itemize}
\end{definition}

\begin{warning}
    We can approximate the true distribution with high probability by taking a large enough sample size, NOT guaranteeing that we can find the true distribution.
    \begin{itemize}
        \item Don't need to know where this theorem comes from.
    \end{itemize}
\end{warning}

Consider determining the class of a randomly chosen target point. If we ask a K-ary question about the points in $\mathcal{D}$

\subsubsection{PAC Learning}

\subsection{Decision Trees}

\subsubsection{Structure of a Decision Tree}



\section{Decision Trees}
\begin{summary}
    \begin{itemize}
        \item What stategies can help a NN converge when training?
        \item What hyperparameters does a NN architecture have?
        \item How can we optimize parameters without gradients? 
        \item DL requires a lot of data, what can we do when data is scarce?
    \end{itemize}
\end{summary}

\section{Blackbox Optimization}
\begin{motivation}
    Also known as derivative free optimization, as the derivative is unknown, so you have to use derivative-free or heuristic methods. 
    \begin{equation*}
        x^* = \arg\min_{x \in X} f(x)
    \end{equation*}
    \customFigure[0.5]{../Images/L5_2.png}{}
\end{motivation}

\subsection{Parameters \& Hyperparameters}
\begin{definition}
    Distinction b/w model setting elements and tuning knobs.
    \begin{itemize}
        \item \textbf{Parameters:} Learnable parameters $(W,b)$
        \begin{itemize}
            \item Opt: Gradient Descent
        \end{itemize}
        \item \textbf{Hyperparameters:} Non-differentiable parameters (i.e. discrete)
        \begin{itemize}
            \item E.g. Number of layers, hidden dim, activation, normalization, dropout, ...
            \item Opt: Heuristics
        \end{itemize}
        \customFigure[0.5]{../Images/L5_0.png}{}
    \end{itemize}
\end{definition}
\newpage

\begin{summary}
    \begin{center}
        \begin{tabular}{l}
        \toprule
        \textbf{Types} \\
        \midrule
        \textbf{Grid Search} \\
        \multicolumn{1}{p{\linewidth}}{
        \begin{itemize}
            \item Exhaustive evaluation across a predefined set of values.
            \customFigure[0.3]{../Images/L5_3.png}{}
        \end{itemize}} \\
        \midrule
        \textbf{Coordinate Descent} \\
        \multicolumn{1}{p{\linewidth}}{
        \begin{itemize}
            \item Optimize each hyperparameter one at a time. 
            \customFigure[0.3]{../Images/L5_4.png}{}
        \end{itemize}} \\
        \midrule
        \textbf{Grad-Student Descent} \\
        \multicolumn{1}{p{\linewidth}}{
        \begin{itemize}
            \item Manual and ad-hoc, i.e. "follow your heart"
            \customFigure[0.3]{../Images/L5_5.png}{}
        \end{itemize}} \\
        \midrule
        \textbf{Random Search} \\
        \multicolumn{1}{p{\linewidth}}{
        \begin{itemize}
            \item Sampling hyperparameter configurations from defined distributions.
            \customFigure[0.3]{../Images/L5_6.png}{}
        \end{itemize}} \\
        \bottomrule
        \end{tabular}
    \end{center}
\end{summary}

\section{Bayesian Optimization}
\begin{definition}
    A principled sequential approach for efficient global optimization
    \vspace{1em}

    \textbf{Ingredients:}
    \begin{itemize}
        \item \textbf{Function, $f(x)$}: The numerical values we want to optimize.
        \item \textbf{Space to optimize, $X$}: Parameters or decisions or degrees of freedom to explore.
        \item \textbf{Bayesian model, $g(x)$}: Provides prediction ($\mu$) and uncertainty ($\sigma$).
        \item \textbf{Acquisition function, $A(\mu, \sigma)$}: A strategy to trade off exploration \& exploitation.
    \end{itemize}
\end{definition}

\subsection{Surrogate Model}
\begin{notes}
    Let's approximate the expensive function $f(x)$ with a cheaper function $g(x)$ to model prediction ($\mu$) and uncertainty ($\sigma$).
    \begin{itemize}
        \item Kernel models
        \item Gaussian processes (GP)
        \item Gradient boosted trees
        \item Neural networks
    \end{itemize}
\end{notes}

\subsubsection{Acquisition Function}
\begin{notes}
    Let's mix exploitation and exploration; sometimes, it pays off to explore areas where we have little information.
    \begin{itemize}
        \item Acquisition functions encapsulate the heuristic of what to sample next, how useful is unobserved data?
        \item E.g. Expected Improvement (EI), Probability of Improvement (PI), Upper Confidence Bound (UCB), ...
        \item $\mu$: exploitation 
        \item $\sigma$: exploration 
    \end{itemize}
\end{notes}

\subsection{BayesOpt Loop}
\begin{definition}
    Iterative process of modelling and sampling. 
    \begin{enumerate}
        \item Set a termination criteria (budget, iterations, maxima)
        \item Evaluate $f(x)$ on initial set of points (random)
        \item \textbf{Loop:} while criteria is not met:
        \begin{enumerate}
            \item Update surrogate model on all data
            \item Optimize acquisition function to find a maxima ($x_{\text{new}}$)
            \item Evaluate $f(x_{\text{new}}$)
        \end{enumerate}
    \end{enumerate}
\end{definition}

\begin{warning}
    \begin{itemize}
        \item Define your hyperparameter space (bounds, datatypes, etc.)
        \item Simplify it. 
        \item Use a platform to launch monitor and launch models. 
        \item \textbf{Libraries:} Optuna, Ray, BoTorch, Ax...
    \end{itemize}
\end{warning}

\begin{example}
    \begin{itemize}
        \item 2 random points
        \item Criteria: 10 evaluations.
        \item Normalize $f(x)$ to a smaller range b/c gradients are sensitive to scale.
        \item Repeat for 10 iterations:
        \begin{itemize}
            \item Pick points that maximizes acquistion function.
            \item Update surrogate model w/ the new point and its evaluation, i.e. $f(x_i)$ to get more certainty and better predictions.
        \end{itemize}
    \end{itemize}
    \vspace{1em}

    Look at L5.
\end{example}


\newpage

\end{document}